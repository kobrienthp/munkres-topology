\subsection{Connected Spaces}

\begin{exercise}[ID=3.23.1]
  Let $\mathcal{T}$ and $\mathcal{T}'$ be two topologies on $X$.
  If $\mathcal{T}' \supset \mathcal{T}$, what does connectedness of $X$ in one topology imply about connectedness in the other?
\end{exercise}
%
\begin{solution}
  \textbf{Claim.} If $X$ is connected in the topology $\mathcal{T'}$, then it is connected in the topology $\mathcal{T}$.
  In general, connectedness of $X$ in $\mathcal{T}$ does not imply connectedness of $X$ in $\mathcal{T}'$.
  \begin{proof}
    We proceed by proving the contrapositive.
    Let $X = A \cup B$ be a separation of $X$ in the topology $\mathcal{T}$ so that $X$ is disconnected in this topology.
    By definition of a separation, $A$ and $B$ are open sets in $\mathcal{T}$.
    Since $\mathcal{T}'$ is finer than $\mathcal{T}$, the sets $A$ and $B$ are also open in $\mathcal{T}'$.
    It follows that they also form a separation of $X$ in the topology $\mathcal{T}'$.
    Thus, $X$ is disconnected also in $\mathcal{T}'$.
    It follows that if $X$ is connected in the topology $\mathcal{T'}$, then it is connected in the topology $\mathcal{T}$.

    Consider the case where $X$ is a two-point space and let $\mathcal{T}$ be the trivial topology on $X$.
    Since there exists only a single nonempty open set in $\mathcal{T}$, namely $X$, there exists no separation of $X$.
    In the discrete topology $\mathcal{T}'$, however, we have the separation $X = \{a\} \cup \{b\}$ of $X$, where $a \neq b$ are the two elements of $X$.
    Since $\mathcal{T}'$ is finer than $\mathcal{T}$, it follows that connectedness in one topology does not necessarily imply connectedness in a finer topology.
  \end{proof}
\end{solution}
\newpage

\begin{exercise}[ID=3.23.2]
  Let $\{A_n\}$ be a sequence of connected subspaces of $X$, such that $A_n \cap A_{n+1} \neq \emptyset$ for all $n$.
  Show that $\bigcup A_n$ is connected.
\end{exercise}
%
\begin{solution}
  \begin{proof}
    Suppose by way of contradiction that $\bigcup A_n = U \cap V$ is a separation of $\bigcup A_n$, \ie, $U$ and $V$ are disjoint nonempty open sets whose union is $\bigcup A_n$.
    Since for all $n \in \ZZ_+$, $A_n$ is a subspace of $X$, $\bigcup A_n$ is a subspace of $X$ and $A_n \subset \bigcup A_n$, we have that $A_n$ is a connected subspace of $\bigcup A_n$.
    In particular, $A_1$ is a connected subspace of $\bigcup A_n$.
    By Theorem 2.23.2, that means that $A_1$ must lie entirely within either $U$ or $V$.
    Without loss of generality, assume that $A_1 \subset U$.

    Suppose that $A_n$ sits entirely within $U$ for some $n \in \ZZ_+$.
    As a connected subspace of $\bigcup A_n$, it must be the case that $A_{n+1}$ also lies either entirely within $U$ or $V$.
    Since $A_n \cap A_{n+1} \neq \emptyset$, it must be that $A_{n+1}$ intersects $U$ and, thus, must lie entirely within $U$.
    By induction, we then have that $A_n$ is disjoint from $V$ for all $n \in \ZZ_+$.
    But then $V = V \cap \bigcup A_n = \emptyset$, which contradicts our assumption that $\bigcup A_n$ is not connected.
  \end{proof}
\end{solution}
\newpage

\begin{exercise}[ID=3.23.3]
  Let $\{A_\alpha\}$ be a collection of connected subspaces of $X$;
  let $A$ be a connected subspace of $X$.
  Show that if $A \cap A_\alpha \neq \emptyset$ for all $\alpha$, then $A \cup \left(\bigcup A_\alpha\right)$ is connected.
\end{exercise}
%
\begin{solution}
  \begin{proof}
    Let $\{B_\alpha\}$ be the collection of sets defined by $B_\alpha = A \cup A_\alpha$, for all $\alpha$.
    Since $A \cap A_\alpha \neq \emptyset$, by hypothesis, Theorem 2.23.3 implies that $B_\alpha$ is a connected subspace of $X$ for all $\alpha$.
    By construction, $\bigcap B_\alpha \supset A \neq \emptyset$.
    Thus, by Theorem 2.23.3,
    %
    \begin{equation*}
      \bigcup B_\alpha = \bigcup \left(A \cup A_\alpha\right) = A \cup \left(\bigcup A_\alpha\right)
    \end{equation*}
    %
    is connected.
  \end{proof}
\end{solution}
\newpage

\begin{exercise}[ID=3.23.4]
  Show that if $X$ is an infinite set, it is connected in the finite complement topology.
\end{exercise}
%
\begin{solution}
  \begin{proof}
    Suppose by way of contradiction that $X = U \cup V$ is a separation of the infinite set $X$ in the finite complement topology.
    By definition of a separation, $U$ and $V$ are disjoint, so that
    %
    \begin{equation*}
      U \cap V = (X \setminus U^c) \cap (X \setminus V^c) = X \setminus (U^c \cup V^c) = \emptyset,
    \end{equation*}
    %
    where $U^c$ and $V^c$ are the complements of $U$ and $V$, respectively.
    It follows that $U^c \cup V^c = X$.
    As $U$ and $V$ are open sets, both $U^c$ and $V^c$ are finite, implying that their union is also finite.
    This contradicts our assumption that $X$ is infinite, thus, $X$ must be connected in the finite complement topology.
  \end{proof}
\end{solution}
\newpage

\begin{exercise}[ID=3.23.7]
  Is the space $\RR_l$ connected?
  Justify your answer.
\end{exercise}
%
\begin{solution}
  \textbf{Claim.} The space $\RR_l$ is not connected.
  \begin{proof}
    We may write $\RR$ as the union of two sets $\RR = (-\infty, 0) \cup [0, \infty)$.
    Both $(-\infty, 0)$ and $[0, \infty)$ are clearly disjoint and nonempty.
    Furthermore, they are both open sets in the lower limit topology.
    Thus, $\RR = (-\infty, 0) \cup [0, \infty)$ is a separation of $\RR$ in the lower limit topology, from which it follows that the space $\RR_l$ is not connected.
  \end{proof}
\end{solution}
\newpage

\begin{exercise}[ID=3.23.10]
  Let $\{X_\alpha\}_{\alpha \in J}$ be an indexed family of connected spaces;
  let $X$ be the product space
  %
  \begin{equation*}
    X = \prod_{\alpha \in J} X_\alpha.
  \end{equation*}
  %
  Let $\ba = (a_\alpha)$ be a fixed point of $X$.
  %
  \begin{enumerate}[label={(\alph*)}, align=left, leftmargin=\parindent, listparindent=\parindent, labelwidth=0pt, itemindent=!]
    \item
      Given any finite subset $K$ of $J$, let $X_K$ denote the subspace of $X$ consisting of all points $\bx = (x_\alpha)$ such that $x_\alpha = a_\alpha$ for $\alpha \not\in K$.
      Show that $X_K$ is connected.
    \item
      Show that the union $Y$ of the spaces $X_K$ is connected.
    \item
      Show that $X$ equals the closure of $Y$;
      conclude that $X$ is connected.
  \end{enumerate}
\end{exercise}
%
\begin{solution}
  \begin{enumerate}[label={(\alph*)}, align=left, leftmargin=\parindent, listparindent=\parindent, labelwidth=0pt, itemindent=!]
    \item
      Given any finite subset $K$ of $J$, let $X_K$ denote the subspace of $X$ consisting of all points $\bx = (x_\alpha)$ such that $x_\alpha = a_\alpha$ for $\alpha \not\in K$.
      Show that $X_K$ is connected.
  \end{enumerate}
  \begin{proof}
    By hypothesis, we have that for all $\alpha \in K$, $X_\alpha$ is a connected space.
    Since $K$ is finite, by Theorem 3.23.6 we have that $\prod_{\alpha\in K} X_\alpha$ is also connected.
    We will show that $\prod_{\alpha\in K} X_\alpha$ is homeomorphic to the product space $X_K$, thus proving that $X_K$ is connected.

    Consider the map $f: \prod_{\alpha\in K} X_\alpha \rightarrow X_K$ given by the formula
    %
    \begin{equation*}
      \pi_\alpha(f(\bx)) =
      \begin{cases}
        x_\alpha  &\text{if } \alpha \in K, \\
        a_\alpha  &\text{if } \alpha \in J \setminus K,
      \end{cases}
    \end{equation*}
    %
    where $\pi_\alpha: X_K \rightarrow X_\alpha$ is the projection from $X_K$ onto the $\alpha^{\text{th}}$ component.
    Let $\bx, \by \in \prod_\alpha X_\alpha$.
    Then $f(\bx) = f(\by)$ implies that $\pi_\alpha(f(\bx)) = \pi_\alpha(f(\by))$ for all $\alpha \in J$.
    In particular, for $\alpha \in K$ this implies that $x_\alpha = y_\alpha$.
    It follows that $\bx = \by$ and, thus, $f$ is injective.

    Now let $\by \in X_K$ and let $\bx \in \prod_{\alpha\in K} X_\alpha$ with $x_\alpha = y_\alpha$ for all $\alpha \in K$.
    Then for $\alpha \in K$ we have that $\pi_\alpha(f(\bx)) = x_\alpha = y_\alpha$.
    Furthermore, for $\alpha \in J \setminus K$, $\pi_\alpha(f(\bx)) = a_\alpha = y_\alpha$.
    It follows that $f(\bx) = \by$ and, thus, $f$ is surjective.

    Note that for each $\alpha \in K$, the function $(\pi_\alpha \circ f): \prod_{\beta\in K} X_\beta \rightarrow X_\alpha, \bx \mapsto x_\alpha$ is merely the projection from $\prod_{\beta\in K} X_\beta$ onto its $\alpha^{\text{th}}$ component and is, thus, continuous.
    Furthermore, for $\alpha \in J \setminus K$, the function $(\pi_\alpha \circ f): \prod_{\beta\in K} X_\beta \rightarrow X_\alpha, \bx \mapsto a_\alpha$ is constant and, thus, continuous.
    Since for all $\alpha \in J$, the coordinate functions $\pi_\alpha \circ f$ are continuous, $f$ is continuous by Theorem 2.18.4.

    Consider the function $g: X_K \rightarrow \prod_{\alpha\in K} X_\alpha, \by \mapsto (y_\alpha)_{\alpha\in K}$.
    Then for any $\by \in X_K$,
    %
    \begin{equation*}
      (f \circ g)(\by) = f((y_\alpha)_{\alpha\in K}) = \by.
    \end{equation*}
    %
    Thus, $g = f\inv$.
    Since the coordinate functions of $g$ are simply the projections from $\prod_{\alpha\in K} X_\alpha$ onto its $\alpha^{\text{th}}$ component and, thus, continuous, we also have that $g$ is continuous.

    Thus, $f: \prod_{\alpha \in K} X_\alpha \rightarrow X_K$ is a continuous bijection with continuous inverse, \ie, a homeomorphism.
    It follows immediately that connectivity of $\prod_{\alpha \in K}$ implies connectivity of $X_K$.
  \end{proof}
  \bigskip

  \begin{enumerate}[label={(\alph*)}, align=left, leftmargin=\parindent, listparindent=\parindent, labelwidth=0pt, itemindent=!]
    \addtocounter{enumi}{1} 
    \item
      Show that the union $Y$ of the spaces $X_K$ is connected.
  \end{enumerate}
  \begin{proof}
    Let $K \subset J$ be an arbitrary finite subset.
    For any $\alpha \in K$, $\pi_\alpha(X_K) = X_\alpha$ which contains $a_\alpha$.
    Furthermore, for $\alpha \in J \setminus K$, $\pi_\alpha(X_K) = \{a_\alpha\}$.
    It follows that for any arbitrary choice of $K$ we have that $\ba \in X_K$.
    Thus, by Theorem 3.23.3,
    %
    \begin{equation*}
    Y = \left(\bigcup_{K \subset J} X_K\right),
    \end{equation*}
    %
    where $K$ is understood to be finite, is connected.
  \end{proof}
  \bigskip

  \begin{enumerate}[label={(\alph*)}, align=left, leftmargin=\parindent, listparindent=\parindent, labelwidth=0pt, itemindent=!]
    \addtocounter{enumi}{2} 
    \item
      Show that $X$ equals the closure of $Y$;
      conclude that $X$ is connected.
  \end{enumerate}
  \begin{proof}
    Let $\bx \in X$ and let $U = \prod_\alpha U_\alpha$ be a basic open neighborhood of $\bx$.
    Denote by $K$ the finite set of indices $\alpha$ for which $U_\alpha$ need not be $X_\alpha$.
    Let $\by \in X_K$ be defined by $y_\alpha = x_\alpha$ for $\alpha \in K$ and $y_\alpha = a_\alpha$ for $\alpha \in J \setminus K$.
    For $\alpha \in K$, we have that $y_\alpha = x_\alpha \in U_\alpha$ since $U_\alpha$ is a neighborhood of $x_\alpha$, whereas for $\alpha \in J \setminus K$ we trivially have that $y_\alpha = a_\alpha \in X_\alpha = U_\alpha$.
    Since $X_K \subset Y$, it follows that $\by \in U \cap Y$.
    Thus, $\bx \in \overline{Y}$ and so $X \subset \overline{Y}$.
    Since $X$ must also contain $\overline{Y}$, we have that $X = \overline{Y}$.
    By Theorem 3.23.4, since $Y$ is connected and $Y \subset X \subset \overline{Y}$, we have that $X$ is connected.
  \end{proof}
\end{solution}
\newpage

\begin{exercise}[ID=3.23.10]
  Let $p: X \rightarrow Y$ be a quotient map.
  Show that if each set $p\inv(\{y\})$ is connected, and if $Y$ is connected, then $X$ is connected.
\end{exercise}
%
\begin{solution}
  \begin{proof}
    Suppose by way of contradiction that $X = U \cup V$ is a separation of $X$.
    Since $p$ is a quotient map and $U$ and $V$ are open and nonempty by definition, we have that $p(U)$ and $p(V)$ are open nonempty subsets of $Y$.
    Furthermore, surjectivity of quotient maps implies that $Y = p(U) \cup p(V)$.

    Let $y \in Y$.
    By hypothesis, $p\inv(\{y\})$ is connected and, thus, must be contained entirely in either $U$ or $V$.
    It follows that $p(U) \cap p(V) = \emptyset$.
    Since $p(U)$ and $p(V)$ are nonempty, disjoint open sets, $Y = p(U) \cup p(V)$ is a separation of $Y$, contradicting our hypothesis that $Y$ is connected.
    Thus, $X$ is connected.
  \end{proof}
\end{solution}
\newpage

