\subsection{Connected Subspaces of the Real Line}

\begin{exercise}[ID=3.24.1]
  \begin{enumerate}[label={(\alph*)}, align=left, leftmargin=\parindent, listparindent=\parindent, labelwidth=0pt, itemindent=!]
    \item
      Show that no two of the spaces $(0, 1)$, $(0, 1]$, and $[0, 1]$ are homeomorphic.
    \item
      Suppose that there exist imbeddings $f: X \rightarrow Y$ and $g: Y \rightarrow X$.
      Show by means of an example that $X$ and $Y$ need not be homeomorphic.
    \item 
      Show that $\RR^n$ and $\RR$ are not homeomorphic if $n > 1$.
    \end{enumerate}
\end{exercise}
%
\begin{solution}
  \begin{enumerate}[label={(\alph*)}, align=left, leftmargin=\parindent, listparindent=\parindent, labelwidth=0pt, itemindent=!]
    \item
      Show that no two of the spaces $(0, 1)$, $(0, 1]$, and $[0, 1]$ are homeomorphic.
  \end{enumerate}
  %
  \begin{proof}
    Suppose by way of contradiction there is a homeomorphism
    %
    \begin{equation*}
      f: (0, 1] \rightarrow (0, 1).
    \end{equation*}
    %
    Then the map
    %
    \begin{align*}
      g: (0, 1) &\rightarrow (0, 1) \setminus f(\{1\}) \\
              x &\mapsto f(x)
    \end{align*}
    %
    is also a homeomorphism.
    But since $f(1) \in (0, 1)$ by hypothesis, $(0, 1) \setminus f(\{1\}) = (0, f(1)) \cup (f(1), 1)$ is disconnected.
    Since $(0, 1)$ is connected, its image under a continuous map must also be connected.
    This contradicts that $g$, and therefore $f$, are homeomorphisms.
    Thus, $(0, 1)$ and $(0, 1]$ are not homeomorphic.
    The proofs for the other spaces follow analogously.
  \end{proof}
  \bigskip

  \begin{enumerate}[label={(\alph*)}, align=left, leftmargin=\parindent, listparindent=\parindent, labelwidth=0pt, itemindent=!]
    \addtocounter{enumi}{1} 
    \item
      Suppose that there exist imbeddings $f: X \rightarrow Y$ and $g: Y \rightarrow X$.
      Show by means of an example that $X$ and $Y$ need not be homeomorphic.
  \end{enumerate}
  \begin{proof}
    We have seen in part (a) that the spaces $(0, 1)$ and $(0, 1]$ are not homeomorphic.
    Nevertheless, we can construct imbeddings between them.
    Consider the map
    %
    \begin{align*}
      f:  (0, 1]  &\rightarrow (0, 1) \\
                x & \mapsto x / 2.
    \end{align*}
    %
    Then the map $f': (0, 1] \rightarrow (0, 1/2]$ obtained by restricting the range of $f$ to its image set $f((0, 1])$, considered as a subspace of $(0, 1)$ is a homeomorphism.
    Thus, $f$ is an imbedding of $(0, 1]$ in $(0, 1)$.

    Consider now the map
    %
    \begin{equation*}
      g: (0, 1) \rightarrow (0, 1]
    \end{equation*}
    %
    given by the inclusion function.
    Then the map $g': (0, 1) \rightarrow (0, 1)$ obtained by restricting the range of $g$ to its image set $g((0, 1))$, considered as a subspace of $(0, 1]$ is just the identity map and, thus, a homeomorphism.
    Thus, $g$ is an imbedding of $(0, 1)$ in $(0, 1]$.
  \end{proof}
  \bigskip

  \begin{enumerate}[label={(\alph*)}, align=left, leftmargin=\parindent, listparindent=\parindent, labelwidth=0pt, itemindent=!]
    \addtocounter{enumi}{2} 
    \item 
      Show that $\RR^n$ and $\RR$ are not homeomorphic if $n > 1$.
  \end{enumerate}
  \begin{proof}
    Suppose by way of contradiction there is a homeomorphism
    %
    \begin{equation*}
      f: \RR^n \rightarrow \RR
    \end{equation*}
    %
    for some integer $n > 1$.
    Then the map
    %
    \begin{align*}
      g: \RR^n \setminus \{\mathbf{0}\} &\rightarrow \RR \setminus f(\{\mathbf{0}\}) \\
                                      x &\mapsto f(x)
    \end{align*}
    %
    is also a homeomorphism.
    But since $f(\mathbf{0}) \in \RR$ by hypothesis, $\RR \setminus f(\{\mathbf{0}\}) = (-\infty, f(\mathbf{0})) \cup (f(\mathbf{0}), \infty)$ is disconnected.
    Since we have seen previously that punctured euclidean space $\RR^n \setminus \{\mathbf{0}\}$ is connected for $n > 1$, its image under a continuous map must also be connected.
    This contradicts that $g$, and therefore $f$, are homeomorphisms.
    Thus, $\RR^n$ and $\RR$ are not homeomorphic if $n > 1$.
  \end{proof}
\end{solution}
\newpage

\begin{exercise}[ID=3.24.2]
  Let $f: S^1 \rightarrow \RR$ be a continuous map.
  Show there exists a point $x$ of $S^1$ such that $f(x) = f(-x)$.
\end{exercise}
%
\begin{solution}
  \begin{proof}
    Let $f: S^1 \rightarrow \RR$ be a continuous map.
    Then the map
    %
    \begin{align*}
      g: S^1  &\rightarrow \RR \\
            x &\mapsto f(x) - f(-x)
    \end{align*}
    %
    is also continuous.

    Assume by way of contradiction that for all $x \in S^1$, $f(x) \neq f(-x)$.
    Then $g(S^1) \subset \RR \setminus \{0\} = (-\infty, 0) \cup (0, \infty)$, which is disconnected.
    Since $g(x) = -g(-x)$, it must be that $g(S^1) \cap (-\infty, 0) \neq \emptyset$ and $g(S^1) \cap (0, \infty) \neq \emptyset$.
    But as $S^1$ is a connected space, this contradicts the fact that the image of $S^1$ under a continuous map is connected and, thus, must lie either entirely within $(-\infty, 0)$ or within $(0, \infty)$.
    Thus, if $f: S^1 \rightarrow \RR$ is a continuous map, then there exists a point $x$ of $S^1$ such that $f(x) = f(-x)$.
  \end{proof}
\end{solution}
\newpage

\begin{exercise}[ID=3.24.3]
  Let $f: X \rightarrow X$ be a continuous map.
  Show that if $X = [0, 1]$, there is a point $x$ such that $f(x) = x$.
  What happens if $X$ equals $[0, 1)$ or $(0, 1)$?
\end{exercise}
%
\begin{solution}
  \begin{proof}
    Let $f: [0, 1] \rightarrow [0, 1]$ be a continuous map.
    Then the map
    %
    \begin{align*}
      g: [0, 1] &\rightarrow [-1, 1] \\
              x &\mapsto f(x) - x
    \end{align*}
    %
    is also continuous.
    Since $g(0) = f(0) \geq 0$ and $g(1) = f(1) - 1 \leq 0$, by the intermediate value theorem there exists an $x \in [0, 1]$ satisfying $g(x) = 0$ and, thus, $f(x) = x$.

    We now consider what happens when $X = [0, 1)$ or $X = (0, 1)$.
    Consider the map
    %
    \begin{align*}
      f: X  &\rightarrow X \\
         x  &\mapsto \frac{x + 1}{2}.
    \end{align*}
    %
    The only solution to $f(x) = x$ in this case would be $x = 1$.
    However, $1 \not\in X$ for either case.
    Thus, for $X = [0, 1)$ or $X = (0, 1)$ a continous map $f: X \rightarrow X$ need not have a fixed point.
  \end{proof}
\end{solution}
\newpage

