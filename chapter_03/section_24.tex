\subsection{Connected Subspaces of the Real Line}

\begin{exercise}[ID=3.24.1]
  \begin{enumerate}[label={(\alph*)}, align=left, leftmargin=\parindent, listparindent=\parindent, labelwidth=0pt, itemindent=!]
    \item
      Show that no two of the spaces $(0, 1)$, $(0, 1]$, and $[0, 1]$ are homeomorphic.
    \item
      Suppose that there exist imbeddings $f: X \rightarrow Y$ and $g: Y \rightarrow X$.
      Show by means of an example that $X$ and $Y$ need not be homeomorphic.
    \item 
      Show that $\RR^n$ and $\RR$ are not homeomorphic if $n > 1$.
    \end{enumerate}
\end{exercise}
%
\begin{solution}
  \begin{enumerate}[label={(\alph*)}, align=left, leftmargin=\parindent, listparindent=\parindent, labelwidth=0pt, itemindent=!]
    \item
      Show that no two of the spaces $(0, 1)$, $(0, 1]$, and $[0, 1]$ are homeomorphic.
  \end{enumerate}
  %
  \begin{proof}
    Suppose by way of contradiction there is a homeomorphism
    %
    \begin{equation*}
      f: (0, 1] \rightarrow (0, 1).
    \end{equation*}
    %
    Then the map
    %
    \begin{align*}
      g: (0, 1) &\rightarrow (0, 1) \setminus f(\{1\}) \\
              x &\mapsto f(x)
    \end{align*}
    %
    is also a homeomorphism.
    But since $f(1) \in (0, 1)$ by hypothesis, $(0, 1) \setminus f(\{1\}) = (0, f(1)) \cup (f(1), 1)$ is disconnected.
    Since $(0, 1)$ is connected, its image under a continuous map must also be connected.
    This contradicts that $g$, and therefore $f$, are homeomorphisms.
    Thus, $(0, 1)$ and $(0, 1]$ are not homeomorphic.
    The proofs for the other spaces follow analogously.
  \end{proof}
  \bigskip

  \begin{enumerate}[label={(\alph*)}, align=left, leftmargin=\parindent, listparindent=\parindent, labelwidth=0pt, itemindent=!]
    \addtocounter{enumi}{1} 
    \item
      Suppose that there exist imbeddings $f: X \rightarrow Y$ and $g: Y \rightarrow X$.
      Show by means of an example that $X$ and $Y$ need not be homeomorphic.
  \end{enumerate}
  \begin{proof}
    We have seen in part (a) that the spaces $(0, 1)$ and $(0, 1]$ are not homeomorphic.
    Nevertheless, we can construct imbeddings between them.
    Consider the map
    %
    \begin{align*}
      f:  (0, 1]  &\rightarrow (0, 1) \\
                x & \mapsto x / 2.
    \end{align*}
    %
    Then the map $f': (0, 1] \rightarrow (0, 1/2]$ obtained by restricting the range of $f$ to its image set $f((0, 1])$, considered as a subspace of $(0, 1)$ is a homeomorphism.
    Thus, $f$ is an imbedding of $(0, 1]$ in $(0, 1)$.

    Consider now the map
    %
    \begin{equation*}
      g: (0, 1) \rightarrow (0, 1]
    \end{equation*}
    %
    given by the inclusion function.
    Then the map $g': (0, 1) \rightarrow (0, 1)$ obtained by restricting the range of $g$ to its image set $g((0, 1))$, considered as a subspace of $(0, 1]$ is just the identity map and, thus, a homeomorphism.
    Thus, $g$ is an imbedding of $(0, 1)$ in $(0, 1]$.
  \end{proof}
  \bigskip

  \begin{enumerate}[label={(\alph*)}, align=left, leftmargin=\parindent, listparindent=\parindent, labelwidth=0pt, itemindent=!]
    \addtocounter{enumi}{2} 
    \item 
      Show that $\RR^n$ and $\RR$ are not homeomorphic if $n > 1$.
  \end{enumerate}
  \begin{proof}
    Suppose by way of contradiction there is a homeomorphism
    %
    \begin{equation*}
      f: \RR^n \rightarrow \RR
    \end{equation*}
    %
    for some integer $n > 1$.
    Then the map
    %
    \begin{align*}
      g: \RR^n \setminus \{\mathbf{0}\} &\rightarrow \RR \setminus f(\{\mathbf{0}\}) \\
                                      x &\mapsto f(x)
    \end{align*}
    %
    is also a homeomorphism.
    But since $f(\mathbf{0}) \in \RR$ by hypothesis, $\RR \setminus f(\{\mathbf{0}\}) = (-\infty, f(\mathbf{0})) \cup (f(\mathbf{0}), \infty)$ is disconnected.
    Since we have seen previously that punctured euclidean space $\RR^n \setminus \{\mathbf{0}\}$ is connected for $n > 1$, its image under a continuous map must also be connected.
    This contradicts that $g$, and therefore $f$, are homeomorphisms.
    Thus, $\RR^n$ and $\RR$ are not homeomorphic if $n > 1$.
  \end{proof}
\end{solution}
\newpage

\begin{exercise}[ID=3.24.2]
  Let $f: S^1 \rightarrow \RR$ be a continuous map.
  Show there exists a point $x$ of $S^1$ such that $f(x) = f(-x)$.
\end{exercise}
%
\begin{solution}
  \begin{proof}
    Let $f: S^1 \rightarrow \RR$ be a continuous map.
    Then the map
    %
    \begin{align*}
      g: S^1  &\rightarrow \RR \\
            x &\mapsto f(x) - f(-x)
    \end{align*}
    %
    is also continuous.

    Assume by way of contradiction that for all $x \in S^1$, $f(x) \neq f(-x)$.
    Then $g(S^1) \subset \RR \setminus \{0\} = (-\infty, 0) \cup (0, \infty)$, which is disconnected.
    Since $g(x) = -g(-x)$, it must be that $g(S^1) \cap (-\infty, 0) \neq \emptyset$ and $g(S^1) \cap (0, \infty) \neq \emptyset$.
    But as $S^1$ is a connected space, this contradicts the fact that the image of $S^1$ under a continuous map is connected and, thus, must lie either entirely within $(-\infty, 0)$ or within $(0, \infty)$.
    Thus, if $f: S^1 \rightarrow \RR$ is a continuous map, then there exists a point $x$ of $S^1$ such that $f(x) = f(-x)$.
  \end{proof}
\end{solution}
\newpage

\begin{exercise}[ID=3.24.3]
  Let $f: X \rightarrow X$ be a continuous map.
  Show that if $X = [0, 1]$, there is a point $x$ such that $f(x) = x$.
  What happens if $X$ equals $[0, 1)$ or $(0, 1)$?
\end{exercise}
%
\begin{solution}
  \begin{proof}
    Let $f: [0, 1] \rightarrow [0, 1]$ be a continuous map.
    Then the map
    %
    \begin{align*}
      g: [0, 1] &\rightarrow [-1, 1] \\
              x &\mapsto f(x) - x
    \end{align*}
    %
    is also continuous.
    Since $g(0) = f(0) \geq 0$ and $g(1) = f(1) - 1 \leq 0$, by the intermediate value theorem there exists an $x \in [0, 1]$ satisfying $g(x) = 0$ and, thus, $f(x) = x$.

    We now consider what happens when $X = [0, 1)$ or $X = (0, 1)$.
    Consider the map
    %
    \begin{align*}
      f: X  &\rightarrow X \\
         x  &\mapsto \frac{x + 1}{2}.
    \end{align*}
    %
    The only solution to $f(x) = x$ in this case would be $x = 1$.
    However, $1 \not\in X$ for either case.
    Thus, for $X = [0, 1)$ or $X = (0, 1)$ a continous map $f: X \rightarrow X$ need not have a fixed point.
  \end{proof}
\end{solution}
\newpage

\begin{exercise}[ID=3.24.4]
  Let $X$ be an ordered set in the order topology.
  Show that if $X$ is connected, then $X$ is a linear continuum.
\end{exercise}
%
\begin{solution}
  \begin{proof}
    Let $X$ be an ordered set in the order topology which is connected.
    Suppose by way of contradiction that $X$ is not a linear continuum.
    We consider two possibilities.

    Suppose first that $X$ does not satisfy the least upper bound property.
    Then there exists some nonempty subset $A$ of $X$ which is bounded above but for which there exists no least upper bound in $X$.
    Let $B = \{ b \in X \mid a \leq b, \forall a \in A \}$ be the set of upper bounds of $A$.
    Since $A$ has no least upper bound, $B$ does not have a smallest element.
    Thus, defining open sets
    %
    \begin{equation*}
      U = \bigcup_{a \in A} (-\infty, a) \; \text{ and }\; V = \bigcup_{b \in B} (b, \infty),
    \end{equation*}
    %
    we have that $X = U \cup V$ is a separation of $X$, contradicting our assumption that $X$ is connected.

    Suppose instead there exist $x, y \in X$ with $x < y$ such that for any $z \in X$, it is not the case that $x < z < y$.
    Then we have that $X = (-\infty, y) \cup (x, \infty)$ is a separation of $X$, contradicting our assumption that $X$ is connected.
  \end{proof}
\end{solution}
\newpage

\begin{exercise}[ID=3.24.6]
  Show that if $X$ is a well-ordered set, then $X \times [0, 1)$ in the dictionary order is a linear continuum.
\end{exercise}
%
\begin{solution}
  \begin{proof}
    Let $X$ be a well-ordered set and let $A \subset X \times [0, 1)$ be a nonempty subset of $X \times [0, 1)$ in the dictionary order which is bounded above.

    \subsubsection*{The dictionary order satisfies the least upper bound property}
    %
    By hypothesis there exists some upper bound $(x, u) \in X \times [0, 1)$ of $A$.
    It must then be the case that $\pi_1(A) \subset X$ is bounded from above by $x$, where $\pi_1$ is the projection map from $A$ to the first factor.
    Since $X$ is well-ordered by hypothesis, $\pi_1(A)$ must possess a least upper bound $z \in X$.
    We consider the two cases $\pi_1\inv(\{z\}) = \emptyset$ and $\pi_1\inv(\{z\}) \neq \emptyset$ and show that in each case there exists a least upper bound for $A$.
    
    Suppose $\pi_1\inv(\{z\}) = \emptyset$.
    Then $(z, 0)$ is an upper bound for $A$.
    Let $(x, u)$ be some upper bound for $A$.
    Since $z$ is the least upper bound for $\pi_1(A)$, we have that $z \leq x$.
    Furthermore, since $0 \leq u$ we have that $(z, 0) \leq (x, u)$ and so $(z, 0)$ is the least upper bound for $A$.

    Suppose instead that $\pi_1\inv(\{z\}) \neq \emptyset$.
    Then $u$ must bound $\pi_2(\pi_1\inv(\{z\})) \subset [0, 1)$ from above, where $\pi_2$ is the projection map from $A$ to the second factor.
    Since $[0, 1)$ is a linear continuum it possesses the least upper bound property.
    It follows that $\pi_2(\pi_1\inv(\{z\}))$ has a least upper bound $w \in [0, 1)$.

    We now show that $(z, w) \in X \times [0, 1)$ is the least upper bound for $A$.
    Let $(y, v) \in A$.
    Since $z$ is the least upper bound for $\pi_1(A)$, it must be that $y \leq z$.
    If $y < z$, then $(y, v) < (z, w)$.
    If instead $y = z$, we note that since $w$ is the least upper bound for $\pi_2(A)$, it must be that $v \leq w$.
    Thus, for arbitrary $(y, v) \in X \times [0, 1)$, we see that $(y, v) \leq (z, w)$, from which it follows that $(z, w)$ is an upper bound for $A$.

    Now let $(x, u)$ be an upper bound for $A$.
    Then for all $(y, v) \in A$, either $y < x$ or $y = x$ and $v \leq u$.
    Suppose first that $y < x$.
    Since $z$ is the least upper bound for $\pi_1(A)$, then $z \leq x$.
    If $z < x$, then $(z, w) < (x, u)$.
    If instead $z = x$, we note that $w$ is the least upper bound for $\pi_2(A)$ and so $w \leq u$.
    Thus, $(z, w) \leq (x, u)$.
    Suppose instead that $y = x$. Since $z$ is the least upper bound for $\pi_1(A)$, then $z = x$.
    As above, since $w$ is the least upper bound for $\pi_2(A)$, we have that $w \leq u$.
    Thus, $(z, w) \leq (x, u)$.
    It follows that $(z, w)$ furnishes a least upper bound for the arbitrary nonempty, bounded from above subset $A$ of $X \times [0, 1)$ in the dictionary order.

    \subsubsection*{If $(x, u) < (y, v)$, there exists $(z, w)$ such that $(x, u) < (z, w) < (y, v)$}
    %
    Let $(4x, u), (y, v) \in X \times [0, 1)$ with $(x, u) < (y, v)$.
    Suppose $x < y$.
    Since $[0, 1)$ is a linear continuum, we can always find $w \in [0, 1)$ such that $u < w$.
    Choosing $z = x$, we have $(x, u) < (z, w) < (y, v)$.

    Suppose instead that $x = y$ and $u < v$.
    Again, as $[0, 1)$ is a linear continuum, we can find $w \in [0, 1)$ such that $u < w < v$.
    Choosing $z = x$, we have $(x, u) < (z, w) < (y, v)$.
    Thus, we see that $X \times [0, 1)$ in the dictionary order is a linear continuum.
  \end{proof}
\end{solution}
\newpage

\begin{exercise}[ID=3.24.7]
  \begin{enumerate}[label={(\alph*)}, align=left, leftmargin=\parindent, listparindent=\parindent, labelwidth=0pt, itemindent=!]
    \item
      Let $X$ and $Y$ be ordered sets in the order topology.
      Show that if $f: X \rightarrow Y$ is order preserving and surjective, then $f$ is a homeomorphism.
    \item
      Let $X = Y = \overline{\RR}_+$.
      Given a positive integer $n$, show that the function $f(x) = x^n$ is order preserving and surjective.
      Conclude that its inverse, the $n$\textit{th root function}, is continous.
    \item 
      Let $X$ be the subspace $(-\infty, -1) \cup [0, \infty)$ of $\RR$.
      Show that the function $f: X \rightarrow \RR$ defined by setting $f(x) = x + 1$ if $x < -1$, and $f(x) = x$ if $x \geq 0$, is order preserving and surjective.
      Is $f$ a homeomorphism?
      Compare with (a).
    \end{enumerate}
\end{exercise}
%
\begin{solution}
  \begin{enumerate}[label={(\alph*)}, align=left, leftmargin=\parindent, listparindent=\parindent, labelwidth=0pt, itemindent=!]
    \item
      Let $X$ and $Y$ be ordered sets in the order topology.
      Show that if $f: X \rightarrow Y$ is order preserving and surjective, then $f$ is a homeomorphism.
  \end{enumerate}
  \begin{proof}
    Let $f: X \rightarrow Y$ be order preserving and surjective.
    We begin by showing that $f$ must be bijective with order preserving inverse.
    Given $u, v \in X$, if $u \neq v$ then either $u < v$ or $v < u$, since $X$ is an ordered set.
    Without loss of generality, suppose that $u < v$.
    Since $f$ is order preserving, we have that $f(u) < f(v)$ so that $f(u) \neq f(v)$.
    Thus, $f$ injective in addition to being surjective and is, therefore, a bijection.

    Let $a, b \in Y$ and let $u = f\inv(a)$ and $v = f\inv(b)$.
    Suppose by way of contradiction that $f\inv$ is not order preserving.
    Since $f$ is bijective, $u = v$ if and only if $a = b$.
    Suppose that $a < b$ so that $u \neq v$.
    Since we assume that $f\inv$ is not order preserving, it is possible that $v < u$.
    As $f$ \textit{is} order preserving, this would imply that $b = f(v) < f(u) = a$, contradicting our assumption that $a < b$.
    Thus, $f\inv: Y \rightarrow X$ is order preserving.

    We now show that continuity of $f$ follows directly from it being order preserving and bijective.
    Under the order topology, basic open sets in $Y$ are either of the form (i) open intervals $(a, b)$ in $Y$, (ii) intervals of the form $[a_0, b)$, where $a_0$ is the smallest element (if any) in $Y$, (iii) intervals of the form $(a, b_0]$, where $b_0$ is the largest element (if any) of $Y$.
    Since $f\inv$ is order preserving, we thus consider sets in $X$ of the form (i) $f\inv((a, b)) = (f\inv(a), f\inv(b))$, (ii) $f\inv([a_0, b)) = [f\inv(a_0), f\inv(b))$, (iii) $f\inv((a, b_0]) = (f\inv(a), f\inv(b_0)]$.
    As these are nothing more than the basic open sets in $X$ under the order topology, we have that $f$ is continuous.

    Since $f\inv$ is also order preserving and bijective, it must also be continuous.
    Thus, $f: X \rightarrow Y$ is a homeomorphism.
  \end{proof}

  %
  \begin{enumerate}[label={(\alph*)}, align=left, leftmargin=\parindent, listparindent=\parindent, labelwidth=0pt, itemindent=!]
    \addtocounter{enumi}{1} 
    \item
      Let $X = Y = \overline{\RR}_+$.
      Given a positive integer $n$, show that the function $f(x) = x^n$ is order preserving and surjective.
      Conclude that its inverse, the $n$\textit{th root function}, is continous.
  \end{enumerate}
  %
  \begin{proof}~

    \subsubsection*{$x^n$ is order preserving}
    %
    Let $x, y \in \overline{\RR}_+$ with $x < y$ and define $\delta = y - x > 0$.
    We proceed by induction.

    For $n = 1$, clearly $x^n < y^n$.
    Assume that for some $n \in \ZZ_+$, $x^n < y^n$.
    Then
    %
    \begin{equation*}
      x^{n+1} = x \cdot x^n < x \cdot y^n < y \cdot y^n = y^{n+1}.
    \end{equation*}
    %
    By induction, it follows that $x^n$ is order preserving.
    \bigskip

    %
    \subsubsection*{$x^n$ is surjective}
    %
    Let $y \in \overline{\RR}_+$ and consider the set $S = \{s \in \overline{\RR}_+ \mid s^n \leq y \}$.
    Given $t \in \overline{\RR}_+ = \max\{1, y\}$, we have that $t^n > y$ and, thus, $S$ is bounded from above.
    By the least upper bound property of the real numbers, we know then that $x = \sup S$ exists and is in $\overline{\RR}_+$.
    We will show that $x^n = y$, thus making $x^n$ surjective.

    For the case $y = 0$, we have that $S = \{0\}$ and, thus, $x = \sup S = 0$.
    It follows that $x^n = 0 = y$, and we are done.

    Suppose that $y > 0$ and that $x^n < y$.
    Let $\delta = y - x^n > 0$.
    Consider $x + \epsilon$, where
    %
    \begin{equation*}
      \epsilon = \min\{1, \frac{\delta}{n (x + 1)^{n-1}}\} > 0.
    \end{equation*}
    %
    Then
    %
    \begin{equation*}
      \begin{aligned}
        (x + \epsilon)^n  &= x^n + (x + \epsilon)^n - x^n \\
                          &= x^n + \epsilon ((x + \epsilon)^{n-1} + (x + \epsilon)^{n-2} x + \ldots + x^{n-1}) \\
                          &< x^n + n \epsilon (x + \epsilon)^{n-1} \\
                          &\leq x^n + n \epsilon (x + 1)^{n-1} \\
                          &\leq x^n + \delta \\
                          &= y.
      \end{aligned}
    \end{equation*}
    %
    By construction, we have that $x + \epsilon \in S$ and $x + \epsilon > x$, contradicting the fact that $x$ is an upper bound for $S$.

    Suppose instead that $x^n > y$ and $y > 0$.
    Let $\delta = x^n - y > 0$.
    Consider $x - \epsilon$, where
    %
    \begin{equation*}
      \epsilon = \frac{\delta}{n x^{n-1}} > 0.
    \end{equation*}
    %
    Then
    %
    \begin{equation*}
      \begin{aligned}
        (x - \epsilon)^n  &= x^n + (x - \epsilon)^n - x^n \\
                          &= x^n - \epsilon (x^{n-1} + x^{n-2} (x - \epsilon) + \ldots + (x - \epsilon)^{n-1}) \\
                          &> x^n - n \epsilon x^{n-1} \\
                          &= x^n - \delta \\
                          &= y.
      \end{aligned}
    \end{equation*}
    %
    By construction, we now have that $(x - \epsilon)^n > y$ and $x - \epsilon < x$, contradicting the fact that $x$ is the least upper bound for $S$.
    Since $x$ exists, it must be the case that $x^n = y$.
    Thus, $x^n$ is surjective.
    It follows that its inverse, the $n$\textit{th root function}, is continous.
  \end{proof}

  %
  \begin{enumerate}[label={(\alph*)}, align=left, leftmargin=\parindent, listparindent=\parindent, labelwidth=0pt, itemindent=!]
  %
    \addtocounter{enumi}{2} 
    \item 
      Let $X$ be the subspace $(-\infty, -1) \cup [0, \infty)$ of $\RR$.
      Show that the function $f: X \rightarrow \RR$ defined by setting $f(x) = x + 1$ if $x < -1$, and $f(x) = x$ if $x \geq 0$, is order preserving and surjective.
      Is $f$ a homeomorphism?
      Compare with (a).
  \end{enumerate}
  %
  \begin{proof}~

    \subsection*{$f$ is order preserving}
    %
    Let $x, y \in X$ with $x < y$.
    There are three cases to consider: (i) $x < y < -1$, (ii) $x < -1$, $y \geq 0$, (iii) $0 \leq x < y$.

    (i) Suppose first that $x < y < -1$.
    Then $f(x) = x + 1$ and $f(y) = y + 1$ implies
    %
    \begin{equation*}
      \begin{aligned}
        f(y) - f(x) &= (y + 1) - (x + 1) \\
                    &= y - x \\
                    &> 0.
      \end{aligned}
    \end{equation*}
    %
    (ii) Suppose next that $x < -1$ and $y \geq 0$, so that $y - x > 1$.
    Then $f(x) = x + 1$ and $f(y) = y$ implies
    %
    \begin{equation*}
      \begin{aligned}
        f(y) - f(x) &= y - (x + 1) \\
                    &= y - x - 1 \\
                    &> 0.
      \end{aligned}
    \end{equation*}
    %
    (iii) Finally, suppose that $0 \leq x < y$.
    Then $f(x) = x$ and $f(y) = y$ implies
    %
    \begin{equation*}
      \begin{aligned}
        f(y) - f(x) &= y - x \\
                    &> 0.
      \end{aligned}
    \end{equation*}
    %
    Thus, $f: X \rightarrow \RR$ is order preserving.

    %
    \subsection*{$f$ is surjective}
    %
    Let $y \in \RR$.
    Suppose $y < 0$ and consider $x = y - 1 \in (-\infty, -1)$.
    Then
    %
    \begin{equation*}
      f(x) = x + 1 = (y - 1) + 1 = y.
    \end{equation*}
    %
    Suppose instead that $y \geq 0$ and consider $x = y \in [0, \infty)$.
    Then
    %
    \begin{equation*}
      f(x) = x = y.
    \end{equation*}
    %
    Thus, $f: X \rightarrow \RR$ is surjective.
    
    %
    \subsection*{$f$ is not a homeomorphism}
    %
    As an order preserving function, $f: X \rightarrow \RR$ is injective.
    Since we have seen that it is also surjective, it possesses an inverse, $f\inv: \RR \rightarrow X$.
    We show that $f\inv$ is not continuous, thus showing that $f$ is not a homeomorphism.

    Let $0 < \epsilon < 1$.
    The set $(-\epsilon, \epsilon) \cap X$ is open in $X$ in the subspace topology since $(-\epsilon, \epsilon)$ is open in $\RR$.
    But
    %
    \begin{equation*}
      f\left((-\epsilon, \epsilon) \cap X\right) = f\left([0, \epsilon)\right) = [0, \epsilon),
    \end{equation*}
    %
    which is not open in $\RR$.
    Thus, $f\inv$ is not continuous and $f$ is not a homeomorphism.

    %
    \subsection*{Comparison with (a)}
    %
    As $X$ is not convex in $\RR$, it is not guaranteed that the subspace topology on $X$ coincides with the order topology on $X$.
    In fact, for $0 < \epsilon < 1$, the set $(-\epsilon, \epsilon) \cap X = [0, \epsilon)$ is open in $X$ in the subpsace topology, but is not open in the order topology on $X$, since $[0, \epsilon)$ cannot be expressed as a union of open intervals in $X$.
    Thus, $X$ as a subspace of $\RR$ does not carry the order topology and is therefore not covered by (a).
  \end{proof}
  %
\end{solution}
%
\newpage

%
\begin{exercise}[ID=3.24.8]
  \begin{enumerate}[label={(\alph*)}, align=left, leftmargin=\parindent, listparindent=\parindent, labelwidth=0pt, itemindent=!]
    \item
    Is a product of path-connected spaces necessarily path connected?
    \item
      If $A \subset X$ and $A$ is path connected, is $\overline{A}$ necessarily path connected?
    \item 
      If $f: X \rightarrow Y$ is continuous and $X$ is path connected, is $f(X)$ necessarily path connected?
    \item 
      If $\{A_\alpha\}$ is a collection of path-connected subspaces of $X$ and if $\bigcap A_\alpha \neq \emptyset$, is $\bigcup A_\alpha$ necessarily path connected?
    \end{enumerate}
\end{exercise}
%
\begin{solution}
  \begin{enumerate}[label={(\alph*)}, align=left, leftmargin=\parindent, listparindent=\parindent, labelwidth=0pt, itemindent=!]
    \item
      Is a product of path-connected spaces necessarily path connected?
  \end{enumerate}
  \textbf{Claim.} Given an indexed family of path-connected spaces $\{X_\alpha\}$, the product space $\prod_\alpha X_\alpha$ is path connected.
  %
  \begin{proof}
    Let $\{X_\alpha\}$ be an indexed family of path-connected spaces and let $\bx = (x_\alpha)$ and $\by = (y_\alpha)$ be elements of the product space $\prod_\alpha X_\alpha$.
    By hypothesis, for each $\alpha$ there exists a continuous function $f_\alpha: [0, 1] \rightarrow X_\alpha$ such that $f_\alpha(0) = x_\alpha$ and $f_\alpha(1) =y_\alpha$.
    Consider then the function
    %
    \begin{equation*}
      \begin{aligned}
        h: [0, 1] &\rightarrow \prod_\alpha X_\alpha \\
        t         &\mapsto (f_\alpha(t))_\alpha.
      \end{aligned}
    \end{equation*}
    %
    By Theorem 2.19.6, such a function is continuous if and only if each function $f_\alpha$ is continuous.
    Thus, by construction, $h$ is a continuous function.
    Furthermore, we have that
    %
    \begin{equation*}
      h(0) = (f_\alpha(0))_\alpha = (x_\alpha) = \bx \qquad\text{and}\qquad h(1) = (f_\alpha(1))_\alpha = (y_\alpha)_\alpha = \by.
    \end{equation*}
    %
    Thus, $\prod_\alpha X_\alpha$ is a path-connected space.
  \end{proof}
  %

  %
  \begin{enumerate}[label={(\alph*)}, align=left, leftmargin=\parindent, listparindent=\parindent, labelwidth=0pt, itemindent=!]
    \addtocounter{enumi}{1} 
    \item
      If $A \subset X$ and $A$ is path connected, is $\overline{A}$ necessarily path connected?
  \end{enumerate}
  %
  \textbf{Claim.} $\overline{A}$ is not necessarily path connected.
  %
  \begin{proof}
    Consider the set
    %
    \begin{equation*}
      S = \{(x, \sin{(1/x)}) \mid 0 < x \leq 1\}.
    \end{equation*}
    %
    Let $(x, \sin{(1/x)})$ and $(y, \sin{(1/y)})$ be points in $S$, where $x < y$.
    Consider the map
    %
    \begin{equation*}
      \begin{aligned}
        f:\,  &[x, y] \rightarrow S \\
              & t \mapsto (t, \sin{(1/t)}).
      \end{aligned}
    \end{equation*}
    %
    Clearly, $f(x) = (x, \sin{(1/x)})$ and $f(y) = (y, \sin{(1/y)})$.
    Furthermore, since both coordinate functions are continuous, so too is $f$ continuous.
    Thus, $S$ is seen to be path connected.
    However, from Example 3.24.7, we know that its closure $\overline{S}$ in $\RR^2$, the topologist's sine curve, is not path connected.
  \end{proof}
  %

  %
  \begin{enumerate}[label={(\alph*)}, align=left, leftmargin=\parindent, listparindent=\parindent, labelwidth=0pt, itemindent=!]
    \addtocounter{enumi}{2} 
    \item 
      If $f: X \rightarrow Y$ is continuous and $X$ is path connected, is $f(X)$ necessarily path connected?
  \end{enumerate}
  %
  \textbf{Claim.} Given a continuous map $f: X \rightarrow Y$, where $X$ is path connected, $f(X)$ is also path connected.
  %
  \begin{proof}
    Let $y_0, y_1 \in f(X)$.
    Then there exist points $x_0 \in f\inv(\{y_0\}) \subset X$ and $x_1 \in f\inv(\{y_1\}) \subset X$ satisfying $f(x_0) = y_0$ and $f(x_1) = y_1$.
    Since $X$ is path connected, there exists a path $g: [0, 1] \rightarrow X$ satisfying $g(0) = x_0$ and $g(1) = x_1$.
    Consider the map $f \circ g: [0, 1] \rightarrow Y$.
    Since both $f$ and $g$ are continuous, so too is their composition.
    Furthermore,
    %
    \begin{equation*}
      \begin{aligned}
        (f \circ g)(0)  &= f(g(0)) = f(x_0) = y_0, \\
        (f \circ g)(1)  &= f(g(1)) = f(x_1) = y_1,
      \end{aligned}
    \end{equation*}
    %
    and for all $t \in [0, 1]$, $(f \circ g)(t) \in f(X)$.
    Thus, $f(X)$ is path connected.
  \end{proof}
  %
\end{solution}
%

%
  \begin{enumerate}[label={(\alph*)}, align=left, leftmargin=\parindent, listparindent=\parindent, labelwidth=0pt, itemindent=!]
    \addtocounter{enumi}{3} 
    \item 
      If $\{A_\alpha\}$ is a collection of path-connected subspaces of $X$ and if $\bigcap A_\alpha \neq \emptyset$, is $\bigcup A_\alpha$ necessarily path connected?
  \end{enumerate}
  %
  \textbf{Claim.} Given a collection $\{A_\alpha\}$ of path-connected subspaces of $X$, where $\bigcap A_\alpha \neq \emptyset$, the subspace $\bigcup A_\alpha$ is also path connected.
  %
  \begin{proof}
    Let $a \in \bigcap A_\alpha \neq \emptyset$.
    Let $x, y \in \bigcup A_\alpha$.
    Then there exist path-connected subspaces $A_{\alpha_1}$ and $A_{\alpha_2}$ containing $x$ and $y$, respectively.
    Since $A_{\alpha_1}$ is path connected and must contain $a$, there exists a path $f: [0, 1] \rightarrow A_{\alpha_1}$ joining $x$ to $a$.
    Similarly, since $A_{\alpha_2}$ is path connected and also contains $a$, there exists a path $g: [0, 1] \rightarrow A_{\alpha_2}$ joining $a$ to $y$.
    Define a map $h: [0, 1] \rightarrow \bigcup A_\alpha$ by the equation
    %
    \begin{equation*}
      h(t) =
      \begin{cases}
        f(2t)     & \text{if} \quad 0 \leq t \leq 1/2 \\
        g(2t - 1) & \text{if} \quad 1/2 < t \leq 1.
      \end{cases}
    \end{equation*}
    %
    By construction, $h(0) = x$ and $h(1) = y$.
    Furthermore, $h(1/2) = f(1) = a = g(0)$, so $h$ is continuous by the pasting lemma.
    It follows that $h$ is a path joining $x$ to $y$.
    Thus, $\bigcup A_\alpha$ is path connected.
  \end{proof}
  %
\newpage

