\subsection{The Metric Topology}

\begin{exercise}[ID=2.20.1]
    \begin{enumerate}[label={(\alph*)}, align=left, leftmargin=\parindent, listparindent=\parindent, labelwidth=0pt, itemindent=!]
        \item In $\RR^n$, define
        %
        \begin{equation*}
            d'(\bx, \by) = |x_1 - y_1| + \cdots + |x_n - y_n|.
        \end{equation*}
        %
        Show that $d'$ is a metric that induces the usual topology of $\RR^n$.
        \item  More generally, given $p \geq 1$, define
        %
        \begin{equation*}
            d'(\bx, \by) = \left[\sum_{i=1}^n |x_i - y_i|^p\right]^{1/p}
        \end{equation*}
        %
        for $\bx, \by \in \RR^n$.
        Assume that $d'$ is a metric.
        Show that it induces the usual topology on $\RR^n$.
    \end{enumerate}
\end{exercise}

\begin{solution}
    \begin{enumerate}[label={(\alph*)}, align=left, leftmargin=\parindent, listparindent=\parindent, labelwidth=0pt, itemindent=!]
        \item In $\RR^n$, define
        %
        \begin{equation*}
            d'(\bx, \by) = |x_1 - y_1| + \cdots + |x_n - y_n|.
        \end{equation*}
        %
        Show that $d'$ is a metric that induces the usual topology of $\RR^n$.
    \end{enumerate}
    %
    \begin{proof}
        To prove that $d'$ is a metric on $\RR^n$, we must show that 1) $\forall \bx, \by \in \RR^n,~ d'(\bx, \by) \geq 0$ with equality if and only if $\bx = \by$, 2) $d'(\bx, \by) = d'(\by, \bx)$ and 3) $d'$ satisfies the triangle inequality.

        1) Given that the standard metric on $\RR$ satisfies $|x_i - y_i| \geq 0$ for all $x_i, y_i \in \RR$, we have that $d'(\bx, \by) = |x_1 - y_1| + \cdots + |x_n - y_n| \geq 0$ for all $\bx, \by \in \RR^n$.

        Suppose that $d'(\bx, \by) = 0$.
        Then $|x_1 - y_1| + \cdots + |x_n - y_n| = 0$.
        As the standard metric on $\RR$ is positive definite, we have that every term in the sum is greater than or equal to zero, thus, they must all be zero.
        Therefore, $\forall i \in \{1, \ldots, n\},~ |x_i - y_i| = 0$ and so $x_i = y_i$.
        It follows that $\bx = \by$.

        Conversely, suppose that $\bx = \by$.
        Then
        %
        \begin{align*}
            d'(\bx, \by)    &= |x_1 - y_1| + \cdots + |x_n - y_n| \\
                            &= |x_1 - x_1| + \cdots + |x_n - x_n| \\
                            &= 0.
        \end{align*}
        %

        2) Let $\bx, \by \in \RR^n$.
        Then
        %
        \begin{align*}
            d'(\bx, \by)    &= |x_1 - y_1| + \cdots + |x_n - y_n| \\
                            &= |y_1 - x_1| + \cdots + |y_n - x_n| \\
                            &= d'(\by, \bx).
        \end{align*}
        %

        3) Let $\bx, \by, \bz \in \RR^n$.
        As the standard metric on $\RR$ satisfies the triangle inequality, we have that $\forall i \in \{1, \ldots, n\},~ |x_i - z_i| \leq |x_i - y_i| + |y_i - z_i|$.
        Therefore,
        %
        \begin{align*}
            |x_1 - z_1| + \cdots + |x_n - z_n|  &\leq \left(|x_1 - y_1| + |y_1 - z_1|\right) + \cdots + \left(|x_n - y_n| + |y_n - z_n|\right) \\
                                                &= \left(|x_1 - y_1| + \cdots + |x_n - y_n|\right) + \left(|y_1 - z_1| + \cdots + |y_n - z_n|\right).
        \end{align*}
        From here it follows that $d'(\bx, \bz) \leq d'(\bx, \by) + d'(\by, \bz)$.
        Thus, $d'$ is seen to be a metric on $\RR^n$.

        To show that $d'$ induces the usual topology on $\RR^n$, we will leverage Lemma 2.20.2, which states that given two metrics $d$ and $d'$ on a set $X$ which induce topologies $\mathcal{T}$ and $\mathcal{T}'$, respectively, the topology $\mathcal{T}'$ is finer than $\mathcal{T}$ if and only if for each $x \in X$ and each $\epsilon >0$, there exists a $\delta > 0$ such that $B_{d'}(x, \delta) \subset B_d(x, \epsilon)$.
        Let $\bx \in \RR^n$ and let $B_\rho(\bx, \epsilon)$ be an $\epsilon$-ball centered at $\bx$ in the square metric $\rho$.
        We show that the open ball $B_{d'}(\bx, \epsilon)$ is contained in $B_\rho(\bx, \epsilon)$.
        Let $\by \in B_{d'}(\bx, \epsilon)$.
        Then
        %
        \begin{align*}
                            & d'(\bx, \by) = |x_1 - y_1| + \cdots + |x_n - y_n|  < \epsilon \\
            \implies\qquad  & \forall i \in \{1, \ldots, n\},~ |x_i - y_i| < \epsilon \\
            \implies\qquad  & \max \{|x_i - y_i|\}_{i=1}^n < \epsilon \\
            \implies\qquad  & \rho(\bx, \by) < \epsilon \\
            \implies\qquad  & \by \in B_\rho(\bx, \epsilon).
        \end{align*}
        %
        Thus, $B_{d'}(\bx, \epsilon) \subset B_\rho(\bx, \epsilon)$ and so the metric topology induced by $d'$ is finer than the standard topology.

        On the other hand, let $B_{d'}(\bx, \epsilon)$ be an $\epsilon$-ball centered at $\bx$ in the $d'$ metric.
        We show that the open ball $B_\rho(\bx, \epsilon/n)$ is contained in $B_{d'}(\bx, \epsilon)$.
        Let $\by \in B_\rho(\bx, \epsilon/n)$.
        Then
        %
        \begin{align*}
                            & \rho(\bx, \by)  = \max\{|x_i - y_i|\}_{i=1}^n < \epsilon / n \\
            \implies\qquad  & \forall i \in \{1, \ldots, n\},~ |x_i - y_i| < \epsilon / n \\
            \implies\qquad  & |x_1 - y_1| + \cdots + |x_n - y_n| < \epsilon \\
            \implies\qquad  & d'(\bx, \by) < \epsilon \\
            \implies\qquad  & \by \in B_{d'}(\bx, \epsilon).
        \end{align*}
        %
        Thus, $B_\rho(\bx, \epsilon / n) \subset B_{d'}(\bx, \epsilon)$ and so the standard topology is finer than the topology induced by $d'$.

        We have, thus, shown that the $d'$ is a metric that induces the standard topology on $\RR^n$.
    \end{proof}
    \bigskip

    \begin{enumerate}[label={(\alph*)}, align=left, leftmargin=\parindent, listparindent=\parindent, labelwidth=0pt, itemindent=!]
        \addtocounter{enumi}{1} 
         \item  More generally, given $p \geq 1$, define
         %
         \begin{equation*}
             d'(\bx, \by) = \left[\sum_{i=1}^n |x_i - y_i|^p\right]^{1/p}
         \end{equation*}
         %
         for $\bx, \by \in \RR^n$.
         Assume that $d'$ is a metric.
         Show that it induces the usual topology on $\RR^n$.
    \end{enumerate}
    %
    \begin{proof}
        Let $\bx \in \RR^n$ and let $B_\rho(\bx, \epsilon)$ be an $\epsilon$-ball centered at $\bx$ in the square metric $\rho$.
        We show that the open ball $B_{d'}(\bx, \epsilon)$ is contained in $B_\rho(\bx, \epsilon)$.
        Let $\by \in B_{d'}(\bx, \epsilon)$.
        Then $d'(\bx, \by) = \left[\sum_{i=1}^n |x_i - y_i|^p\right]^{1/p} < \epsilon$ implies $\sum_{i=1}^n |x_i - y_i|^p < \epsilon^p$.
        Since $|x_i - y_i| \geq 0$ for all indices $i \in \{1, \ldots, n\}$, it follows that each term in the sum is greater than or equal to zero and, thus, each term must be less than $\epsilon^p$.
        From here we have that $\forall i \in \{1, \ldots, n\},~ |x_i - y_i| < \epsilon$ and, in particular, $\max \{|x_i - y_i|\}_{i=1}^n < \epsilon$. 
        Then $\rho(\bx, \by) < \epsilon$ and so $\by \in B_\rho(\bx, \epsilon)$.
        Thus, $B_{d'}(\bx, \epsilon) \subset B_\rho(\bx, \epsilon)$ and (by Theorem 2.20.2) the metric topology induced by $d'$ is finer than the standard topology.

        On the other hand, let $B_{d'}(\bx, \epsilon)$ be an $\epsilon$-ball centered at $\bx$ in the $d'$ metric.
        We show that the open ball $B_\rho(\bx, \epsilon/n)$ is contained in $B_{d'}(\bx, \epsilon)$.
        Let $\by \in B_\rho(\bx, \epsilon/n^{1/p})$.
        Then
        %
        \begin{align*}
                            & \rho(\bx, \by)  = \max\{|x_i - y_i|\}_{i=1}^n < \epsilon / n^{1/p} \\
            \implies\qquad  & \forall i \in \{1, \ldots, n\},~ |x_i - y_i| < \epsilon / n^{1/p} \\
            \implies\qquad  & \forall i \in \{1, \ldots, n\},~ |x_i - y_i|^p < \epsilon^p / n \\
            \implies\qquad  & \sum_{i=1}^n |x_i - y_i|^p < \epsilon^p \\
            \implies\qquad  & \left[\sum_{i=1}^n |x_i - y_i|^p\right]^{1/p} < \epsilon \\
            \implies\qquad  & d'(\bx, \by) < \epsilon \\
            \implies\qquad  & \by \in B_{d'}(\bx, \epsilon).
        \end{align*}
        %
        Thus, $B_\rho(\bx, \epsilon / n) \subset B_{d'}(\bx, \epsilon)$ and so (by Theorem 2.20.2) the standard topology is finer than the topology induced by $d'$.

        We have, thus, shown that the metric $d'$ induces the standard topology on $\RR^n$.
    \end{proof}
\end{solution}
\newpage

\begin{exercise}[ID=2.20.2]
    Show that $\RR \times \RR$ in the dictionary order topology is metrizable.
\end{exercise}

\begin{solution}
    \begin{proof}
    \end{proof}
\end{solution}
\newpage

\begin{exercise}[ID=2.20.3]
    Let $X$ be a metric space with metric $d$.
    \begin{enumerate}[label={(\alph*)}, align=left, leftmargin=\parindent, listparindent=\parindent, labelwidth=0pt, itemindent=!]
        \item Show that $d: X \times X \rightarrow \RR$ is continuous.
        \item Let $X'$ denote a space having the same underlying set as $X$.
        Show that if $d: X' \times X' \rightarrow \RR$ is continuous, then the topology of $X'$ is finer than the topology of $X$.
    \end{enumerate}
\end{exercise}

\begin{solution}
    \begin{enumerate}[label={(\alph*)}, align=left, leftmargin=\parindent, listparindent=\parindent, labelwidth=0pt, itemindent=!]
        \item Show that $d: X \times X \rightarrow \RR$ is continuous.
    \end{enumerate}
    %
    \begin{proof}
        Fix an arbitrary point $(x, y) \in X \times X$ and let $\epsilon > 0$.
        Consider the open interval
        %
        \begin{equation*}
            V = (d(x, y) - \epsilon, d(x, y) + \epsilon),~ \epsilon > 0
        \end{equation*}
        %
        in $\RR$, which is a basic open neighborhood of $d(x, y)$.  
        To prove that $d: X \times X \rightarrow \RR$ is continuous, we find an open neighborhood $U$ of $(x, y)$ in $X \times X$ such that $d(U) \subset V$.

        Let $U = B_d(x, \epsilon/2) \times B_d(y, \epsilon/2)$.
        Since $X$ is a metric space, the open balls with respect to $d$ form basic open sets in $X$.
        Being a product of such open balls, the set $U$ is open in the product space $X \times X$ and, thus, an open neighborhood of $(x, y)$.

        Let $(u, v) \in U$.
        From the triangle inequality, we have that
        %
        \begin{equation*}
            d(u, v) \leq d(u, y) + d(y, v)\qquad {\rm and }\qquad d(u, y) \leq d(u, x) + d(x, y).
        \end{equation*}
        %
        Combining these, we get
        %
        \begin{equation*}
            d(u, v) \leq d(x, y) + d(u, x) + d(y, v).
        \end{equation*}
        %
        Since $u \in B_d(x, \epsilon/2)$ and $v \in B_d(y, \epsilon/2)$, we have that $d(u, x) < \epsilon/2$ and $d(y, v) < \epsilon/2$, thus, $d(u, v) < d(x, y) + \epsilon$.

        Similarly, we have
        %
        \begin{equation*}
            d(x, y) \leq d(x, u) + d(u, y)\qquad {\rm and }\qquad d(u, y) \leq d(u, v) + d(v, y),
        \end{equation*}
        %
        giving
        %
        \begin{align*}
            d(x, y) &\leq d(u, v) + d(x, u) + d(v, y) < d(u, v) + \epsilon,
        \end{align*}
        %
        or, equivalently, $d(x, y) - \epsilon < d(u, v)$.
        Thus, $(u, v) \in U$ implies that $d(x, y) - \epsilon < d(u, v) < d(x, y) + \epsilon$, so that $d(u, v) \in V$.

        More generally, since $(u, v) \in U$ was chosen arbitrarily, we have $d(U) \subset V$ and, thus, $d: X \times X \rightarrow \RR$ is continuous.
    \end{proof}
    \bigskip

    \begin{enumerate}[label={(\alph*)}, align=left, leftmargin=\parindent, listparindent=\parindent, labelwidth=0pt, itemindent=!]
        \addtocounter{enumi}{1} 
        \item Let $X'$ denote a space having the same underlying set as $X$.
        Show that if $d: X' \times X' \rightarrow \RR$ is continuous, then the topology of $X'$ is finer than the topology of $X$.
    \end{enumerate}
    %
    \begin{proof}
        Fix an arbitrary point $x \in X$ and $\epsilon > 0$.
        Consider the open ball $B_d(x, \epsilon)$ as a basic open neighborhood of $x$ in the topology on $X$.
        We show that there exists an open neighborhood $U$ of $x$ in the topology on $X'$ contained entirely within $B_d(x, \epsilon)$.
        By Theorem 2.13.3, this is equivalent to the statement that the topology on $X'$ is finer than the topology on $X$.

        Define the function
        %
        \begin{equation*}
            f_x: X \rightarrow \RR,\quad y \mapsto d(x, y).
        \end{equation*}
        %
        As $d$ is continuous, it is continuous in each variable separately and so $f_x$ is continuous.
        This holds for both the topology on $X$ and on $X'$, by hypothesis.
        Since $f_x(x) = 0$, the open interval $(-\epsilon, \epsilon)$ is an open neighborhood of $f_x(x)$ in $\RR$.
        Continuity of $f_x$ means there exists an open neighborhood $U$ of $x$ in $X'$ such that $f_x(U) \subset (-\epsilon, \epsilon)$.

        Expressing $B_d(x, \epsilon)$ in terms of $f_x$ as
        %
        \begin{equation*}
            B_d(x, \epsilon) = \{y \in X \mid f_x(y) < \epsilon\},
        \end{equation*}
        %
        we see that $B_d(x, \epsilon) = f_x\inv((-\epsilon, \epsilon))$.
        Since $f(U) \subset (-\epsilon, \epsilon)$, it immediately follows that 
        %
        \begin{equation*}
            U \subset f_x\inv(f_x(U)) \subset f_x\inv((-\epsilon, \epsilon)) = B_d(x, \epsilon),
        \end{equation*}
        %
        as desired.
        Since $x$ and $\epsilon$ are arbitrary, we may construct such an open neighborhood of any $x \in X'$ contained in any open ball $B_d(x, \epsilon)$.
        Thus, the topology on $X'$ is finer than that on $X$.
    \end{proof}
\end{solution}
\newpage

\begin{exercise}[ID=2.20.4]
    Consider the product, uniform, and box topologies on $\RR^\omega$.
    \begin{enumerate}[label={(\alph*)}, align=left, leftmargin=\parindent, listparindent=\parindent, labelwidth=0pt, itemindent=!]
        \item In which topologies are the following functions from $\RR$ to $\RR^\omega$ continuous?
        %
        \begin{align*}
            f(t) &= (t, 2t, 3t, \ldots), \\
            g(t) &= (t, t, t, \ldots), \\
            h(t) &= (t, \tfrac{1}{2} t, \tfrac{1}{3} t, \ldots).
        \end{align*}
        %
        \item In which topologies do the following sequences converge?
        %
        \begin{alignat*}{2}
            & \bw_1 = (1, 1, 1, 1, \ldots),\qquad   & & \bx_1 = (1, 1, 1, 1, \ldots), \\
            & \bw_2 = (0, 2, 2, 2, \ldots),\qquad   & & \bx_2 = (0, \tfrac{1}{2}, \tfrac{1}{2}, \tfrac{1}{2}, \ldots), \\
            & \bw_3 = (0, 0, 3, 3, \ldots),\qquad   & & \bx_3 = (0, 0, \tfrac{1}{3}, \tfrac{1}{3}, \ldots), \\
            & \quad\ldots                           & & \quad\ldots \\
            & \by_1 = (1, 0, 0, 0, \ldots),\qquad   & & \bz_1 = (1, 1, 0, 0, \ldots), \\
            & \by_2 = (\tfrac{1}{2}, \tfrac{1}{2}, 0, 0),                       & & \bz_2 = (\tfrac{1}{2}, \tfrac{1}{2}, 0, 0, \ldots), \\
            & \by_3 = (\tfrac{1}{3}, \tfrac{1}{3}, \tfrac{1}{3}, 0, \ldots),    & & \bz_3 = (\tfrac{1}{3}, \tfrac{1}{3}, 0, 0, \ldots), \\
            & \quad\ldots                                                       & & \quad\ldots
        \end{alignat*}
        %
    \end{enumerate}
\end{exercise}

\begin{solution}
    \begin{enumerate}[label={(\alph*)}, align=left, leftmargin=\parindent, listparindent=\parindent, labelwidth=0pt, itemindent=!]
        \item In which topologies are the following functions from $\RR$ to $\RR^\omega$ continuous?
        %
        \begin{align*}
            f(t) &= (t, 2t, 3t, \ldots), \\
            g(t) &= (t, t, t, \ldots), \\
            h(t) &= (t, \tfrac{1}{2} t, \tfrac{1}{3} t, \ldots).
        \end{align*}
        %
    \end{enumerate}
    %
    \textbf{Claim.} The function $f: \RR \rightarrow \RR^\omega$ is continuous in the product topology, but not in the uniform topology nor in the box topology.
    \begin{proof}~

        \subsubsection*{Product topology}
        %
        Let $t \in \RR$ be arbitrary and let $V = \prod_\alpha V_\alpha$ be a basic open neighborhood of $f(t) = (t, 2t, 3t, \ldots)$ in the product topology.
        Since a basic open set in the product topology only restricts finitely many coordinates, let $J \subset \ZZ_+$ be the set of indices for which $V_\alpha \neq \RR$.
        For each $\alpha \in J$, choose $\epsilon_\alpha > 0$ so that $V_\alpha = (\alpha t - \epsilon_\alpha, \alpha t + \epsilon_\alpha)$.
        Let $\delta = \min\{\epsilon_\alpha / \alpha \mid \alpha \in J\}$ and consider the open interval $U = (t - \delta, t+ \delta)$ centered at $t$ in $\RR$.
        We show that $f(U) \subset V$.

        Given $x \in U$, we have that
        %
        \begin{equation*}
            \alpha (t - \delta) < \alpha x < \alpha (t + \delta).
        \end{equation*}
        %
        Since $\delta \leq \epsilon_{\alpha} / \alpha,~ \forall \alpha \in J$, it follows that
        %
        \begin{equation*}
            \alpha t - \epsilon_\alpha < \alpha x < \alpha t + \epsilon_\alpha
        \end{equation*}
        %
        so that $\alpha x = f(x)_\alpha \in V_\alpha$ for all $\alpha \in J$.
        Furthermore, since $V_\alpha = \RR$ for all other indices, $f(x)_\alpha \in V_\alpha$ also for $\alpha \notin J$ so that $f(x) \in V$.
        Thus, $f(U) \subset V$, proving that $f$ is continuous in the product topology.

        %
        \subsubsection*{Uniform and box topologies}
        %
        Now choose $0 < \epsilon < 1$ and let $V = B_{\overline{\rho}}(f(t), \epsilon)$ be an open ball in the uniform metric $\overline{\rho}$ centered at $f(t)$.
        Let $U = (t - \delta, t + \delta)$ be an open interval centered at $t$ in $\RR$, for some $\delta > 0$.
        We claim that $f(U) \not\subset V$.

        Let $t \neq x \in U$ so that $\abs{t - x} > 0$.
        Then for any index $\alpha > 1 / \abs{t - x}$, we have
        %
        \begin{equation*}
          \abs{\pi_\alpha(f(t)) - \pi_\alpha(f(x))} = \alpha \abs{t - x} > 1
        \end{equation*}
        %
        so that 
        %
        \begin{equation*}
          \overline{d}(\pi_\alpha(f(t)), \pi_\alpha(f(x))) = 1.
        \end{equation*}
        %
        Since for sufficiently large indices, the coorinates of $f(x)$ must differ from those of $f(t)$ by a number greater than 1, it follows that
        %
        \begin{equation*}
          \sup\left\{\overline{d}(\pi_\alpha(f(t)), \pi_\alpha(f(x))) \mid \alpha \in \ZZ_+\right\} = 1 > \epsilon.
        \end{equation*}
        %
        Hence, $f(x) \not\in  V$.
        As this result does not depend on our choice of $\delta$, there exists no $\delta > 0$ such that $f(U) \subset V$.
        It follows that $f$ is not continuous in the uniform topology.

        Finally, since the box topology is finer than the uniform topology, we have that $f$ is also not continuous in the box topology.
    \end{proof}
    \bigskip

    \noindent\textbf{Claim.} The function $g: \RR \rightarrow \RR^\omega$ is continuous in the product and uniform topologies, but not in the box topology.
    %
    \begin{proof}~

        \subsubsection*{Product and uniform topologies}
        %
        Let $t \in \RR$ be arbitrary and let $\epsilon > 0$.
        Let $V = B_{\overline{\rho}}(g(t), \epsilon)$ be an open ball in the uniform metric $\overline{\rho}$ centered at $g(t)$, and let $U = (t - \epsilon, t + \epsilon)$ be an open interval centered at $t$ in $\RR$.
        We show that $g(U) \subset V$.

        Given $x \in U$, we have that
        %
        \begin{equation*}
          \abs{t - x} < \epsilon.
        \end{equation*}
        %
        If we choose $\epsilon \leq 1$, then for all $\alpha \in \ZZ_+$,
        %
        \begin{equation*}
        \overline{d}(\pi_\alpha(g(t)), \pi_\alpha(g(x))) = \abs{t - x} < \epsilon.
        \end{equation*}
        %
        Thus,
        %
        \begin{align*}
          \overline{\rho}(g(t), g(x)) &= \sup\{\overline{d}(\pi_\alpha(g(t)), \pi_\alpha(g(x))) \mid \alpha \in \ZZ_+\} \\
                                      &= \sup\{\abs{t - x}\} \\
                                      &= \abs{t - x} \\
                                      &< \epsilon,
        \end{align*}
        %
        from which it follows that $g(U) \subset V$.

        If, instead, we choose $\epsilon > 1$, then for all $\alpha \in \ZZ_+$,
        %
        \begin{equation*}
          \overline{d}(\pi_\alpha(g(t)), \pi_\alpha(g(x))) \leq 1 < \epsilon.
        \end{equation*}
        %
        Thus,
        %
        \begin{equation*}
          \overline{\rho}(g(t), g(x)) \leq 1 < \epsilon,
        \end{equation*}
        %
        from which it again follows that $g(U) \subset V$.
        Therefore, $g$ is continuous in the uniform topology.

        Furthermore, since the product topology is coarser than the uniform topology, we have that $g$ is also continuous in the product topology.

        %
        \subsubsection*{Box topology}
        %
        Finally, consider the open neighborhood $V = \prod_\alpha (t - \tfrac{1}{\alpha}, t + \tfrac{1}{\alpha})$ of $g(t) = (t, t, t, \ldots)$ in the box topology and let $U = (t - \delta, t + \delta)$ be an open interval centered at $t$ in $\RR$, for some $\delta > 0$.
        We claim that $g(U) \not\subset V$.

        Suppose by way of contradiction that $g(U) \subset V$.
        Given $x \in U$, we have that
        %
        \begin{equation*}
            t - \delta < x < t + \delta.
        \end{equation*}
        %
        Since $g(U) \subset V$, we require that $x = g(x)_\alpha \in (t - \tfrac{1}{\alpha}, t + \tfrac{1}{\alpha})$, for all $\alpha \in \ZZ_+$.
        From here it follows that for all $\alpha \in \ZZ_+$,
        %
        \begin{align*}
            \delta          &\leq 1 / \alpha \\
            \implies \delta &\leq \inf\{1 / \alpha \mid \alpha \in \ZZ_+\} \\
                            &= 0,
        \end{align*}
        %
        contradicting our assumption that $\delta > 0$ and, thus, that $U$ is an open neighborhood of $t$.
        Thus, $g$ is not continuous in the box topology.
    \end{proof}
    \bigskip

    \noindent\textbf{Claim.} The function $h: \RR \rightarrow \RR^\omega$ is continuous in the product and uniform topologies, but not in the box topology.
    %
    \begin{proof}~

        \subsubsection*{Product and uniform topologies}
        %
        Let $t \in \RR$ be arbitrary and let $\epsilon > 0$.
        Let $V = B_{\overline{\rho}}(h(t), \epsilon)$ be an open ball in the uniform metric $\overline{\rho}$ centered at $h(t)$, and let $U = (t - \epsilon, t + \epsilon)$ be an open interval centered at $t$ in $\RR$.
        We show that $h(U) \subset V$.

        Given $x \in U$, we have that
        %
        \begin{equation*}
          \abs{t - x} < \epsilon.
        \end{equation*}
        %
        If we choose $\epsilon \leq 1$, then for all $\alpha \in \ZZ_+$,
        %
        \begin{equation*}
        \overline{d}(\pi_\alpha(h(t)), \pi_\alpha(h(x))) = \abs{t - x} / \alpha.
        \end{equation*}
        %
        Thus,
        %
        \begin{equation*}
          \overline{\rho}(h(t), h(x)) = \sup\{\abs{t - x} / \alpha \mid \alpha \in \ZZ_+\} = \abs{t - x} < \epsilon,
        \end{equation*}
        %
        from which it follows that $h(U) \subset V$.

        If, instead, we choose $\epsilon > 1$, then for all $\alpha \in \ZZ_+$,
        %
        \begin{equation*}
          \overline{d}(\pi_\alpha(h(t)), \pi_\alpha(h(x))) \leq 1 < \epsilon.
        \end{equation*}
        %
        Thus,
        %
        \begin{equation*}
          \overline{\rho}(h(t), h(x)) \leq 1 < \epsilon,
        \end{equation*}
        %
        from which it again follows that $h(U) \subset V$.
        Therefore, $h$ is continuous in the uniform topology.

        Furthermore, since the product topology is coarser than the uniform topology, we have that $h$ is also continuous in the product topology.

        %
        \subsubsection*{Box topology}
        %
        Finally, consider the open neighborhood $V = \prod_\alpha (\tfrac{t}{\alpha} - \tfrac{1}{\alpha^2}, \tfrac{t}{\alpha} + \tfrac{1}{\alpha^2})$ of $h(t) = (t, \tfrac{1}{2} t, \tfrac{1}{3} t, \ldots)$ in the box topology and let $U = (t - \delta, t + \delta)$ be an open interval centered at $t$ in $\RR$, for some $\delta > 0$.
        We claim that $h(U) \not\subset V$.

        Suppose by way of contradiction that $h(U) \subset V$.
        Given $x \in U$, we have that
        %
        \begin{equation*}
            (t - \delta) / \alpha < x / \alpha < (t + \delta) / \alpha.
        \end{equation*}
        %
        Since $h(U) \subset V$, we require that $x = h(x)_\alpha \in (t - \tfrac{1}{\alpha^2}, t + \tfrac{1}{\alpha^2})$, for all $\alpha \in \ZZ_+$.
        From here it follows that for all $\alpha \in \ZZ_+$,
        %
        \begin{align*}
            \delta / \alpha &\leq 1 / \alpha^2 \\
            \implies \delta &\leq 1 / \alpha \\
            \implies \delta &\leq \inf\{1 / \alpha \mid \alpha \in \ZZ_+\} \\
                            &= 0,
        \end{align*}
        %
        contradicting our assumption that $\delta > 0$ and, thus, that $U$ is an open neighborhood of $t$.
        Thus, $h$ is not continuous in the box topology.
    \end{proof}
    \bigskip

    \begin{enumerate}[label={(\alph*)}, align=left, leftmargin=\parindent, listparindent=\parindent, labelwidth=0pt, itemindent=!]
        \addtocounter{enumi}{1} 
        \item In which topologies do the following sequences converge?
        %
        \begin{alignat*}{2}
            & \bw_1 = (1, 1, 1, 1, \ldots),\qquad   & & \bx_1 = (1, 1, 1, 1, \ldots), \\
            & \bw_2 = (0, 2, 2, 2, \ldots),\qquad   & & \bx_2 = (0, \tfrac{1}{2}, \tfrac{1}{2}, \tfrac{1}{2}, \ldots), \\
            & \bw_3 = (0, 0, 3, 3, \ldots),\qquad   & & \bx_3 = (0, 0, \tfrac{1}{3}, \tfrac{1}{3}, \ldots), \\
            & \quad\ldots                           & & \quad\ldots \\
            & \by_1 = (1, 0, 0, 0, \ldots),\qquad   & & \bz_1 = (1, 1, 0, 0, \ldots), \\
            & \by_2 = (\tfrac{1}{2}, \tfrac{1}{2}, 0, 0),                       & & \bz_2 = (\tfrac{1}{2}, \tfrac{1}{2}, 0, 0, \ldots), \\
            & \by_3 = (\tfrac{1}{3}, \tfrac{1}{3}, \tfrac{1}{3}, 0, \ldots),    & & \bz_3 = (\tfrac{1}{3}, \tfrac{1}{3}, 0, 0, \ldots), \\
        \end{alignat*}
    \end{enumerate}
    %
    \textbf{Claim.} The sequence $\bw_1, \bw_2, \ldots$ converges in the product topology, but not in the uniform nor box topologies.
    %
    \begin{proof}~

        \subsubsection*{Product topology}
        %
        Let $\mathbf{a} = (0, 0, \ldots) \in \RR^\omega$ and let
        %
        \begin{equation*}
            V = \prod_{\alpha \in \ZZ_+} V_\alpha
        \end{equation*}
        %
        be an arbitrary basic open neighborhood of $\mathbf{a}$ in the product topology.
        By definition of the product topology, there must exist $N \in \ZZ_+$ such that $V_\alpha = \RR$ for all $\alpha \geq N$.
        We claim that for all integers $n \geq N$, each coordinate $\pi_\alpha(\bw_n)$ of $\bw_n$ lies in $V_\alpha$, so that $\bw_n \in V$.

        Let $n \geq N$.
        Note that $\bw_n$ can be written coordinatewise as
        %
        \begin{equation*}
            \pi_\alpha(\bw_n) = \begin{cases}
                0   & \text{if } \alpha < n, \\
                n   & \text{if } \alpha \geq n.
            \end{cases}
        \end{equation*}
        %
        For indices $\alpha < n$, we have that
        %
        \begin{equation*}
            \pi_\alpha(\bw_n) = \pi_\alpha(\mathbf{a}) = 0.
        \end{equation*}
        %
        Hence, since $V_\alpha$ is a neighborhood of $0$, it follows that $\pi_\alpha(\bw_n) \in V_\alpha$.
        For $\alpha \geq n$, we have that
        %
        \begin{equation*}
            \pi_\alpha(\bw_n) = n,
        \end{equation*}
        %
        which clearly lies in $V_\alpha = \RR$, proving that $\bw_n$ lies in $V$.
        Since $V$ is an arbitrary open neighborhood of $\mathbf{a}$, we see that the sequence $\bw_1, \bw_2, \ldots$ converges to $\mathbf{a}$ in the product topology.

        %
        \subsubsection*{Uniform and box topologies}
        %
        Now let $\mathbf{a} = (a_1, a_2, \ldots)$ be an arbitrary point in $\RR^\omega$ and let $B_{\overline{\rho}}(\mathbf{a}, \tfrac{1}{2})$ be an open ball centered at $\mathbf{a}$ in the uniform topology.
        We claim that there exists no $N \in \ZZ_+$ such that for all $n \geq N,~ \bw_n \in B_{\overline{\rho}}(\mathbf{a}, \tfrac{1}{2})$.

        Suppose that for some $N \in \ZZ_+,~ \bw_N \in B_{\overline{\rho}}(\mathbf{a}, \tfrac{1}{2})$.
        Since $\overline{\rho}(\mathbf{a}, \bw_N) < \tfrac{1}{2}$, it follows that $\abs{a_\alpha - \pi_\alpha(\bw_N)} < \tfrac{1}{2}$ for all $\alpha \in \ZZ_+$.
        In particular,
        %
        \begin{alignat*}{2}
            &           && \quad \abs{a_{N+1} - \pi_{N+1}(\bw_N)} < \tfrac{1}{2} \\
            & \implies  && \quad a_{N+1} - \tfrac{1}{2} < N < a_{N+1} + \tfrac{1}{2} \\
            & \implies  && \quad a_{N+1} + \tfrac{1}{2} < N + 1 = \pi_{N+1}(\bw_{N+1}) \\
            & \implies  && \quad \abs{a_{N+1} - \pi_{N+1}(\bw_{N+1})} > \tfrac{1}{2} \\
            & \implies  && \quad \bw_{N+1} \not\in B_{\overline{\rho}}(\mathbf{a}, \tfrac{1}{2}).
        \end{alignat*}
        %
        Therefore, if for some $N \in \ZZ_+$ we have $\bw_N \in B_{\overline{\rho}}(\mathbf{a}, \tfrac{1}{2})$, then $\bw_{N+1} \not\in B_{\overline{\rho}}(\mathbf{a}, \tfrac{1}{2})$.
        Thus, the sequence $\bw_1, \bw_2, \ldots$ does not converge to $\mathbf{a}$.
        Since $\mathbf{a} \in \RR^\omega$ is arbitrary, the sequence does not converge to any point in $\RR^\omega$ in the uniform topology.
        Furthermore, since the uniform topology is coarser than the box topology, the sequence fails to converge also in the box topology.
    \end{proof}
    \bigskip

    \noindent\textbf{Claim.} The sequence $\bx_1, \bx_2, \ldots$ converges in the product and uniform topologies, but not in the box topology.
    %
    \begin{proof}~

        \subsubsection*{Product and uniform topologies}
        %
        Let $\mathbf{a} = (0, 0, \ldots) \in \RR^\omega$.
        Fix $\epsilon > 0$ and let $B_{\overline{\rho}}(\mathbf{a}, \epsilon)$ be a basic open neighborhood of $\mathbf{a}$ in the uniform topology.
        We claim that there exists $N \in \ZZ_+$ such that for any integer $n \geq N$, $\bx_n \in B_{\overline{\rho}}(\mathbf{a}, \epsilon)$.

        Choose $N > 1 / \epsilon$ and let $n \geq N$.
        Note that $\bx_n$ can be written coordinatewise as
        %
        \begin{equation*}
            \pi_\alpha(\bx_n) = \begin{cases}
                0               & \text{if } \alpha < n, \\
                \tfrac{1}{n}    & \text{if } \alpha \geq n.
            \end{cases}
        \end{equation*}
        %
        Then 
        %
        \begin{align*}
          \overline{\rho}(\mathbf{a}, \bx_n)  &= \sup\{\overline{d}(a_\alpha, \pi_\alpha(\bx_n)) \mid \alpha \in \ZZ_+\} \\
                                              &= \sup\{0, 1/n\} \\
                                              &= 1/n \\
                                              &\leq 1/N \\
                                              &< \epsilon.
        \end{align*}
        %
        Thus, $\bx_n \in B_{\overline{\rho}}(\mathbf{a}, \epsilon)$.
        It follows that the sequence $\bx_1, \bx_2, \ldots$ converges to $\mathbf{a}$ in the uniform topology.
        Furthermore, since the product topology is coarser than the uniform topology, the sequence also converges to $\mathbf{a}$ in the product topology.

        %
        \subsubsection*{Box topology}
        %
        Now let $\mathbf{a} = (a_1, a_2, \ldots)$ be an arbitrary point in $\RR^\omega$ and let
        %
        \begin{equation*}
            V = \prod_{\alpha\in\ZZ_+} V_\alpha
        \end{equation*}
        %
        be a basic open neighborhood of $\mathbf{a}$ in the box topology, where
        %
        \begin{equation*}
            V_\alpha = (a_\alpha - \tfrac{1}{\alpha}, a_\alpha + \tfrac{1}{\alpha})
        \end{equation*}
        %
        for each $\alpha \in \ZZ_+$.
        We claim that there exists no $N \in \ZZ_+$ such that for all $n \geq N$, $\bx_n \in V$.

        Suppose that for some $N \in \ZZ_+$, $\bx_N \in V$, \textit{i.e.}, for all $\alpha \in \ZZ_+,~ \pi_\alpha(\bx_N) \in V_\alpha$.
        In particular, for $\alpha = 2 N (N + 1)$
        %
        \begin{alignat*}{2}
            &           && \quad \pi_\alpha(\bx_N) \in V_\alpha \\
            & \implies  && \quad a_\alpha - \tfrac{1}{\alpha} < \tfrac{1}{N} < a_\alpha + \tfrac{1}{\alpha} \\
            & \implies  && \quad \tfrac{1}{N} - \tfrac{2}{\alpha} < a_\alpha - \tfrac{1}{\alpha} \\
            & \implies  && \quad \tfrac{1}{N + 1} < a_\alpha - \tfrac{1}{\alpha} \\
            & \implies  && \quad \pi_\alpha(\bx_{N+1}) = \tfrac{1}{N+1} \not\in V_\alpha.
        \end{alignat*}
        %
        Therefore, if for some $N \in \ZZ_+$ we have that $\bx_N \in V$, then $\bx_{N+1} \not\in V$.
        Thus, the sequence $\bx_1, \bx_2, \ldots$ does not converge to $\mathbf{a}$.
        Since $\mathbf{a} \in \RR^\omega$ is arbitrary, the sequence does not converge to any point in $\RR^\omega$ in the box topology.
    \end{proof}
    \bigskip

    \noindent\textbf{Claim.} The sequence $\by_1, \by_2, \ldots$ converges in the product and uniform topologies, but not in the box topology.
    %
    \begin{proof}~

        \subsubsection*{Product and uniform topologies}
        %
        Let $\mathbf{a} = (0, 0, \ldots) \in \RR^\omega$.
        Fix $\epsilon > 0$ and let $B_{\overline{\rho}}(\mathbf{a}, \epsilon)$ be a basic open neighborhood of $\mathbf{a}$ in the uniform topology.
        We claim that there exists $N \in \ZZ_+$ such that for any integer $n \geq N$, $\by_n \in B_{\overline{\rho}}(\mathbf{a}, \epsilon)$.

        Choose $N > 1 / \epsilon$ and let $n \geq N$.
        Note that $\by_n$ can be written coordinatewise as
        %
        \begin{equation*}
            \pi_\alpha(\by_n) = \begin{cases}
                \tfrac{1}{n}  & \text{if } \alpha \leq n, \\
                0             & \text{if } \alpha > n.
            \end{cases}
        \end{equation*}
        %
        Then 
        %
        \begin{align*}
          \overline{\rho}(\mathbf{a}, \by_n)  &= \sup\{\overline{d}(a_\alpha, \pi_\alpha(\by_n)) \mid \alpha \in \ZZ_+\} \\
                                              &= \sup\{0, 1/n\} \\
                                              &= 1/n \\
                                              &\leq 1/N \\
                                              &< \epsilon.
        \end{align*}
        %
        Thus, $\by_n \in B_{\overline{\rho}}(\mathbf{a}, \epsilon)$.
        It follows that the sequence $\by_1, \by_2, \ldots$ converges to $\mathbf{a}$ in the uniform topology.
        Furthermore, since the product topology is coarser than the uniform topology, the sequence also converges to $\mathbf{a}$ in the product topology.

        %
        \subsubsection*{Box topology}
        %
        Now, let $\mathbf{a} = (a_1, a_2, \ldots)$ be an arbitrary point in $\RR^\omega$ and let
        %
        \begin{equation*}
            V = \prod_{\alpha\in\ZZ_+} V_\alpha
        \end{equation*}
        %
        be a basic open neighborhood of $\mathbf{a}$ in the box topology, where 
        %
        \begin{equation*}
            V_\alpha = (a_\alpha - \tfrac{1}{2\alpha}, a_\alpha + \tfrac{1}{2\alpha})
        \end{equation*}
        %
        for each $\alpha \in \ZZ_+$.
        We claim that there exists no $N \in \ZZ_+$ such that for all $n \geq N,~ \by_n \in V$.

        Suppose that for some $N \in \ZZ_+$, we have that $\by_N \in V$, \textit{i.e.}, for all $\alpha \in \ZZ_+,~ \pi_\alpha(\by_N) \in V_\alpha$.
        In particular, for $\alpha = N+1$,
        %
        \begin{alignat*}{2}
            &           && \quad \pi_{N+1}(\by_N) \in V_{N+1} \\
            & \implies  && \quad a_{N+1} - \tfrac{1}{2(N+1)} < 0 < a_{N+1} + \tfrac{1}{2(N+1)} \\
            & \implies  && \quad a_{N+1} + \tfrac{1}{2(N+1)} < \tfrac{1}{N + 1} = \pi_{N+1}(\by_{N+1}) \\
            & \implies  && \quad \pi_{N+1}(\by_{N+1}) \not\in V_{N+1}.
        \end{alignat*}
        %
        Therefore, if for some $N \in \ZZ_+$ we have that $\by_N \in V$, then $\by_{N+1} \not\in V$.
        Thus, the sequence $\by_1, \by_2, \ldots$ does not converge to $\mathbf{a}$.
        Since $\mathbf{a} \in \RR^\omega$ is arbitrary, the sequence does not converge to any point in $\RR^\omega$ in the box topology.
    \end{proof}
    \bigskip

    \noindent\textbf{Claim.} The sequence $\bz_1, \bz_2, \ldots$ converges in the product, uniform and box topologies.
    %
    \begin{proof}
        Let $\mathbf{a} = (0, 0, \ldots) \in \RR^\omega$ and let
        %
        \begin{equation*}
            V = \prod_{\alpha\in\ZZ_+} V_\alpha
        \end{equation*}
        %
        be an arbitrary open neighborhood of $\mathbf{a}$ in the box topology, where $V_\alpha = (-\epsilon_\alpha, \epsilon_\alpha)$ with $\epsilon_\alpha > 0$, for all $\alpha \in \ZZ_+$.
        We claim there exists $N \in \ZZ_+$ such that for any integer $n \geq N$, $\bz_n \in V$.

        Define $\epsilon = \min\{\epsilon_1, \epsilon_2\}$ and choose $N > 1 / \epsilon$.
        Let $n \geq N$ and note that $\bz_n$ can be written coordinatewise as
        %
        \begin{equation*}
            \pi_\alpha(\bz_n) = \begin{cases}
                \tfrac{1}{n}    & \text{if } \alpha \leq 2, \\
                0               & \text{if } \alpha > 2.
            \end{cases}
        \end{equation*}
        %
        For indices $\alpha > 2$, we have that
        %
        \begin{equation*}
            \pi_\alpha(\bz_n) = \pi_\alpha(\mathbf{a}) = 0 \in (-\epsilon_\alpha, \epsilon_\alpha).
        \end{equation*}
        %
        For $\alpha \leq 2$, we have that
        %
        \begin{equation*}
            \pi_\alpha(\bz_n) = \tfrac{1}{n} \leq \tfrac{1}{N} \in (-\epsilon, \epsilon) \subset (-\epsilon_\alpha, \epsilon_\alpha).
        \end{equation*}
        %
        Then, for all $\alpha \in \ZZ_+,~ \pi_\alpha(\bz_n) \in V_\alpha$ and so $\bz_n \in V$.
        Thus, the sequence $\bz_1, \bz_2, \ldots$ converges to $\mathbf{a}$ in the box topology.
        Furthermore, since both the product and uniform topologies are coarser than the box topology, the sequence also converges to $\mathbf{a}$ in the product and uniform topologies.
    \end{proof}
\end{solution}
\newpage

\begin{exercise}[ID=2.20.5]
    Let $\RR^\infty$ be the subset of $\RR^\omega$ consisting of all sequences that are eventually zero.
    What is the closure of $\RR^\infty$ in the uniform topology?
\end{exercise}
%
\begin{solution}
  \textbf{Claim.} The closure of $\RR^\infty$ in the uniform topology is given by the set of sequences
  %
  \begin{equation*}
    A = \{ (x_1, x_2, \ldots) \in \RR^\omega \mid \forall \epsilon > 0,~ \exists N \in \ZZ_+ ~\textrm{such that}~ \overline{\rho}(\{ x_N, x_{N+1}, \ldots \}, \mathbf{0}) < \epsilon \},
  \end{equation*}
  %
  where $\mathbf{0} = (0, 0, \ldots)$.
  \begin{proof}
    Let $\bx \in A$.
    Fix $\epsilon > 0$ and let $B_{\overline{\rho}}(\bx, \epsilon)$ be an open ball centered at $\bx$ in the uniform topology.
    By definition of the set $A$, there exists $N \in \ZZ_+$ such that $\overline{\rho}(\{ x_N, x_{N+1}, \ldots \}, \mathbf{0}) < \epsilon$.
    Consider the point $\by \in \RR^\infty$ given by the formula
    %
    \begin{equation*}
        y_i =
        \begin{cases}
            x_i & \text{if } i < N, \\
            0   & \text{if } i \geq N.
        \end{cases}
    \end{equation*}
    %
    Then
    %
    \begin{align*}
      \overline{\rho}(\bx, \by) &= \sup\{\min(\abs{x_i - y_i}, 1) \mid i \in \ZZ_+\} \\
                                &= \sup\{\min(\abs{x_N - 0}, 1), \min(\abs{x_{N+1} - 0}, 1), \ldots\} \\
                                &= \overline{\rho}(\{ x_N, x_{N+1}, \ldots \}, \mathbf{0}).
    \end{align*}
    %
    Since $\bx \in A$, we have that $\overline{\rho}(\bx, \by) = \overline{\rho}(\{ x_N, x_{N+1}, \ldots \}, \mathbf{0}) < \epsilon$.
    Thus, $\by \in B_{\overline{\rho}}(\bx, \epsilon) \cap \RR^\infty$, from which it follows that $\bx \in \overline{\RR^\infty}$.

    Conversely, suppose $\bx$ is an arbitrary point in the closure of $\RR^\infty$.
    Then for every $\epsilon > 0$, there exists $\by \in \RR^\infty$ such that
    %
    \begin{equation*}
      \overline{\rho}(\bx, \by) < \epsilon.
    \end{equation*}
    %
    Since $\by \in \RR^\infty$, there exists $N \in \ZZ_+$ such that $y_n = 0$ for all $n \geq N$.
    Consequently,
    %
    \begin{equation*}
      \overline{\rho}(\bx, \by) = \sup\big( \{ \min(\abs{x_i - y_i}, 1) \mid i < N \} \cup \{ \min(\abs{x_n - 0}, 1) \mid n \geq N \big) < \epsilon.
    \end{equation*}
    %
    This implies that
    %
    \begin{equation*}
      \overline{\rho}(\{ x_N, x_{N+1}, \ldots \}, \mathbf{0}) < \epsilon,
    \end{equation*}
    %
    so $\bx \in A$.
    Hence, $\overline{\RR^\infty} = A$ in the uniform topology.
  \end{proof}
\end{solution}
\newpage

%
\begin{exercise}[ID=2.20.6]
  Let $\overline{\rho}$ be the uniform metric on $\RR^\omega$.
  Given $\bx = (x_1, x_2, \ldots) \in \RR^\omega$ and given $0 < \epsilon < 1$, let
  %
  \begin{equation*}
    U(\bx, \epsilon) = (x_1 - \epsilon, x_1 + \epsilon) \times \cdots \times (x_n - \epsilon, x_n + \epsilon) \times \cdots.
  \end{equation*}
  \begin{enumerate}[label={(\alph*)}, align=left, leftmargin=\parindent, listparindent=\parindent, labelwidth=0pt, itemindent=!]
    \item Show that $U(\bx, \epsilon)$ is not equal to the $\epsilon$-ball $B_{\overline{\rho}}(\bx, \epsilon)$.
    \item Show that $U(\bx, \epsilon)$ is not even open in the uniform topology.
    \item Show that
      %
      \begin{equation*}
        B_{\overline{\rho}}(\bx, \epsilon) = \bigcup_{\delta < \epsilon} U(\bx, \delta).
      \end{equation*}
      %
  \end{enumerate}
\end{exercise}
%
\begin{solution}
  \begin{enumerate}[label={(\alph*)}, align=left, leftmargin=\parindent, listparindent=\parindent, labelwidth=0pt, itemindent=!]
    \item Show that $U(\bx, \epsilon)$ is not equal to the $\epsilon$-ball $B_{\overline{\rho}}(\bx, \epsilon)$.
  \end{enumerate}
  %
  \begin{proof}
    Let $0 < \epsilon < 1$.
    Consider the point $\by \in \RR^\omega$ whose components are given by
    %
    \begin{equation*}
      y_\alpha = x_\alpha + \epsilon (1 - \tfrac{1}{\alpha})
    \end{equation*}
    %
    for all $\alpha \in \ZZ_+$.
    Given $\alpha \in \ZZ_+$, we have
    %
    \begin{equation*}
      x_\alpha - \epsilon < x_\alpha + \epsilon (1 - \tfrac{1}{\alpha}) < x_\alpha + \epsilon
    \end{equation*}
    %
    so that $\by \in U(\bx, \epsilon)$.
    Furthermore,
    %
    \begin{align*}
      \overline{\rho}(\bx, \by) &= \sup\{ \min(\abs{x_\alpha - y_\alpha}, 1) \mid \alpha \in \ZZ_+ \} \\
                                &= \sup\{ \min(\epsilon (1 - \tfrac{1}{\alpha}), 1) \mid \alpha \in \ZZ_+ \} \\
                                &= \sup\{ \epsilon (1 - \tfrac{1}{\alpha}) \mid \alpha \in \ZZ_+ \} \\
                                &= \epsilon.
    \end{align*}
    %
    Thus, $\by \not\in B_{\overline{\rho}}(\bx, \epsilon)$.
    It follows that $U(\bx, \epsilon)$ is not equal to the $\epsilon$-ball $B_{\overline{\rho}}(\bx, \epsilon)$.
  \end{proof}
  \bigskip

  \begin{enumerate}[label={(\alph*)}, align=left, leftmargin=\parindent, listparindent=\parindent, labelwidth=0pt, itemindent=!]
    \addtocounter{enumi}{1} 
    \item Show that $U(\bx, \epsilon)$ is not even open in the uniform topology.
  \end{enumerate}
  %
  \begin{proof}
    Let $0 < \epsilon < 1$ and let $\by \in U(\bx, \epsilon)$ with
    %
    \begin{equation*}
      y_\alpha = x_\alpha + \epsilon (1 - \tfrac{1}{\alpha})
    \end{equation*}
    %
    for all $\alpha \in \ZZ_+$.
    Now fix $0 < \delta < \epsilon$ and let $\bz \in \RR^\omega$ with
    %
    \begin{equation*}
      z_\alpha = y_\alpha + \tfrac{\delta}{2} (1 - \tfrac{1}{\alpha}),
    \end{equation*}
    %
    for all $\alpha \in \ZZ_+$.
    We then have
    %
    \begin{equation*}
      \overline{\rho}(\by, \bz) = \sup\{ \min(\tfrac{\delta}{2} (1 - \tfrac{1}{\alpha}), 1) \mid \alpha \in \ZZ_+\} = \tfrac{\delta}{2} < \delta,
    \end{equation*}
    %
    from which it follows that $\bz \in B_{\overline{\rho}}(\by, \delta)$.

    Now consider the distance between $x_\alpha$ and $z_\alpha$ for $\alpha > 1 + 2 \epsilon / \delta$.
    We have
    %
    \begin{equation*}
      \abs{x_\alpha - z_\alpha} = \left(\epsilon + \tfrac{\delta}{2}\right) \left(1 - \tfrac{1}{\alpha}\right) > \left(\epsilon + \tfrac{\delta}{2}\right) \left(1 - \frac{1}{1 + 2 \epsilon / \delta}\right) = \epsilon,
    \end{equation*}
    %
    so that $\bz \not\in U(\bx, \epsilon)$ and, hence, $B_{\overline{\rho}}(\by, \delta) \not\subset U(\bx, \epsilon)$.
    Then $\by$ is a point in $U(\bx, \epsilon)$ for which there is no open neighborhood contained in $U(\bx, \epsilon)$.
    Thus, $U(\bx, \epsilon)$ is not closed in the uniform topology.
  \end{proof}
  \bigskip

  \begin{enumerate}[label={(\alph*)}, align=left, leftmargin=\parindent, listparindent=\parindent, labelwidth=0pt, itemindent=!]
    \addtocounter{enumi}{2} 
    \item Show that
      %
      \begin{equation*}
        B_{\overline{\rho}}(\bx, \epsilon) = \bigcup_{\delta < \epsilon} U(\bx, \delta).
      \end{equation*}
      %
  \end{enumerate}
  %
  \begin{proof}
    Let $\by \in \bigcup_{\delta < \epsilon} U(\bx, \delta)$.
    Then there exists $\delta < \epsilon$ such that $\by \in U(\bx, \delta)$ so that for all $\alpha \in \ZZ_+$, $\abs{x_\alpha - y_\alpha} < \delta < \epsilon$.
    It follows that
    %
    \begin{equation*}
      \overline{\rho}(\bx, \by) < \epsilon.
    \end{equation*}
    %
    Thus, $\by \in B_{\overline{\rho}}(\bx, \epsilon)$ and so $\bigcup_{\delta < \epsilon} U(\bx, \delta) \subset B_{\overline{\rho}}(\bx, \epsilon)$.

    Conversely, let $\by \in B_{\overline{\rho}}(\bx, \epsilon)$.
    Then there exists $\delta \in \RR$ such that for all $\alpha \in \ZZ_+$,
    %
    \begin{equation*}
      \abs{x_\alpha - y_\alpha} \leq \overline{\rho}(\bx, \by) < \delta < \epsilon \implies y_\alpha \in (x_\alpha - \delta, x_\alpha + \delta).
    \end{equation*}
    %
    It follows that $\by \in U(\bx, \delta) \subset \bigcup_{\delta < \epsilon} U(\bx, \delta)$.
    Thus, $B_{\overline{\rho}}(\bx, \epsilon) \subset \bigcup_{\delta < \epsilon} U(\bx, \delta)$ and so
    %
    \begin{equation*}
      B_{\overline{\rho}}(\bx, \epsilon) = \bigcup_{\delta < \epsilon} U(\bx, \delta).
    \end{equation*}
    %
  \end{proof}
\end{solution}
\newpage

%
\begin{exercise}[ID=2.20.8]
  Let $X$ be the subset of $\RR^\omega$ consisting of all sequences $\bx$ such that $\sum x_i^2$ converges.
  Then the formula
  %
  \begin{equation*}
    d(\bx, \by) = \left[ \sum_{i=1}^\infty (x_i - y_i)^2 \right]^{1/2}
  \end{equation*}
  %
  defines a metric on $X$.
  On $X$ we have the three topologies it inherits from the box, uniform, and product topologies on $\RR^\omega$.
  We have also the topology given by the metric $d$, which we call the $\ell^2$\textit{-topology}.
  %
  \begin{enumerate}[label={(\alph*)}, align=left, leftmargin=\parindent, listparindent=\parindent, labelwidth=0pt, itemindent=!]
    \item
      Show that on $X$, we have the inclusions
      %
      \begin{equation*}
        \text{box topology} \supset \ell^2\text{-topology} \supset \text{uniform topology}.
      \end{equation*}
      %
    \item
      The set $\RR^\infty$ of all sequences that are eventually zero is contained in $X$.
      Show that the four topologies that $\RR^\infty$ inherits as a subspace of $X$ are all distinct.
    \item
      The set
      %
      \begin{equation*}
        H = \prod_{n\in\ZZ_+} [0, 1/n]
      \end{equation*}
      %
      is contained in $X$; it is called the \textit{\textbf{Hilbert cube}}.
      Compare the four topologies that $H$ inherits as a subspace of $X$.
  \end{enumerate}
\end{exercise}
%
\begin{solution}
  \begin{enumerate}[label={(\alph*)}, align=left, leftmargin=\parindent, listparindent=\parindent, labelwidth=0pt, itemindent=!]
    \item
      Show that on $X$, we have the inclusions
      %
      \begin{equation*}
        \text{box topology} \supset \ell^2\text{-topology} \supset \text{uniform topology}.
      \end{equation*}
      %
  \end{enumerate}
  %
  \begin{proof}~

    \subsubsection*{$\ell^2$-topology $\supset$ uniform topology}
    %
    Let $\bx \in X$ and let $\epsilon > 0$ and consider the basic open set $U = X \cap B_{\overline{\rho}}(\bx, \epsilon)$ of $X$ in the topology it inherits as a subspace from the uniform topology on $\RR^\omega$.
    Let $0 < \delta < \epsilon$ and let $\by \in B_d(\bx, \delta)$.
    Then
    %
    \begin{alignat*}{2}
      &           && \qquad \left[ \sum_{i=1}^\infty \abs{x_i - y_i}^2 \right]^{1/2} < \delta \\
      & \implies  && \qquad \abs{x_i - y_i} < \delta,~ \forall i \in \ZZ_+ \\
      & \implies  && \qquad \overline{\rho}(\bx, \by) \leq \delta < \epsilon,
    \end{alignat*}
    %
    so that $\by \in B_{\overline{\rho}}(\bx, \epsilon)$.
    Therefore, $B_d(\bx, \delta) \subset B_{\overline{\rho}}(\bx, \epsilon)$, from which it follows that $X \cap B_d(\bx, \delta)$ is a basic open neighborhood of $\bx$ in the $\ell^2$-topology contained in $U$.
    Thus, $U$ is open in the $\ell^2$-topology induced by $d$, and so we have
    %
    \begin{equation*}
      \ell^2\text{-topology} \supset \text{uniform topology}.
    \end{equation*}

    %
    \subsubsection*{Box topology $\supset$ $\ell^2$-topology}
    %
    Let $\bx \in X$ and let $\epsilon > 0$ and consider the open ball $B_d(\bx, \epsilon)$ centered at $\bx$ with respect to the metric $d$.
    Let $\by \in V = \prod_{i \in \ZZ_+} V_i$, where
    %
    \begin{equation*}
      V_i = (x_i - \frac{\epsilon}{2^{i/2}}, x_i + \frac{\epsilon}{2^{i/2}}).
    \end{equation*}
    %
    Then
    %
    \begin{alignat*}{2}
      &           && \qquad \forall i \in \ZZ_+,~ \abs{x_i - y_i} < \frac{\epsilon}{2^{i/2}} \\
      & \implies  && \qquad \sum_{i=1}^\infty \abs{x_i - y_i}^2 < \epsilon^2 \\
      & \implies  && \qquad d(\bx, \by) < \epsilon,
    \end{alignat*}
    %
    so that $\by \in B_d(\bx, \epsilon)$.
    Therefore, $V \subset B_d(\bx, \epsilon)$.
    As $V$ is a basic open neighborhood of $\bx$ in the box topology, $B_d(\bx, \epsilon)$ is open in the box topology.
    Thus, we have
    %
    \begin{equation*}
      \text{box topology} \supset \ell^2\text{-topology}.
    \end{equation*}
  \end{proof}
  \bigskip

  %
  \begin{enumerate}[label={(\alph*)}, align=left, leftmargin=\parindent, listparindent=\parindent, labelwidth=0pt, itemindent=!]
    \addtocounter{enumi}{1} 
    \item
      The set $\RR^\infty$ of all sequences that are eventually zero is contained in $X$.
      Show that the four topologies that $\RR^\infty$ inherits as a subspace of $X$ are all distinct.
  \end{enumerate}
  %
  \begin{proof}
  \end{proof}
\end{solution}
\newpage

%
\begin{exercise}[ID=2.20.11]
  Show that if $d$ is a metric for $X$, then
  %
  \begin{equation*}
    d'(x, y) = d(x, y) / (1 + d(x, y))
  \end{equation*}
  %
  is a bounded metric that gives the topology of $X$.
\end{exercise}
%
\begin{solution}
  \begin{proof}
    We must prove that i) $d'$ is a metric, ii) $d'$ is bounded and iii) $d'$ induces the same topology on $X$ as $d$.
    %
    \subsubsection*{i) $d'$ is a metric}
    %
    Let $x, y, z \in X$.
    We will show 1) $d'(x, y) \geq 0$ with equality if and only if $x = y$, 2) $d(x, y) = d(y, x)$ and 3) $d(x, y) + d(y, z) \geq d(x, z)$.
    \bigskip

    \noindent 1) Suppose $x \neq y$.
    Then
    %
    \begin{alignat*}{2}
      &           && \qquad d(x, y) > 0 \\
      & \implies  && \qquad 1 + d(x, y) > 0 \\
      & \implies  && \qquad 1 / (1 + d(x, y)) > 0 \\
      & \implies  && \qquad d(x, y) / (1 + d(x, y)) = d'(x, y) > 0.
    \end{alignat*}
    %
    Now, suppose $x = y$.
    Then
    %
    \begin{equation*}
      d(x, y) = 0 \implies d(x, y) / (1 + d(x, y)) = d'(x, y) = 0.
    \end{equation*}
    %
    Conversely, suppose $d'(x, y) = 0$.
    Then
    %
    \begin{equation*}
      d(x, y) / (1 + d(x, y)) = 0 \implies d(x, y) = 0 \implies x = y.
    \end{equation*}
    %
    Thus, $d'(x, y) \geq 0$ for all $x, y \in X$ with equality if and only if $x = y$.
    \bigskip

    \noindent 2) $d'(x, y) = d(x, y) / (1 + d(x, y)) = d(y, x) / (1 + d(y, x)) = d'(y, x)$.
    \bigskip

    \noindent 3) TODO


    \subsubsection*{ii) $d'$ is bounded}
    %
    Let $x, y \in X$.
    Then
    %
    \begin{alignat*}{2}
      &           && \qquad 0 \leq d(x, y) < 1 + d(x, y) \\
      & \implies  && \qquad d(x, y) / (1 + d(x, y)) = d'(x, y) < 1.
    \end{alignat*}
    %
    Thus, $d'(x, y)$ is bounded.

    \subsubsection*{iii) $d'$ induces the same topology on $X$ as $d$}
    %
    Let $x \in X$ and $\epsilon > 0$.

    Fixing $\delta = \epsilon / (1 + \epsilon)$, for $y \in B_{d'}(x, \delta)$ we have
    %
    \begin{alignat*}{2}
      &           && \qquad d'(x, y) = d(x, y) / (1 + d(x, y)) < \delta \\
      & \implies  && \qquad d(x, y) < \delta / (1 - \delta) = \epsilon.
    \end{alignat*}
    %
    Thus, $y \in B_d(x, \epsilon)$ and so $B_{d'}(x, \delta) \subset B_d(x, \epsilon)$.
    It follows that the topology induced on $X$ by $d'$ is finer than the topology induced by $d$.

    Conversely, fixing $\delta = \epsilon$, for $y \in B_d(x, \delta)$ we have
    %
    \begin{alignat*}{2}
      &           && \qquad 0 \leq d(x, y) < \delta \\
      & \implies  && \qquad d(x, y) / (1 + d(x, y)) = d'(x, y) < \delta = \epsilon.
    \end{alignat*}
    %
    Thus, $y \in B_{d'}(x, \epsilon)$ and so $B_d(x, \delta) \subset B_{d'}(x, \epsilon)$.
    It follows that the topology induced on $X$ by $d$ is finer than the topology induced by $d'$.
    Therefore, the topologies induced on $X$ by $d$ and by $d'$ are equivalent.
  \end{proof}
\end{solution}

