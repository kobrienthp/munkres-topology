\subsection{The Metric Topology (continued)}

\begin{exercise}[ID=2.21.1]
  Let $A \subset X$.
  If $d$ is a metric for the topology of $X$, show that $\restr{d}{A \times A}$ is a metric for the subspace topology on $A$.
\end{exercise}
%
\begin{solution}
  \begin{proof}
    Since $\restr{d}{A \times A}$ coincides with $d$ on $A$, it immediately satisfies the metric axioms.
    It remains to show that its corresponding metric topology on $A$ is the same as the topology on $A$ as a subspace of $X$.

    Let $B_d(x, \epsilon)$ be an open ball in $X$ centered at some point $x \in A \subset X$ with $\epsilon > 0$.
    Let $U = A \cap B_d(x, \epsilon)$ be a basic open set in the subspace topology on $A$.
    Then the open ball $B_{\restr{d}{A \times A}}(x, \epsilon)$ in the metric topology on $A$ induced by $\restr{d}{A \times A}$ satisfies
    %
    \begin{align*}
      B_{\restr{d}{A \times A}}(x, \epsilon) &= \{ y \in X \mid \restr{d}{A \times A}(x, y) < \epsilon \} \\
                                    &= \{ y \in A \mid d(x, y) < \epsilon \} \\
                                    &= A \cap B_d(x, \epsilon) \\
                                    &= U
    \end{align*}
    %
    and is, thus, open in the subspace topology on $A$.
    It follows that the metric topology on $A$ induced by $\restr{d}{A \times A}$ is finer than the subspace topology on $A$.
    
    Conversely, given an open ball $B_{\restr{d}{A \times A}}(x, \epsilon)$ in the metric topology on $A$ centered at some point $x \in A$ with $\epsilon > 0$, we can always write
    %
    \begin{equation*}
      A \cap B_d(x, \epsilon) = B_{\restr{d}{A \times A}}(x, \epsilon),
    \end{equation*}
    %
    showing that it is in fact a basic open set in the subspace topology on $A$.
    It follows that the subspace topology on $A$ is finer than the metric topology on $A$ induced by $\restr{d}{A \times A}$.
    Thus, the topologies are equivalent and so $\restr{d}{A \times A}$ is a metric for the subspace topology on $A$.
  \end{proof}
\end{solution}
\newpage


\begin{exercise}[ID=2.21.2]
  Let $X$ and $Y$ be metric spaces with metrics $d_X$ and $d_Y$, respectively.
  Let $f: X \rightarrow Y$ have the property that for every pair of points $x_1, x_2$ of $X$,
  %
  \begin{equation*}
    d_Y(f(x_1), f(x_2)) = d_X(x_1, x_2).
  \end{equation*}
  %
  Show that $f$ is an imbedding.
  It is called an \textit{\textbf{isometric imbedding}} of $X$ in $Y$.
\end{exercise}
%
\begin{solution}
  \begin{proof}
    Let $Z$ denote the iamge set $f(X)$, considered as a subspace of $Y$.
    We wish to show that the function $f': X \rightarrow Z$ obtained by restricting the range of $f$ is a homeomorphism, \textit{i.e.}, it is a continuous bijective function with continuous inverse.
    In the following, let $d_Z = \restr{d_Y}{f(X) \times f(X)}$.

    \subsubsection*{$f'$ is a bijection}
    %
    Let $x_1, x_2 \in X$.
    Then
    %
    \begin{equation*}
      x_1 \neq x_2 \implies d_Z(f'(x_1), f'(x_2)) = d_X(x_1, x_2) > 0 \implies  f'(x_1) \neq f'(x_2).
    \end{equation*}
    %
    Thus, $f'$ is injective.

    By definition, $f'$ is a surjective function from $X$ to $Z$, thus, it is a bijection.

    \subsubsection*{$f'$ is continuous}
    %
    Let $x_1 \in X$ and $\epsilon > 0$.
    Then for $\delta = \epsilon$,
    %
    \begin{alignat*}{2}
      d_X(x_1, x_2) < \delta \implies d_Z(f'(x_1), f'(x_2)) = d_X(x_1, x_2) < \epsilon.
    \end{alignat*}
    %
    Thus, $f'$ is continuous.

    \subsubsection*{$f^{\prime -1}$ is continuous}
    %
    Let $y_1 \in Z$ and $\epsilon > 0$.
    We show that for $\delta = \epsilon$,
    %
    \begin{equation*}
      d_Z(y_1, y_2) < \delta \implies d_X(f^{\prime -1}(y_1), f^{\prime -1}(y_2)) < \epsilon.
    \end{equation*}
    %
    Since $f'$ is a bijection, there exist $x_1, x_2 \in X$ such that $y_1 = f'(x_1)$ and $y_2 = f'(x_2)$.
    Then for $d_Z(y_1, y_2) < \delta$ we have
    %
    \begin{equation*}
      d_X(f^{\prime -1}(y_1), f^{\prime -1}(y_2)) = d_X(x_1, x_2) = d_Z(f'(x_1), f'(x_2)) = d_Z(y_1, y_2) < \epsilon.
    \end{equation*}
    %
    It follows that $f^{\prime -1}$ is continuous and so $f'$ is a homeomorphism.
    Thus, $f$ is an imbedding of $X$ in $Y$.
  \end{proof}
\end{solution}
\newpage

\begin{exercise}[ID=2.21.3]
  Let $X_n$ be a metric space with metric $d_n$, for $n \in \ZZ_+$.
  \begin{enumerate}[label={(\alph*)}, align=left, leftmargin=\parindent, listparindent=\parindent, labelwidth=0pt, itemindent=!]
    \item
      Show that
      %
      \begin{equation*}
        \rho(x, y) = \max\{ d_1(x_1, y_1), \ldots, d_n(x_n, y_n) \}
      \end{equation*}
      %
      is a metric for the product space $X_1 \times \cdots \times X_n$.
    \item
      Let $\overline{d}_i = \min\{ d_i, 1 \}$.
      Show that
      %
      \begin{equation*}
        D(x, y) = \sup\{ \overline{d}_i(x_i, y_i) / i \}
      \end{equation*}
      %
      is a metric for the product space $\prod X_i$.
  \end{enumerate}
\end{exercise}
%
\begin{solution}
  \begin{enumerate}[label={(\alph*)}, align=left, leftmargin=\parindent, listparindent=\parindent, labelwidth=0pt, itemindent=!]
    \item
      Show that
      %
      \begin{equation*}
        \rho(x, y) = \max\{ d_1(x_1, y_1), \ldots, d_n(x_n, y_n) \}
      \end{equation*}
      %
      is a metric for the product space $X_1 \times \cdots \times X_n$.
  \end{enumerate}
  \begin{proof}~

    \subsubsection*{$\rho$ as a metric on the product space}
    %
    We begin by showing that $\rho$ satisfies the criteria of a metric.
    Let $x, y, z \in X_1 \times \cdots \times X_n$.
    \bigskip

    \noindent 1. Since for all $i \in \{1, \ldots, n\}$, $d_i$ is a metric, we have that
    %
    \begin{equation*}
      d_i(x_i, y_i) \geq 0
    \end{equation*}
    %
    with equality if and only if $x_i = y_i$.
    It immediately follows that $\rho(x, y) = 0$ if and only if $x_i = y_i$ for all $i \in \{1, \ldots, n\}$, which is true only if $x = y$.

    Suppose, instead, that $x \neq y$.
    Then there exists $i \in \{1, \ldots, n\}$ such that $x_i \neq y_i$.
    It follows that $d_i(x_i, y_i) > 0$ and so $\rho(x, y) \geq d_i(x_i, y_i) > 0$.

    Thus, $\rho(x, y) \geq 0$ with equality if and only if $x = y$.
    \bigskip

    \noindent 2. For all $i \in \{1, \ldots, n\}$,
    %
    \begin{equation*}
      d_i(x_i, y_i) = d_i(y_i, x_i).
    \end{equation*}
    %
    Thus,
    %
    \begin{align*}
      \rho(x, y)  &= \max\{ d_1(x_1, y_1), \ldots, d_n(x_n, y_n) \} \\
                  &= \max\{ d_1(y_1, x_1), \ldots, d_n(y_n, x_n) \} \\
                  &= \rho(y, x).
    \end{align*}
    \bigskip

    \noindent 3. Since for all $i \in \{1, \ldots, n\}$,
    %
    \begin{equation*}
      d_i(x_i, z_i) \leq d_i(x_i, y_i) + d_i(y_i, z_i)
    \end{equation*}
    %
    it follows that
    %
    \begin{align*}
      \rho(x, z)  &= \max\{ d_i(x_i, z_i) \mid i \in \{1, \ldots, n\} \} \\
                  &\leq \max\{ d_i(x_i, y_i) + d_i(y_i, z_i) \mid i \in \{1, \ldots, n\} \} \\
                  &\leq \max\{ d_i(x_i, y_i) \mid i \in \{1, \ldots, n\} \} + \max\{ d_i(y_i, z_i) \mid i \in \{1, \ldots, n\} \} \\
                  &= \rho(x, y) + \rho(y, z).
    \end{align*}
    %

    \subsubsection*{Equivalence of metric and product topologies}
    %
    Let $x \in X_1 \times \cdots \times X_n$ and let $U = \prod_i U_i$ be a basic open neighborhood of $x$ in the product topology, with $U_i = B_{d_i}(x_i, \epsilon_i)$ and $\epsilon_i > 0$, for all $i \in \{1, \ldots, n\}$.
    Let $\delta = \min\{\epsilon_1, \ldots, \epsilon_n\}$ and consider $y \in B_\rho(x, \delta)$.
    Then
    %
    \begin{alignat*}{2}
      &           && \rho(x, y) < \delta \\
      & \implies  && \max\{ d_i(x_i, y_i) \mid i \in \{1, \ldots, n\} \} < \delta \\
      & \implies  && \forall i \in \{1, \ldots, n\},~ d_i(x_i, y_i) < \delta \\
      & \implies  && \forall i \in \{1, \ldots, n\},~ y_i \in B_{d_i}(x_i, \delta).
    \end{alignat*}
    %
    Since for all $i \in \{1, \ldots, n\}$, $\delta \leq \epsilon_i$, we have that $B_{d_i}(x_i, \delta) \subset B_{d_i}(x_i, \epsilon_i)$.
    It follows that $y \in U$ and so $B_\rho(x, \delta) \subset U$.
    Thus, the metric topology induced by $\rho$ is finer than the product topology on $X_1 \times \cdots \times X_n$.

    Conversely, let $x \in X_1 \times \cdots \times X_n$ and let $B_\rho(x, \epsilon)$, with $\epsilon > 0$, be a basic open neighborhood of $x$ in the metric topology induced by $\rho$.
    Let $\delta = \epsilon$ and consider $y \in U = \prod_i U_i$ with $U_i = B_{d_i}(x_i, \delta)$ for all $i \in \{1, \ldots, n\}$.
    Then
    %
    \begin{alignat*}{2}
      &           && \forall i \in \{1, \ldots, n\},~ d_i(x_i, y_i) < \delta \\
      & \implies  && \rho(x, y) = \max\{ d_i(x_i, y_i) \mid i \in \{1, \ldots, n\} \} < \delta. 
    \end{alignat*}
    %
    It follows that $y \in B_\rho(x, \delta) = B_\rho(x, \epsilon)$ and so $U \subset B_\rho(x, \epsilon)$.
    Therefore, the product topology is finer than the metric topology induced by $\rho$ on $X_1 \times \cdots \times X_n$.
    Thus, the topologies are equivalent and so $\rho$ is a metric for the product space $X_1 \times \cdots \times X_n$.
  \end{proof}
  \bigskip

  \begin{enumerate}[label={(\alph*)}, align=left, leftmargin=\parindent, listparindent=\parindent, labelwidth=0pt, itemindent=!]
    \addtocounter{enumi}{1} 
    \item
      Let $\overline{d}_i = \min\{ d_i, 1 \}$.
      Show that
      %
      \begin{equation*}
        D(x, y) = \sup\{ \overline{d}_i(x_i, y_i) / i \}
      \end{equation*}
      %
      is a metric for the product space $\prod X_i$.
  \end{enumerate}
  \begin{proof}~

    \subsubsection*{$D$ as a metric on the product space}
    %
    We begin by showing that $D$ satisfies the criteria of a metric.
    Let $x, y, z \in \prod_i X_i$.
    \bigskip

    \noindent 1. Since for all $i \in \{1, \ldots, n\}$, $\overline{d}_i$ is a metric, we have that
    %
    \begin{equation*}
      \overline{d}_i(x_i, y_i) \geq 0
    \end{equation*}
    %
    with equality if and only if $x_i = y_i$.
    Since $D(x, y) \geq \overline{d}_i(x_i, y_i)$ for all $i \in \ZZ_+$, it follows that $D(x, y) \geq 0$ and, furthermore, that $D(x, y) = 0$ if and only only if $x_i = y_i$ for all $i \in \ZZ_+$, which is true if $x = y$.

    Thus, $D(x, y) \geq 0$ with equality if and only if $x = y$.
    \bigskip

    \noindent 2. For all $i \in \ZZ_+$,
    %
    \begin{equation*}
      \overline{d}_i(x_i, y_i) = \overline{d}_i(y_i, x_i).
    \end{equation*}
    %
    Thus,
    %
    \begin{align*}
      D(x, y) &= \sup\{\overline{d}_i(x_i, y_i) / i \mid i \in \ZZ_+\} \\
              &= \sup\{\overline{d}_i(y_i, x_i) / i \mid i \in \ZZ_+\} \\
              &= D(y, x)
    \end{align*}
    \bigskip

    \noindent 3. Since for all $i \in \ZZ_+$,
    %
    \begin{equation*}
      \overline{d}_i(x_i, z_i) \leq \overline{d}_i(x_i, y_i) + \overline{d}_i(y_i, z_i)
    \end{equation*}
    %
    it follows that
    %
    \begin{align*}
      D(x, z) &= \sup\{\overline{d}_i(x_i, z_i) / i \mid i \in \ZZ_+\} \\
              &\leq \sup\{\overline{d}_i(x_i, y_i) / i + \overline{d}_i(y_i, z_i) / i \mid i \in \ZZ_+\} \\
              &\leq \sup\{\overline{d}_i(x_i, y_i) / i \mid i \in \ZZ_+\} + \sup\{\overline{d}_i(y_i, z_i) / i \mid i \in \ZZ_+\} \\
              &= D(x, y) + D(y, z).
    \end{align*}
    %

    \subsubsection*{Equivalence of metric and product topologies}
    %
    Let $x \in \prod X_i$ and let $U = \prod U_i$ be a basic open neighborhood of $x$ in the product topology, with
    %
    \begin{equation*}
      U_i = B_{d_i}(x_i, \epsilon_i),
    \end{equation*}
    %
    $\epsilon_i > 0$, for finitely many $i \in J \subset \ZZ_+$, and $U_i = \RR$ otherwise.

    Define $\tilde{\epsilon} := \min\{\epsilon_i \mid i \in J\}$ and $i_0 := \max{J}$.
    Let $\delta = \min(\tilde{\epsilon} / i_0,~ 1)$ and consider $y \in B_D(x, \delta)$.
    Then for all $i \in J$,
    %
    \begin{equation*}
      \begin{alignedat}{2}
        &           && \min\big(d_i(x_i, y_i),~ 1) / i < \tilde{\epsilon} / i_0 \\
        & \implies  && \min\big(d_i(x_i, y_i),~ 1) <  (i / i_0)~ \tilde{\epsilon} \\
        & \implies  && \min\big(d_i(x_i, y_i),~ 1) <  \tilde{\epsilon},
      \end{alignedat}
    \end{equation*}
    %
    since, by definition, $i \leq i_0$.
    Furthermore, since by definition $\tilde{\epsilon} \leq \epsilon_i$, we have that
    %
    \begin{equation*}
      \begin{alignedat}{2}
        &           && \min\big(d_i(x_i, y_i),~ 1) < \epsilon_i \\
        & \implies  && d_i(x_i, y_i) < \epsilon_i,
      \end{alignedat}
    \end{equation*}
    from which it we see that $y_i \in B_{d_i}(x, \epsilon_i) = U_i$.
    For $i \not\in J$, we further have that $y_i \in \RR = U_i$.
    It follows that $y \in U$ and so $B_D(x, \delta) \subset U$.
    Thus, the metric topology induced by $D$ is finer than the product topology on $\prod X_i$.

    Conversely, let $x \in \prod X_i$ and let $B_D(x, \epsilon)$, with $\epsilon > 0$, be a basic open neighborhood of $x$ in the metric topology induced by $D$.
    For any $\epsilon > 0$, there exists a positive integer $N$ such that for all $i \geq N$, $i \epsilon > 1$.

    Let $y \in U = \prod U_i$, where $U_i = B_{d_i}(x_i, \delta_i)$ with $\delta_i = i \epsilon$ for $i \in J = \{1, \ldots, N + 1\}$, and $U_i = \RR$ for i $\not\in J$.
    Then for all $i \in J$,
    %
    \begin{equation*}
      \begin{alignedat}{2}
        &           && d_i(x_i, y_i) < i \epsilon \\
        & \implies  && \overline{d}_i(x_i, y_i) < i \epsilon \\
        & \implies  && \overline{d}_i(x_i, y_i) / i < \epsilon.
      \end{alignedat}
    \end{equation*}
    %
    Since $J$ is finite, $\sup\{\overline{d}_i(x_i, y_i) / i\} = \max\{\overline{d}_i(x_i, y_i) / i\} < \epsilon$.
    For $i \not\in J$, we have that $i \geq N$ and so
    %
    \begin{equation*}
        \overline{d}_i(x_i, y_i) \leq 1 < i \epsilon
    \end{equation*}
    %
    since $N$ was chosen such that $i \epsilon > 1$.
    Therefore,
    %
    \begin{equation*}
      \overline{d}_i(x_i, y_i) / i \leq 1 / i < \epsilon,
    \end{equation*}
    from which we see that
    %
    \begin{equation*}
      \sup\{\overline{d}_i(x_i, y_i) / i\} \leq 1 / i < \epsilon.
    \end{equation*}
    %
    It follows that $y \in B_D(x, \epsilon)$ and so $U \subset B_D(x, \epsilon)$.
    Therefore, the product topology is finer than the metric topology induced by $D$ on $\prod X_i$.
    Thus, the topologies are equivalent and so $D$ is a metric for the product space $\prod X_i$.
  \end{proof}
\end{solution}
\newpage

\begin{exercise}[ID=2.21.5]
  \textit{Theorem.} Let $x_n \rightarrow x$ and $y_n \rightarrow y$ in the space $\RR$.
  Then
  %
  \begin{gather*}
    x_n + y_n \rightarrow x + y, \\
      x_n - y_n \rightarrow x - y, \\
      x_n y_n \rightarrow x y,
  \end{gather*}
  %
  and provided that each $y_n \neq 0$ and $y \neq 0$,
  %
  \begin{equation*}
    x_n / y_n \rightarrow x / y.
  \end{equation*}
  %
\end{exercise}
%
\begin{solution}
  \begin{proof}
    Let $f: \RR \times \RR \rightarrow \RR$ be any of the above arithmetic operations and assume that in the case that $f$ is division, then $y_n \neq 0$ for all $n$ and that $y \neq 0$.
    Let $U \subset \RR$ be an arbitrary open set containing $f(x, y)$.
    By Lemma 2.21.4, $f: \RR \times \RR \rightarrow \RR$ is a continuous function.
    Thus, we have that $f\inv(U)$ is open in $\RR \times \RR$.
    From Section 2.19 we have seen that if $x_n \rightarrow x$ and $y_n \rightarrow y$, then $(x_n, y_n) \rightarrow (x, y)$.
    Therefore, since $f\inv(U)$ is an open neighborhood of $(x, y)$, there exists $N \in \ZZ_+$ such that $\forall n \geq N$, $(x_n, y_n) \in f\inv(U)$.
    It follows that $\forall n \geq N$, $f(x_n, y_n) \in U$.
    Thus, $f(x_n, y_n) \rightarrow f(x, y)$.
  \end{proof}
\end{solution}
\newpage

\begin{exercise}[ID=2.21.6]
  Define $f_n: [0, 1] \rightarrow \RR$ by the equation $f_n(x) = x^n$.
  Show that the sequence $(f_n(x))$ converges for each $x \in [0, 1]$, but that the sequence $(f_n)$ does not converge uniformly.
\end{exercise}
%
\begin{solution}
  \begin{proof}~
    \subsubsection*{Pointwise convergence of $(f_n(x))$}
    %%
    Let $0 \leq x < 1$ and let $\epsilon > 0$.
    Since we have seen in Section 1.4 that for $0 \leq x < 1$, $\inf\{x^n \mid n \in \ZZ_+\} = 0$, it cannot be that $\epsilon$ is a lower bound for $a^n$.
    It follows that $\exists N \in \ZZ^+$ such that $0 \leq x^N < \epsilon$.
    Since $0 \leq x < 1$, for any $n \geq N$, we have that $0 \leq x^n \leq x^N < \epsilon$.
    Thus, there exists an $N \in \ZZ_+$ such that for all $n \geq N$, $x^n \in [0, \epsilon)$.
    Since $[0, \epsilon)$ is an arbitrary basic open set containing 0, we have that $f_n(x) = x^n$ converges to 0 for $0 \leq x < 1$.

    Consider now the case where $x = 1$.
    Then for all $n \geq 1$, $x^n = 1$ is contained in any basic open set $(b, 1]$, $0 \leq b < 1$, which contains 1.
    Thus, $f_n(x)$ converges to 1 for $x = 1$.
    Therefore, $(f_n(x))$ converges for each $x \in [0, 1]$.
    In particular,
    %
    \begin{equation*}
      (f_n(x)) \rightarrow
      \begin{cases}
        0 & \text{if } 0 \leq x < 1, \\
        1 & \text{if } x = 1.
      \end{cases}
    \end{equation*}
    %

    \subsubsection*{Failure of uniform convergence of $(f_n)$}
    %
    We will show that given $\epsilon > 0$, there exists no positive integer $N$ such that $d(f_n(x), f(x)) < \epsilon$ for all $n \geq N$ and all $x \in [0, 1]$, where
    %
    \begin{equation*}
      f(x) =
      \begin{cases}
        0 & \text{if } 0 \leq x < 1, \\
        1 & \text{if } x = 1.
      \end{cases}
    \end{equation*}
    %
    In particular, we will show that for any $\epsilon > 0$ and any $N \in \ZZ_+$, there exists $x \in [0, 1)$ such that $d(f_N(x), f(x) \geq \epsilon$.
    We proceed by showing that $\sup\{x^N \mid x \in [0, 1)\} = 1$ for all $N \in \ZZ_+$.

    For all $x \in [0, 1)$, $x < 1 \implies x^N < 1^N = 1$.
    Thus, 1 is indeed an upper bound for $\{x^N \mid x \in [0, 1)\}$.
    To show that 1 is the \textit{least} upper bound, consider as a potential upper bound $1 - \epsilon$ with $\epsilon > 0$.
    If we choose $\epsilon \geq 1$, we have that $1 - \epsilon \leq 0 \leq x^N$ for any $x \in [0, 1)$.
    Thus, in this case $1 - \epsilon$ does not furnish an upper bound for our set.
    Choosing instead $0 < \epsilon < 1$ and letting $\delta = \epsilon / N$, we consider $x = 1 - \delta \in [0, 1)$.
    In Section 1.4 we have seen that for $0 < \delta < 1$, $(1 - \delta)^N \geq 1 - N \delta$.
    It follows that
    %
    \begin{equation*}
      x^N \geq 1 - N \delta = 1 - \epsilon.
    \end{equation*}
    %
    Thus, we find that $\sup\{x^N \mid x \in [0, 1)\} = 1$.

    It follows that $\forall \epsilon \in [0, 1)$, $\forall N \in \ZZ_+$, $\exists x \in [0, 1)$ such that $d(f_N(x), f(x)) = \abs{x^N} > \epsilon$.
    Thus, $(f_n)$ does not converge uniformly to $f$.
  \end{proof}
\end{solution}
\newpage

