\documentclass[a4paper,10pt]{article}

% \ExplSyntaxOn
% \cs_gset_eq:NN \xsim_disable_restore:n \use_none:n
% \ExplSyntaxOff
% \xsimsetup{save-exercises=false}

\usepackage{fullpage}
\usepackage[utf8]{inputenc}
\usepackage{amsmath,amsthm,amssymb}
\usepackage{mathrsfs}
\usepackage{mathtools}
\usepackage{enumitem}
\usepackage{xsim}

\usepackage{setspace}

\onehalfspacing

% Redefine \theexercise to allow manual numbering
\renewcommand{\theexercise}{\GetExerciseProperty{ID}}
\renewcommand{\thesolution}{}

%  Define a custom template for exercises
 \DeclareExerciseEnvironmentTemplate{custom}{
  \subsection*{\XSIMmixedcase{\GetExerciseName}~\GetExerciseProperty{ID}} 
 }{
   \par
 }

\xsimsetup{
  exercise/template = custom,
  solution/print = true
}

\newcommand{\bx}{\mathbf{x}}
\newcommand{\by}{\mathbf{y}}
\newcommand{\bz}{\mathbf{z}}
\newcommand{\bw}{\mathbf{w}}
\newcommand{\inv}{^{-1}}
\newcommand{\ZZ}{\mathbb{Z}}
\newcommand{\RR}{\mathbb{R}}

% Metadata
\title{Munkres Chapter 2}
\author{}
% \date{\today}

\begin{document}

\maketitle

\begin{exercise}[ID=2.18.10]
    Let $f: A \rightarrow B$ and $g: C \rightarrow D$ be continuous functions.
    Let us define a map $f \times g: A \times C \rightarrow B \times D$ by the equation
    $$ (f \times g)(a, c) = f(a) \times g(c).$$
    Show that $f \times g$ is continuous.
\end{exercise}

\begin{solution}
    \begin{proof}
        Let $(a, c) \in A \times C$ and let $W \subset B \times D$ be a neighborhood of $(f \times g)(a, c) = f(a) \times g(c)$.
        Under the product topology there exist open sets $U \subset B$ and $V \subset C$ such that the basic open set $U \times V \subset W$ is a neighborhood of $(f \times g)(a, c)$, so that $U$ is a neighborhood of $f(a)$ and $V$ is a neighborhood of $g(c)$.
        By Theorem 2.18.1, continuity of $f$ and $g$ implies there exist neighborhoods $U' \subset A$ of $a$ and $V' \subset C$ of $c$ such that $f(U') \subset U$ and $g(V') \subset V$.
        Thus, $U' \times V'$ is a neighborhood of $a \times c$ such that $(f \times g)(U' \times V') = f(U') \times g(V') \subset U' \times V' \subset W$, and so $f \times g$ is continuous.
    \end{proof}
\end{solution}
\newpage

\begin{exercise}[ID=2.18.11]
    Let $F: X \times Y \rightarrow Z$. We say that $F$ is {\bf continuous in each variable separately} if $\forall y_0 \in Y$, the map $h: X \rightarrow Z$ defined by $h(x) = F(x, y_0)$ is continuous and $\forall x_0 \in X$, the map $k: Y \rightarrow Z$ defined by $k(y) = F(x_0, y)$ is continuous.
    Show that if $F$ is continuous, then $F$ is continuous in each variable separately.
\end{exercise}

\begin{solution}
    \begin{proof}
        Let $(x, y_0) \in X \times Y$ and let $W$ be an open set containing $h(x) = F(x, y_0)$.
        By Theorem 2.18.1, continuity of $F$ implies the existence of an open set $U \times V \subset X \times Y$ containing $(x, y_0)$ such that $F(U \times V) \subset W$.
        Since $U \times \{y_0\} \subset U \times V$, $h(U) = F(U \times \{y_0\}) \subset F(U \times V) \subset W$.
        Thus, $h: X \rightarrow Z$ is continuous.

        By an analogous argument, so too is $k: Y \rightarrow Z$ continuous.
        Thus, $F$ is continuous in each variable separately.
    \end{proof}
\end{solution}
\newpage

\begin{exercise}[ID=2.18.13]
    Let $A \subset X$, $f: A \rightarrow Y$ be continuous and let $Y$ be Hausdorff.
    Show that if $f$ may be extended to a continuous function $g: \overline{A} \rightarrow Y$, then $g$ is uniquely determined by $f$.
\end{exercise}

\begin{solution}
    \begin{proof}
        Let $g: \overline{A} \rightarrow Y$ and $h: \overline{A} \rightarrow Y$ be continuous functions satisfying $x \in A \implies g(x) = h(x) = f(x)$.
        If $A$ is closed, then $A = \overline{A}$, $g(x) = h(x) = f(x)~ \forall x \in \overline{A}$ and we are done.
        Suppose instead that $A \neq \overline{A}$.
        Suppose by way of contradiction that there exists a limit point $x$ of $A$ not contained in $A$ such that $g(x) \neq h(x)$.
        Since $Y$ is Hausdorff, there exist disjoint open subsets $U$ and $V$ of $Y$ containing $g(x)$ and $h(x)$, respectively.
        By hypothesis, $g^{-1}(U)$ and $h^{-1}(V)$ are both open in $\overline{A}$.
        Since $x$ is a limit point of $A$, we have $g^{-1}(U) \cap A \neq \emptyset$ and $h^{-1}(V) \cap A \neq \emptyset$.
        As $g^{-1}(U)$ and $h^{-1}(V)$ are both open sets containing $x$, so too is their intersection an open set containing $x$.
        Since $x$ is a limit point of $A$, there must exist some $y \neq x$ contained in $g^{-1}(U) \cap h^{-1}(V) \cap A$.
        Then $g(y) \in U$ and $h(y) \in V$, but since $y \in A$, $g(y) = h(y) = f(y)$ and so $U \cap V \neq \emptyset$, contradicting our assumption that $g(x) \neq h(x)$.
        Thus, the extensions $g$ and $h$ of $f$ must be the same function.
    \end{proof}
\end{solution}
\newpage

\begin{exercise}[ID=2.19.1]
    Suppose the topology on each space $X_\alpha$ is given by a basis $\mathscr{B}_\alpha$.
    Show that the collection of all sets of the form
    $$\prod_{\alpha \in J} B_\alpha,$$
    where $B_\alpha \in \mathscr{B}_\alpha$ for each $\alpha$, will serve as a basis for the box topology on $\prod_{\alpha \in J} X_\alpha$.

    Furthermore, show that the collection of all sets of the same form, where $B_\alpha \in \mathscr{B}_\alpha$ for finitely many indices $\alpha$ and $B_\alpha = X_\alpha$ for all the remaining indices, will serve as a basis for the product topology on $\prod_{\alpha \in J} X_\alpha$.
\end{exercise}

\begin{solution}
    \begin{proof}
        We first demonstrate that the collection of such products of basis elements in either case is a basis for \textit{some} topology on $\prod X_\alpha$.
        Let $x = (x_\alpha) \in \prod X_\alpha$.
        For all $\alpha \in J$, $\mathscr{B}_\alpha$ forms a basis for the topology on $X_\alpha$ and so there exists a basis element $B_\alpha \in \mathscr{B}_\alpha$ containing $x_\alpha$.
        Since this holds true for all $\alpha$, the product of basis elements $\prod B_\alpha$, thus, contains $x$.
        If we require that only a finite number of $B_\alpha$ are not equal to $X_\alpha$, the product remains well-defined and obviously still contains $x$.

        Now suppose there are two such products $\prod B_\alpha$ and $\prod B'_\alpha$ containing $x$.
        Then $x \in \prod B_\alpha \cap B'_\alpha \implies$ for all $\alpha \in J$, $x_\alpha \in B_\alpha \cap B'_\alpha$.
        Since $\mathscr{B}_\alpha$ is a basis for the topology on $X_\alpha$, there exists another element $B''_\alpha \in \mathscr{B}_\alpha$ such that $x_\alpha \in B''_\alpha \subset B_\alpha \cap B'_\alpha$ and so $x \in \prod B''_\alpha$.
        Let $y = (y_\alpha) \in \prod B''_\alpha$.
        Then for all $\alpha \in J$, $y_\alpha \in B''_\alpha \implies y_\alpha \in B_\alpha \cap B'_\alpha \implies y \in \prod B_\alpha \cap B'_\alpha \implies \prod B''_\alpha \subset \prod B_\alpha \cap B'_\alpha$.
        If only a finite number of $B_\alpha$ and $B'_\alpha$ are not equal to $X_\alpha$, then we may also choose a finite number of $B''_\alpha \neq X_\alpha$.
        Thus, such products are seen to form a basis generating a topology on $\prod X_\alpha$.

        Given a set $V \subset \prod X_\alpha$ which is open under the box topology and a point $x = (x_\alpha) \in V$, by definition of the box topology given in Chapter 2.19 there is a basis element $\prod U_\alpha \subset V$ containing $x$, where $U_\alpha$ is open in $X_\alpha$ for all $\alpha \in J$.
        Because $\mathscr{B}_\alpha$ is a basis for $X_\alpha$ for all $\alpha \in J$, we can choose basis elements $B_\alpha \in \mathscr{B}_\alpha$ such that $x_\alpha \in B_\alpha \subset U_\alpha$ for each $\alpha$.
        Then $x \in \prod B_\alpha \subset \prod U_\alpha \subset V$.
        Thus, by Lemma 2.13.2, the collection of all such products $\prod B_\alpha$ is a basis for the box topology on $\prod X_\alpha$.

        Given now a set $W \subset \prod X_\alpha$ which is open under the product topology and a point $x = (x_\alpha) \in W$, by definition of the product topology given in Chapter 2.19 there is a basis element $\prod U_\alpha \subset W$ containing $x$, where $U_\alpha$ is open in $X_\alpha$ for all $\alpha \in J$ and $U_\alpha = X_\alpha$ except for finitely many values of $\alpha$.
        Let $J'$ denote the finite collection of indices $\alpha$ such that $U_\alpha \neq X_\alpha$.
        Because $\mathscr{B}_\alpha$ is a basis for $X_\alpha$ for all $\alpha \in J$, we can choose basis elements $B_\alpha \in \mathscr{B}_\alpha$ such that $x_\alpha \in B_\alpha \subset U_\alpha$ for all $\alpha \in J'$ and choose $B_\alpha = X_\alpha$ for the remaining $\alpha \in J \setminus J'$, clearly also satisfying $x_\alpha \in B_\alpha \subset U_\alpha$ since in this case $B_\alpha = U_\alpha = X_\alpha$.
        Then $x \in \prod B_\alpha \subset \prod U_\alpha \subset W$.
        Thus, by Lemma 2.13.2, the collection of all such products $\prod B_\alpha$, where $B_\alpha \neq X_\alpha$ only for a finite number of indices $\alpha \in J$, is a basis for the product topology on $\prod X_\alpha$.
    \end{proof}
\end{solution}
\newpage

\begin{exercise}[ID=2.19.2]
    Let $A_\alpha$ be a subspace of $X_\alpha$ for each $\alpha \in J$.
    Show that $\prod A_\alpha$ is a subspace of $\prod X_\alpha$ if both products are given the box topology, or if both products are given the product topology.
\end{exercise}

\begin{solution}
    \begin{proof}
        Let $\prod X_\alpha$ be equipped with the box topology and let $B = \prod B_\alpha$ be a basic open set in $\prod X_\alpha$, where $B_\alpha$ is a basic open set in $X_\alpha$ for all $\alpha \in J$.
        As $A_\alpha$ is a subspace of $X_\alpha$ for all $\alpha \in J$, by Lemma 2.16.1. the set $C_\alpha \coloneq B_\alpha \cap A_\alpha$ is a basic open set in the subspace topology on $A_\alpha$.
        If we equip $\prod A_\alpha$ with the box topology, we may write its basic open sets as $C \coloneq \prod C_\alpha = \prod B_\alpha \cap A_\alpha = B \cap \prod A_\alpha$.
        By Lemma 2.16.1, these are precisely the basic open sets of $\prod A_\alpha$ as a subspace of $\prod X_\alpha$.

        Now let $\prod X_\alpha$ be equipped with the product topology and let $B = \prod U_\alpha$ be a basic open set in $\prod X_\alpha$, where $U_\alpha = B_\alpha$ is a basic open set in $X_\alpha$ for all $\alpha \in J'$, where $J'$ is a finite subset of $J$, and $U_\alpha = X_\alpha$ for all $\alpha \in J \setminus J'$.
        As $A_\alpha$ is a subspace of $X_\alpha$ for all $\alpha \in J$, by Lemma 2.16.1 we have for all $\alpha \in J'$ the set $C_\alpha \coloneq U_\alpha \cap A_\alpha = B_\alpha  \cap A_\alpha$ is a basic open set in the subspace topology on $A_\alpha$, whereas for all $\alpha \in J \setminus J'$ the set $C_\alpha \coloneq U_\alpha \cap A_\alpha = X_\alpha \cap A_\alpha = A_\alpha$ is trivially open in $A_\alpha$.
        If we equip $\prod A_\alpha$ with the product topology, we may write its basic open sets as $C \coloneq \prod C_\alpha = \prod U_\alpha \cap A_\alpha = B \cap \prod A_\alpha$.
        By Lemma 2.16.1, these are precisely the basic open sets of $\prod A_\alpha$ as a subspace of $\prod X_\alpha$.
    \end{proof}
\end{solution}
\newpage

\begin{exercise}[ID=2.19.3]
    Show that if each space $X_\alpha$ is a Hausdorff space, then $\prod X_\alpha$ is a Hausdorff space in both the box and the product topologies.
\end{exercise}

\begin{solution}
    \begin{proof}
        Let $x, y \in \prod X_\alpha$ with $x \neq y$.
        Then we may fix some $\alpha' \in J$ such that $x_{\alpha'} \neq y_{\alpha'}$.
        For all $\alpha \neq \alpha'$, define open sets $U_\alpha = V_\alpha = X_\alpha$.
        By hypothesis, we can choose disjoint open sets $U_{\alpha'}, V_{\alpha'} \subset X_{\alpha'}$ containing $x_{\alpha'}$ and $y_{\alpha'}$, respectively.
        Then we have $x \in U \coloneq \prod U_\alpha$ and $y \in V \coloneq \prod V_\alpha$ with $U \cap V = \left(\prod_\alpha U_\alpha\right) \cap \left(\prod_\alpha V_\alpha\right) = \prod U_\alpha \cap V_\alpha = \emptyset$, since $U_{\alpha'} \cap V_{\alpha'} = \emptyset$.
        As $U_\alpha$ and $V_\alpha$ are both equal to $X_\alpha$ except for the case $\alpha = \alpha'$, the sets $U$ and $V$ are open in $\prod X_\alpha$ in both the box topology and the product topology.
        Thus, $\prod X_\alpha$ is a Hausdorff space in either topology.
    \end{proof}
\end{solution}
\newpage

\begin{exercise}[ID=2.19.4]
    {\bf Claim.} $(X_1 \times \ldots \times X_{n-1}) \times X_n$ is homeomorphic with $X_1 \times \ldots \times X_n$.
\end{exercise}

\begin{solution}
    \begin{proof}
        Define a function
        \begin{align*}
            f: (X_1 \times \ldots \times X_{n-1}) \times X_n & \rightarrow X_1 \times \ldots \times X_n \\
            ((x_1, \ldots, x_{n-1}), x_n)                    & \mapsto (x_1, \ldots, x_n).
        \end{align*}
        We have already proved in Exercise 1.5.2 that $f$ is a bijection, thus it remains only to be shown that $f$ and $f^{-1}$ are continuous.

        Let $U \coloneq U_1 \times \ldots \times U_n$ be a basic open set in $X_1 \times \ldots \times X_n$, where $U_i$ is open in $X_i$ for all $i \in \{1, \ldots, n\}$.
        The preimage set of $U$ is given by $f^{-1}(U) = (U_1 \times \ldots \times U_{n-1}) \times U_n$, which is a basic open set in $(X_1 \times \ldots \times X_{n-1}) \times X_n$.
        Thus, $f$ is continuous.

        Similarly, given a basic open set $V \coloneq (V_1 \times \ldots \times V_{n-1}) \times V_n$ in $(X_1 \times \ldots X_{n-1}) \times X_n$, where $V_i$ is open in $X_i$ for all $i \in \{1, \ldots, n\}$, its image set is given by $f(V) = V_1 \times \ldots \times V_n$, which is a basic open set in $X_1 \times \ldots \times X_n$.
        Thus, $f^{-1}$ is also continuous.
        Therefore, $(X_1 \times \ldots \times X_{n-1}) \times X_n$ is homeomorphic with $X_1 \times \ldots \times X_n$.
    \end{proof}
\end{solution}
\newpage

\begin{exercise}[ID=2.19.6]
    Let $\bx_1, \bx_2, \ldots$ be a sequence of points of the product space $\prod_{\alpha \in J} X_\alpha$.
    Show that this sequence converges to the point $\bx$ if and only if the sequence $\pi_\alpha(\bx_1), \pi_\alpha(\bx_2), \ldots$ converges to $\pi_\alpha(\bx)$ for each $\alpha$.
    Is this fact true if one uses the box topology instead of the product topology?
\end{exercise}

\begin{solution}
    \begin{proof}
        Assume that the sequence $\bx_1, \bx_2, \ldots$ converges to $\bx$.
        We will show that the sequence $\pi_\alpha(\bx_1),$\newline
        $\pi_\alpha(\bx_2), \ldots$ converges to $\pi_\alpha(\bx)$ for all $\alpha \in J$.
        Fix $\alpha \in J$ and let $U_\alpha$ be an open neighborhood of $\pi_\alpha(\bx)$.
        By definition of the product topology, $\pi_\alpha\inv(U_\alpha)$ is an open set in $\prod_\alpha X_\alpha$ and, since $\pi_\alpha(\bx) \in U_\alpha$, it is a neighborhood of $\bx$.
        Then $\exists N \in \ZZ_+$ such that $\bx_n \in \pi_\alpha\inv(U_\alpha)$, for all $n \geq N$.
        Thus, $\forall n \geq N$, $\pi_\alpha(\bx_n) \in U_\alpha$ and so the sequence $\pi_\alpha(\bx_1), \pi_\alpha(\bx_2), \ldots$ converges to $\pi_\alpha(\bx)$.
        Since $\alpha$ was chosen arbitrarily, this holds for all $\alpha \in J$.

        Conversely, assume that the sequence $\pi_\alpha(\bx_1), \pi_\alpha(\bx_2), \ldots$ converges to $\pi_\alpha(\bx)$, for all $\alpha \in J$.
        We will show that the sequence $\bx_1, \bx_2, \ldots$ converges to $\bx$.
        Let $U = \prod_\alpha U_\alpha$ be a basic open set containing $\bx$, where $U_\alpha$ is open in $X_\alpha$ for all $\alpha \in J$ and $U_\alpha \neq X_\alpha$ only for finitely many values of $\alpha$.
        Then $\forall \alpha \in J$, $\pi_\alpha(\bx) \in U_\alpha$, which implies that $\forall \alpha \in J,~ \exists N_\alpha \in \ZZ_+$ such that $\pi_\alpha(\bx_n) \in U_\alpha$ for all $n \geq N_\alpha$.
        Note that we may choose $N_\alpha = 1$ for all but the finitely many values of $\alpha$ for which $U_\alpha \neq X_\alpha$ and may, thus, define $N = \max\{N_\alpha\}_{\alpha \in J}$.
        Then $\pi_\alpha(\bx_n) \in U_\alpha,~ \forall n \geq N$ and for all $\alpha \in J$, implying that $\bx_n \in U,~ \forall n \geq N$.
        Thus, the sequence $\bx_1, \bx_2, \ldots$ converges to $\bx$.
    \end{proof}

    Note that the proof of the reverse implication relies on the fact that we must only choose a finite number of nontrivial $N_\alpha$.
    As this is not possible for an arbitrary open set in the box topology, the implication does not hold when choosing the box topology on $\prod_\alpha X_\alpha$.
\end{solution}
\newpage

\begin{exercise}[ID=2.19.7]
    Let $\mathbb{R}^\infty$ be the subset of $\mathbb{R}^\omega$ consisting of all sequences that are "eventually zero", that is, all sequences $(x_1, x_2, \ldots)$ such that $x_i \neq 0$ for only finitely many values of $i$.
    What is the closure of $\mathbb{R}^\infty$ in $\mathbb{R}^\omega$ in the box and product topologies?
\end{exercise}

\begin{solution}
    {\bf Claim.}  $\overline{\mathbb{R}^\infty} = \mathbb{R}^\infty$ in $\mathbb{R}^\omega$ equipped with the box topology.
    %
    \begin{proof}
        Let $\by = (b_i)$ be a sequence in $R^\omega$ (with the box topology) that is not eventually zero.
        Then there exists an infinite set of indices $J \subset \ZZ_+$ for which $y_i \neq 0$.
        For each such index $i \in J$, by the Hausdorff property of $\RR$ we can choose an open interval $U_i$ containing $y_i$ that does not contain $0$.
        For indices $i \notin J$, let $U_i = R$.
        Then the basic open set $U = \prod_{i\in\ZZ_+} U_i$ is a neighborhood of $\by$.
        But if $\bx = (x_i) \in \RR^\infty$ (so that $x_i = 0$ for all but finitely many $i$), then for all sufficiently large $i \in J$, we have $x_i = 0$ which does not lie in $U_i$ by construction.
        Hence $U \cap \RR^\infty = \emptyset$.
        It follows that any point in the closure of $\RR^\infty$ must be eventually zero.
        Since $R^\infty$ is obviously contained in its closure, we conclude that $\overline{\RR^\infty} = \RR^\infty$.
    \end{proof}
    \bigskip

    \noindent{\bf Claim.}  $\overline{\mathbb{R}^\infty} = \mathbb{R}^\omega$ in $\mathbb{R}^\omega$ equipped with the product topology.
    %
    \begin{proof}
        Let $\mathbf{y} \in \mathbb{R}^\omega$ and let $U = \prod_{i\in\mathbb{Z}_+} U_i$ be a basic open neighborhood of $\mathbf{y}$ in the product topology, where $U_i$ are open sets in $\mathbb{R}$ containing $y_i$ for all $i \in \mathbb{Z}_+$.
        By definition of the product topology, there exists $N \in \mathbb{Z}_+$ such that for all $i \geq N,~ U_i = \mathbb{R}$.
        Now define a sequence $\mathbf{x} = (x_i)$ by
        %
        \begin{equation*}
            x_i = \begin{cases}
                y_i & \text{if } i < N, \\
                0 & \text{if } i \geq N.
            \end{cases}
        \end{equation*}
        %
        By construction, we have that $\mathbf{x} \in \mathbb{R}^\infty$.
        Also, for $i < N$ we have $x_i = y_i \in U_i$ and for $i \geq N$ we have $x_i = 0 \in U_i = \RR$.
        Thus, $\bx \in U$.
        Since $U$ was an arbitrary basic open neighborhood of $\by$, we conclude that every such neighborhood intersects $\RR^\infty$.
        Therefore, $\by$ is in the closure of $\RR^\infty$.
        As $\by$ was arbitrary, it follows that $\overline{\RR^\infty} = \RR^\omega$.
    \end{proof}
\end{solution}
\newpage

\begin{exercise}[ID=2.19.8]
    Given sequences $(a_1, a_2, \ldots)$ and $(b_1, b_2, \ldots)$ of real numbers with $a_i > 0$ for all $i$, define $h: \RR^\omega \rightarrow \RR^\omega$ by the equation
    %
    \begin{equation*}
        h((x_1, x_2, \ldots)) = (a_1 x_1 + b_1, a_2 x_2 + b_2, \ldots).
    \end{equation*}
    %
    Show that if $\RR^\omega$ is given the product topology, $h$ is a homeomorphism of $\RR^\omega$ with itself.
\end{exercise}

\begin{solution}
    {\bf Claim.} If $\RR^\omega$ is given the product topology, $h$ is a homeomorphism of $\RR^\omega$ with itself.
    %
    \begin{proof}
        We start by showing that $h$ is a bijection.
        Let $\{\pi_i\}_{i\in\ZZ_+}$ denote the set of projection mappings from $\RR^\omega$ onto its $i$th factor and let $\bx, \by \in \RR^\omega$ satisfying $h(\bx) = h(\by)$.
        Then $h(\bx) = h(\by) \implies \forall i \in \ZZ_+,~ \pi_i(h(\bx)) = \pi_i(h(\by)) \implies a_i x_i + b_i = a_i y_i + b_i \implies x_i = y_i$.
        Since $x_i = y_i$ for all $i \in \ZZ_+$, we have $\bx = \by$.
        Thus, $h(\bx) = h(\by) \implies  \bx = \by$ and so $h$ is an injection.

        Let $\by \in \RR^\omega$.
        Since $\forall i \in \ZZ_+,~ a_i > 0,~ \exists \bx \in \RR^\omega$ with $x_i = (y_i - b_i) / a_i~, \forall i \in \ZZ_+$.
        Then $\forall i \in \ZZ_+$,
        %
        \begin{align*}
            \pi_i(h(\bx))   &= a_i x_i + b_i \\
                            &= a_i (y_i - b_i) / a_i + b_i \\
                            &= y_i,
        \end{align*}
        %
        and so $h(\bx) = \by$.
        Since $\by$ was arbitrary, we have that $\forall \by \in \RR^\omega, \exists \bx \in \RR^\omega$ such that $h(\bx) = \by$.
        Thus, $h$ is also a surjection, establishing it as a bijection of $\RR^\omega$ with itself.
        As a bijection, $h$ must have an inverse which we see from above must be defined as
        %
        \begin{equation*}
            h\inv(\by) = \left((y_1 - b_1) / a_1, (y_2 - b_2) / a_2, \ldots\right).
        \end{equation*}
        %

        We will now show that $h$ is continuous.
        Let $(\alpha, \beta)$ be an open interval in $\RR$ and let $S = \pi_i\inv\left((\alpha, \beta)\right)$ be a subbasis element for the product topology on $\RR^\omega$ for some fixed $i \in \ZZ_+$.
        Then the preimage of $S$ under $h$ is given by
        %
        \begin{align*}
            h\inv(S)    &= \{\bx \in \RR^\omega \mid h(\bx) \in S\} \\
                        &= \{\bx \in \RR^\omega \mid a_i x_i + b_i \in (\alpha, \beta)\} \\
                        &= \{\bx \in \RR^\omega \mid x_i \in \left((\alpha - b_i) / a_i, (\beta - b_i) / a_i\right)\} \\
                        &= \pi_i\inv\left((\alpha - b_i) / a_i, (\beta - b_i) / a_i\right).
        \end{align*}
        %
        Since $a_i > 0$, $\left((\alpha - b_i) / a_i, (\beta - b_i) / a_i\right)$ is a well-defined open interval in $\RR$, making $h\inv(S)$ another subbasis element for the product topology on $\RR^\omega$ and, thus, open.
        As $\alpha, \beta \in \RR$ and $i \in \ZZ_+$ are arbitrary, we have that the preimage of any subbasis element under $h$ is open.
        Since any basis element $B$ can be written as a finite intersection of such subbasis elements as $B = \bigcap_{i \in J} S_i$, where $J$ is some finite indexing set, we have
        %
        \begin{align*}
            h\inv(B)    &= h\inv\left(\bigcap_{i \in J} S_i\right) \\
                        &= \bigcap_{i \in J} h\inv(S_i).
        \end{align*}
        %
        Since $J$ is finite and $\forall i \in J,~ h\inv(S_i)$ is open, $h\inv(B)$ is open as a finite product of open sets.
        Thus, as $B$ is an arbitrary basis element for the product topology on $\RR^\omega$, we have that $h$ is continuous.

        To show that $h\inv$ is continuous, it suffices to show that for every subbasic open set $S = \pi_i\inv\left((\alpha, \beta)\right)$ in $\RR^\omega$, the set
        %
        \begin{equation*}
            (h\inv)\inv(S) = h(S)
        \end{equation*}
        %
        is open.
        We see from explicit calculation that
        %
        \begin{align*}
            h(S)    &= \{h(\bx) \mid \bx \in S\} \\
                    &= \{(a_1 x_1 + b_1, a_2 x_2 + b_2, \ldots) \mid \forall j \in \ZZ_+,~ x_j \in \RR,~ x_i \in (\alpha, \beta)\} \\
                    &= \{\by \in \RR^\omega \mid  (y_i - b_i) / a_i \in (\alpha, \beta)\} \\
                    &= \{\by \in \RR^\omega \mid y_i \in \left(a_i \alpha + b_i, a_i \beta + b_i\right)\} \\
                    &= \pi_i\inv\left((a_i \alpha + b_i, a_i \beta + b_i)\right),
        \end{align*}
        %
        which is another subbasis element for the product topology on $\RR^\omega$ and, therefore, open.
        Thus, we see that $h\inv$ is also continuous and so $h$ furnishes a homeomorphism between $\RR^\omega$ and itself.
    \end{proof}
\end{solution}
\newpage

\begin{exercise}[ID=2.19.10]
    Let $A$ be a set; let $\{X_\alpha\}_{\alpha \in J}$ be an indexed family of spaces; and let $\{f_\alpha\}_{\alpha \in J}$ be an indexed family of functions $f_\alpha: A \rightarrow X_\alpha$.

    \begin{enumerate}[label={(\alph*)}, align=left, leftmargin=\parindent, listparindent=\parindent, labelwidth=0pt, itemindent=!]
        \item Show there is a unique coarsest topology $\mathcal{T}$ on $A$ relative to which each of the functions $f_\alpha$ is continuous.
        \item Let
        %
        \begin{equation*}
            \mathcal{S}_\beta = \left\{f_\beta\inv(U_\beta) \mid U_\beta \text{ is open in } X_\beta\right\},
        \end{equation*}
        %
        and let $\mathcal{S} = \bigcup S_\beta$.
        Show that $\mathcal{S}$ is a subbasis for $\mathcal{T}$.
        \item Show that a map $g: Y \rightarrow A$ is continuous relative to $\mathcal{T}$ if and only if each map $f_\alpha \circ g$ is continuous.
        \item Let $f: A \rightarrow \prod X_\alpha$ be defined by the equation
        %
        \begin{equation*}
            f(a) = (f_\alpha(a))_{\alpha \in J}
        \end{equation*}
        %
        and let $Z$ denote the subspace $f(A)$ of the product space $\prod X_\alpha$.
        Show that the image under $f$ of each element of $\mathcal{T}$ is an open set in $Z$.
    \end{enumerate}
\end{exercise}

\begin{solution}
    \begin{enumerate}[label={(\alph*)}, align=left, leftmargin=\parindent, listparindent=\parindent, labelwidth=0pt, itemindent=!]
        \item Show there is a unique coarsest topology $\mathcal{T}$ on $A$ relative to which each of the functions $f_\alpha$ is continuous.
    \end{enumerate}
    %
    \begin{proof}
        Let $\mathcal{F} = \{\mathcal{T}' \mid \mathcal{T}'$ is a toplogy on $A$ and for each $\alpha \in J,~ f_\alpha: (A, \mathcal{T}') \rightarrow X_\alpha$ is continuous$\}$.
        Since, for example, the discrete topology on $A$ makes every function continuous, $\mathcal{F} \neq \emptyset$.
        Define
        %
        \begin{equation*}
            \mathcal{T} \coloneq \bigcap_{\mathcal{T}' \in \mathcal{F}} \mathcal{T}'.
        \end{equation*}
        %
        We claim that $\mathcal{T}$ is a topology on $A$ and is the unique coarsest topology relative to which every $f_\alpha$ is continuous.

        We first show that $\mathcal{T}$ is a topology on $A$.
        Since every $\mathcal{T}' \in \mathcal{F}$ is a topology, it must contain both $\emptyset$ and $A$, thus $\mathcal{T}$ must also contain both $\emptyset$ and $A$.
        Let $\{U_i\}_{i \in I}$ be an indexed family of sets, each of which is contained in $\mathcal{T}$.
        Since, by definition of $\mathcal{T}$, each $\mathcal{T}'$ also contains each set $U_i$, it also contains $\bigcup_{i \in I} U_i$.
        As this holds for each topology in $\mathcal{F}$, we have that $\bigcup_{i \in I} U_i \in \mathcal{T}$.
        By an analogous argument, if the index set $I$ is finite, then we have that each topology in $\mathcal{F}$ also contains the intersection of all $U_i$ so that $\bigcap_{i \in I} U_i \in \mathcal{T}$.
        Thus, we have shown that $\mathcal{T}$ is a topology on $A$.

        Next, we show that for each index $\alpha \in J$, $f_\alpha: (A, \mathcal{T}) \rightarrow X_\alpha$ is continuous.
        Let $U_\alpha$ be some open set in $X_\alpha$ and let $\mathcal{T}' \in \mathcal{F}$.
        By definition of $\mathcal{F},~ f_\alpha: (A, \mathcal{T}') \rightarrow X_\alpha$ is continuous and so we have that $f_\alpha\inv(U_\alpha) \in \mathcal{T}'$.
        As this must be true for any $\mathcal{T}' \in \mathcal{F}$, we have that $f_\alpha\inv(U_\alpha) \in \mathcal{T}$.
        Since the set $U_\alpha$ is arbitrary, we have that $f_\alpha$ is continuous with respect to the topology $\mathcal{T}$.

        Finally, we show that $\mathcal{T}$ is the coarsest such topology and is unique.
        Let $\mathcal{T}''$ be another topology on $A$ relative to which $f_\alpha$ is continuous for all $\alpha \in J$.
        Then, by definition, $\mathcal{T}'' \in \mathcal{F}$ and so $\mathcal{T} \subset \mathcal{T}''$, that is, $\mathcal{T}$ is coarser than $\mathcal{T}''$.
        Furthermore, the construction of $\mathcal{T}$ by intersection guarantees its uniqueness.

        Thus, there is a unique coarsest topology on $A$ relative to which $f_\alpha$ is continuous for all $\alpha \in J$. 
    \end{proof}
    \bigskip

    \begin{enumerate}[label={(\alph*)}, align=left, leftmargin=\parindent, listparindent=\parindent, labelwidth=0pt, itemindent=!]
        \addtocounter{enumi}{1} 
        \item Let
        %
        \begin{equation*}
            \mathcal{S}_\beta = \left\{f_\beta\inv(U_\beta) \mid U_\beta \text{ is open in } X_\beta\right\},
        \end{equation*}
        %
        and let $\mathcal{S} = \bigcup S_\beta$.
        Show that $\mathcal{S}$ is a subbasis for $\mathcal{T}$.
    \end{enumerate}
    %
    \begin{proof}
        In order for $\mathcal{S}$ to be a subbasis for $\mathcal{T}$, it must be that (1) the union of all sets contained in $\mathcal{S}$ must be equal to $A$, and (2) the topology $\mathcal{T}_\mathcal{S}$ which $\mathcal{S}$ generates, given by the collection of all unions of finite intersections of elements of $\mathcal{S}$, must be equal to $\mathcal{T}$.

        Since every function $f_\beta$ satisfies $f_\beta\inv(X_\beta) = A$, we see that $A \in S_\beta$ for each $\beta \in J$.
        Thus, the union of all elements of $\mathcal{S}$ must also equal $A$.
    
        Let $V \in \mathcal{S}$.
        By definition, for some $\beta \in J$, there is an open set $U_\beta \subset X_\beta$ such that $V = f_\beta\inv(U_\beta)$.
        Since $\mathcal{T}$ is a topology with respect to which $f_\beta$ is continuous, we have that $f_\beta\inv(U_\beta) \in \mathcal{T}$, and so $\mathcal{S} \subset \mathcal{T}$.
        The definition of a topology guarantees, therefore, that $\mathcal{T}_{\mathcal{S}} \subset \mathcal{T}$, that is, $\mathcal{T}_\mathcal{S}$ is coarser than $\mathcal{T}$.
    
        As $\mathcal{S}$ contains the preimage of all open subsets of $X_\beta$ under $f_\beta$ for all $\beta \in J$, we have that each of the functions $f_\beta$ are continuous with respect to $\mathcal{T}_\mathcal{S}$.
        Since we proved in (a) that $\mathcal{T}$ is the coarsest such topology, we have that $\mathcal{T} \subset \mathcal{T}_\mathcal{S}$.
        Thus, $\mathcal{T}$ is precisely the topology generated by $\mathcal{S}$.
    \end{proof}
    \bigskip

    \begin{enumerate}[label={(\alph*)}, align=left, leftmargin=\parindent, listparindent=\parindent, labelwidth=0pt, itemindent=!]
        \addtocounter{enumi}{2} 
        \item Show that a map $g: Y \rightarrow A$ is continuous relative to $\mathcal{T}$ if and only if each map $f_\alpha \circ g$ is continuous.
    \end{enumerate}
    %
    \begin{proof}
        Assume first that $g: Y \rightarrow A$ is continuous.
        Since each  $f_\alpha: A \rightarrow X_\alpha$ is continuous by hypothesis and the composition of two continuous maps is continuous (by Theorem 2.18.2), it immediately follows that $f_\alpha \circ g$ is continuous for every $\alpha \in J$.

        Conversely, assume that for every $\alpha \in J$, the composition $f_\alpha \circ g$ is continuous.
        Let $U_\alpha$ be any open subset of $X_\alpha$.
        Then, by the definition of continuity,
        %
        \begin{equation*}
            (f_\alpha \circ g)\inv(U_\alpha) = g\inv(f_\alpha\inv(U_\alpha))
        \end{equation*}
        %
        is open in $Y$.
        Since the collection of sets
        %
        \begin{equation*}
            \{f_\alpha\inv(U_\alpha) \mid \alpha \in J \text{ and } U_\alpha \text{ is open in } X_\alpha\}
        \end{equation*}
        forms a subbasis for the topology $\mathcal{T}$ on $A$, the fact that $g\inv(f_\alpha\inv(U_\alpha))$ is open in Y for every such set implies that $g$ is continuous.

        Thus, $g: Y \rightarrow A$ is continuous if and only if each map $f_\alpha \circ g: Y \rightarrow X_\alpha$ is continuous.
    \end{proof}
    \bigskip

    \begin{enumerate}[label={(\alph*)}, align=left, leftmargin=\parindent, listparindent=\parindent, labelwidth=0pt, itemindent=!]
        \addtocounter{enumi}{3} 
        \item Let $f: A \rightarrow \prod X_\alpha$ be defined by the equation
        %
        \begin{equation*}
            f(a) = (f_\alpha(a))_{\alpha \in J}
        \end{equation*}
        %
        and let $Z$ denote the subspace $f(A)$ of the product space $\prod X_\alpha$.
        Show that the image under $f$ of each element of $\mathcal{T}$ is an open set in $Z$.
    \end{enumerate}
    %
    \begin{proof}
        Let $U$ be a basic open set of $\prod X_\alpha$ so that its preimage under $f$, $f\inv(U)$, is a basic open set of $A$ with respect to the topology $\mathcal{T}$.
        The image of $f\inv(U)$ under $f$ is given by
        %
        \begin{align*}
            f(f\inv(U)) &= \{f(a) \mid a \in f\inv(U)\} \\
                        &= \{f(a) \mid a \in A \text{ and } f(a) \in U\} \\
                        &= \{x \in \prod X_\alpha \mid x \in f(A) \text{ and } x \in U\} \\
                        &= f(A) \cap U,
        \end{align*}
        %
        which, by definition of the subspace topology, is open in $Z$.

        Let $V$ be an arbitrary open set in $A$.
        Since $V$ can be expressed as a union of basic open sets $V = \bigcup_\alpha B_\alpha$, we have that its image under $f$,
        %
        \begin{align*}
            f(V)    &= f(\bigcup_\alpha B_\alpha) \\
                    &= \bigcup_\alpha f(B_\alpha),
        \end{align*}
        %
        is equal to a union of open sets in $Z$ and is, thus, open in $Z$.
        Therefore, the image under $f$ of each element of $\mathcal{T}$ is an open set in $Z$.
    \end{proof}
\end{solution}
\newpage

\begin{exercise}[ID=2.20.1]
    \begin{enumerate}[label={(\alph*)}, align=left, leftmargin=\parindent, listparindent=\parindent, labelwidth=0pt, itemindent=!]
        \item In $\RR^n$, define
        %
        \begin{equation*}
            d'(\bx, \by) = |x_1 - y_1| + \cdots + |x_n - y_n|.
        \end{equation*}
        %
        Show that $d'$ is a metric that induces the usual topology of $\RR^n$.
        \item  More generally, given $p \geq 1$, define
        %
        \begin{equation*}
            d'(\bx, \by) = \left[\sum_{i=1}^n |x_i - y_i|^p\right]^{1/p}
        \end{equation*}
        %
        for $\bx, \by \in \RR^n$.
        Assume that $d'$ is a metric.
        Show that it induces the usual topology on $\RR^n$.
    \end{enumerate}
\end{exercise}

\begin{solution}
    \begin{enumerate}[label={(\alph*)}, align=left, leftmargin=\parindent, listparindent=\parindent, labelwidth=0pt, itemindent=!]
        \item In $\RR^n$, define
        %
        \begin{equation*}
            d'(\bx, \by) = |x_1 - y_1| + \cdots + |x_n - y_n|.
        \end{equation*}
        %
        Show that $d'$ is a metric that induces the usual topology of $\RR^n$.
    \end{enumerate}
    %
    \begin{proof}
        To prove that $d'$ is a metric on $\RR^n$, we must show that 1) $\forall \bx, \by \in \RR^n,~ d'(\bx, \by) \geq 0$ with equality if and only if $\bx = \by$, 2) $d'(\bx, \by) = d'(\by, \bx)$ and 3) $d'$ satisfies the triangle inequality.

        1) Given that the standard metric on $\RR$ satisfies $|x_i - y_i| \geq 0$ for all $x_i, y_i \in \RR$, we have that $d'(\bx, \by) = |x_1 - y_1| + \cdots + |x_n - y_n| \geq 0$ for all $\bx, \by \in \RR^n$.

        Suppose that $d'(\bx, \by) = 0$.
        Then $|x_1 - y_1| + \cdots + |x_n - y_n| = 0$.
        As the standard metric on $\RR$ is positive definite, we have that every term in the sum is greater than or equal to zero, thus, they must all be zero.
        Therefore, $\forall i \in \{1, \ldots, n\},~ |x_i - y_i| = 0$ and so $x_i = y_i$.
        It follows that $\bx = \by$.

        Conversely, suppose that $\bx = \by$.
        Then
        %
        \begin{align*}
            d'(\bx, \by)    &= |x_1 - y_1| + \cdots + |x_n - y_n| \\
                            &= |x_1 - x_1| + \cdots + |x_n - x_n| \\
                            &= 0.
        \end{align*}
        %

        2) Let $\bx, \by \in \RR^n$.
        Then
        %
        \begin{align*}
            d'(\bx, \by)    &= |x_1 - y_1| + \cdots + |x_n - y_n| \\
                            &= |y_1 - x_1| + \cdots + |y_n - x_n| \\
                            &= d'(\by, \bx).
        \end{align*}
        %

        3) Let $\bx, \by, \bz \in \RR^n$.
        As the standard metric on $\RR$ satisfies the triangle inequality, we have that $\forall i \in \{1, \ldots, n\},~ |x_i - z_i| \leq |x_i - y_i| + |y_i - z_i|$.
        Therefore,
        %
        \begin{align*}
            |x_1 - z_1| + \cdots + |x_n - z_n|  &\leq \left(|x_1 - y_1| + |y_1 - z_1|\right) + \cdots + \left(|x_n - y_n| + |y_n - z_n|\right) \\
                                                &= \left(|x_1 - y_1| + \cdots + |x_n - y_n|\right) + \left(|y_1 - z_1| + \cdots + |y_n - z_n|\right).
        \end{align*}
        From here it follows that $d'(\bx, \bz) \leq d'(\bx, \by) + d'(\by, \bz)$.
        Thus, $d'$ is seen to be a metric on $\RR^n$.

        To show that $d'$ induces the usual topology on $\RR^n$, we will leverage Lemma 2.20.2, which states that given two metrics $d$ and $d'$ on a set $X$ which induce topologies $\mathcal{T}$ and $\mathcal{T}'$, respectively, the topology $\mathcal{T}'$ is finer than $\mathcal{T}$ if and only if for each $x \in X$ and each $\epsilon >0$, there exists a $\delta > 0$ such that $B_{d'}(x, \delta) \subset B_d(x, \epsilon)$.
        Let $\bx \in \RR^n$ and let $B_\rho(\bx, \epsilon)$ be an $\epsilon$-ball centered at $\bx$ in the square metric $\rho$.
        We show that the open ball $B_{d'}(\bx, \epsilon)$ is contained in $B_\rho(\bx, \epsilon)$.
        Let $\by \in B_{d'}(\bx, \epsilon)$.
        Then
        %
        \begin{align*}
                            & d'(\bx, \by) = |x_1 - y_1| + \cdots + |x_n - y_n|  < \epsilon \\
            \implies\qquad  & \forall i \in \{1, \ldots, n\},~ |x_i - y_i| < \epsilon \\
            \implies\qquad  & \max \{|x_i - y_i|\}_{i=1}^n < \epsilon \\
            \implies\qquad  & \rho(\bx, \by) < \epsilon \\
            \implies\qquad  & \by \in B_\rho(\bx, \epsilon).
        \end{align*}
        %
        Thus, $B_{d'}(\bx, \epsilon) \subset B_\rho(\bx, \epsilon)$ and so the metric topology induced by $d'$ is finer than the standard topology.

        On the other hand, let $B_{d'}(\bx, \epsilon)$ be an $\epsilon$-ball centered at $\bx$ in the $d'$ metric.
        We show that the open ball $B_\rho(\bx, \epsilon/n)$ is contained in $B_{d'}(\bx, \epsilon)$.
        Let $\by \in B_\rho(\bx, \epsilon/n)$.
        Then
        %
        \begin{align*}
                            & \rho(\bx, \by)  = \max\{|x_i - y_i|\}_{i=1}^n < \epsilon / n \\
            \implies\qquad  & \forall i \in \{1, \ldots, n\},~ |x_i - y_i| < \epsilon / n \\
            \implies\qquad  & |x_1 - y_1| + \cdots + |x_n - y_n| < \epsilon \\
            \implies\qquad  & d'(\bx, \by) < \epsilon \\
            \implies\qquad  & \by \in B_{d'}(\bx, \epsilon).
        \end{align*}
        %
        Thus, $B_\rho(\bx, \epsilon / n) \subset B_{d'}(\bx, \epsilon)$ and so the standard topology is finer than the topology induced by $d'$.

        We have, thus, shown that the $d'$ is a metric that induces the standard topology on $\RR^n$.
    \end{proof}
    \bigskip

    \begin{enumerate}[label={(\alph*)}, align=left, leftmargin=\parindent, listparindent=\parindent, labelwidth=0pt, itemindent=!]
        \addtocounter{enumi}{1} 
         \item  More generally, given $p \geq 1$, define
         %
         \begin{equation*}
             d'(\bx, \by) = \left[\sum_{i=1}^n |x_i - y_i|^p\right]^{1/p}
         \end{equation*}
         %
         for $\bx, \by \in \RR^n$.
         Assume that $d'$ is a metric.
         Show that it induces the usual topology on $\RR^n$.
    \end{enumerate}
    %
    \begin{proof}
        Let $\bx \in \RR^n$ and let $B_\rho(\bx, \epsilon)$ be an $\epsilon$-ball centered at $\bx$ in the square metric $\rho$.
        We show that the open ball $B_{d'}(\bx, \epsilon)$ is contained in $B_\rho(\bx, \epsilon)$.
        Let $\by \in B_{d'}(\bx, \epsilon)$.
        Then $d'(\bx, \by) = \left[\sum_{i=1}^n |x_i - y_i|^p\right]^{1/p} < \epsilon$ implies $\sum_{i=1}^n |x_i - y_i|^p < \epsilon^p$.
        Since $|x_i - y_i| \geq 0$ for all indices $i \in \{1, \ldots, n\}$, it follows that each term in the sum is greater than or equal to zero and, thus, each term must be less than $\epsilon^p$.
        From here we have that $\forall i \in \{1, \ldots, n\},~ |x_i - y_i| < \epsilon$ and, in particular, $\max \{|x_i - y_i|\}_{i=1}^n < \epsilon$. 
        Then $\rho(\bx, \by) < \epsilon$ and so $\by \in B_\rho(\bx, \epsilon)$.
        Thus, $B_{d'}(\bx, \epsilon) \subset B_\rho(\bx, \epsilon)$ and (by Theorem 2.20.2) the metric topology induced by $d'$ is finer than the standard topology.

        On the other hand, let $B_{d'}(\bx, \epsilon)$ be an $\epsilon$-ball centered at $\bx$ in the $d'$ metric.
        We show that the open ball $B_\rho(\bx, \epsilon/n)$ is contained in $B_{d'}(\bx, \epsilon)$.
        Let $\by \in B_\rho(\bx, \epsilon/n^{1/p})$.
        Then
        %
        \begin{align*}
                            & \rho(\bx, \by)  = \max\{|x_i - y_i|\}_{i=1}^n < \epsilon / n^{1/p} \\
            \implies\qquad  & \forall i \in \{1, \ldots, n\},~ |x_i - y_i| < \epsilon / n^{1/p} \\
            \implies\qquad  & \forall i \in \{1, \ldots, n\},~ |x_i - y_i|^p < \epsilon^p / n \\
            \implies\qquad  & \sum_{i=1}^n |x_i - y_i|^p < \epsilon^p \\
            \implies\qquad  & \left[\sum_{i=1}^n |x_i - y_i|^p\right]^{1/p} < \epsilon \\
            \implies\qquad  & d'(\bx, \by) < \epsilon \\
            \implies\qquad  & \by \in B_{d'}(\bx, \epsilon).
        \end{align*}
        %
        Thus, $B_\rho(\bx, \epsilon / n) \subset B_{d'}(\bx, \epsilon)$ and so (by Theorem 2.20.2) the standard topology is finer than the topology induced by $d'$.

        We have, thus, shown that the metric $d'$ induces the standard topology on $\RR^n$.
    \end{proof}
\end{solution}
\newpage

\begin{exercise}[ID=2.20.2]
    Show that $\RR \times \RR$ in the dictionary order topology is metrizable.
\end{exercise}

\begin{solution}
    \begin{proof}
    \end{proof}
\end{solution}
\newpage

\begin{exercise}[ID=2.20.3]
    Let $X$ be a metric space with metric $d$.
    \begin{enumerate}[label={(\alph*)}, align=left, leftmargin=\parindent, listparindent=\parindent, labelwidth=0pt, itemindent=!]
        \item Show that $d: X \times X \rightarrow \RR$ is continuous.
        \item Let $X'$ denote a space having the same underlying set as $X$.
        Show that if $d: X' \times X' \rightarrow \RR$ is continuous, then the topology of $X'$ is finer than the topology of $X$.
    \end{enumerate}
\end{exercise}

\begin{solution}
    \begin{enumerate}[label={(\alph*)}, align=left, leftmargin=\parindent, listparindent=\parindent, labelwidth=0pt, itemindent=!]
        \item Show that $d: X \times X \rightarrow \RR$ is continuous.
    \end{enumerate}
    %
    \begin{proof}
        Fix an arbitrary point $(x, y) \in X \times X$ and let $\epsilon > 0$.
        Consider the open interval
        %
        \begin{equation*}
            V = (d(x, y) - \epsilon, d(x, y) + \epsilon),~ \epsilon > 0
        \end{equation*}
        %
        in $\RR$, which is a basic open neighborhood of $d(x, y)$.  
        To prove that $d: X \times X \rightarrow \RR$ is continuous, we find an open neighborhood $U$ of $(x, y)$ in $X \times X$ such that $d(U) \subset V$.

        Let $U = B_d(x, \epsilon/2) \times B_d(y, \epsilon/2)$.
        Since $X$ is a metric space, the open balls with respect to $d$ form basic open sets in $X$.
        Being a product of such open balls, the set $U$ is open in the product space $X \times X$ and, thus, an open neighborhood of $(x, y)$.

        Let $(u, v) \in U$.
        From the triangle inequality, we have that
        %
        \begin{equation*}
            d(u, v) \leq d(u, y) + d(y, v)\qquad {\rm and }\qquad d(u, y) \leq d(u, x) + d(x, y).
        \end{equation*}
        %
        Combining these, we get
        %
        \begin{equation*}
            d(u, v) \leq d(x, y) + d(u, x) + d(y, v).
        \end{equation*}
        %
        Since $u \in B_d(x, \epsilon/2)$ and $v \in B_d(y, \epsilon/2)$, we have that $d(u, x) < \epsilon/2$ and $d(y, v) < \epsilon/2$, thus, $d(u, v) < d(x, y) + \epsilon$.

        Similarly, we have
        %
        \begin{equation*}
            d(x, y) \leq d(x, u) + d(u, y)\qquad {\rm and }\qquad d(u, y) \leq d(u, v) + d(v, y),
        \end{equation*}
        %
        giving
        %
        \begin{align*}
            d(x, y) &\leq d(u, v) + d(x, u) + d(v, y) < d(u, v) + \epsilon,
        \end{align*}
        %
        or, equivalently, $d(x, y) - \epsilon < d(u, v)$.
        Thus, $(u, v) \in U$ implies that $d(x, y) - \epsilon < d(u, v) < d(x, y) + \epsilon$, so that $d(u, v) \in V$.

        More generally, since $(u, v) \in U$ was chosen arbitrarily, we have $d(U) \subset V$ and, thus, $d: X \times X \rightarrow \RR$ is continuous.
    \end{proof}
    \bigskip

    \begin{enumerate}[label={(\alph*)}, align=left, leftmargin=\parindent, listparindent=\parindent, labelwidth=0pt, itemindent=!]
        \addtocounter{enumi}{1} 
        \item Let $X'$ denote a space having the same underlying set as $X$.
        Show that if $d: X' \times X' \rightarrow \RR$ is continuous, then the topology of $X'$ is finer than the topology of $X$.
    \end{enumerate}
    %
    \begin{proof}
        Fix an arbitrary point $x \in X$ and $\epsilon > 0$.
        Consider the open ball $B_d(x, \epsilon)$ as a basic open neighborhood of $x$ in the topology on $X$.
        We show that there exists an open neighborhood $U$ of $x$ in the topology on $X'$ contained entirely within $B_d(x, \epsilon)$.
        By Theorem 2.13.3, this is equivalent to the statement that the topology on $X'$ is finer than the topology on $X$.

        Define the function
        %
        \begin{equation*}
            f_x: X \rightarrow \RR,\quad y \mapsto d(x, y).
        \end{equation*}
        %
        As $d$ is continuous, it is continuous in each variable separately and so $f_x$ is continuous.
        This holds for both the topology on $X$ and on $X'$, by hypothesis.
        Since $f_x(x) = 0$, the open interval $(-\epsilon, \epsilon)$ is an open neighborhood of $f_x(x)$ in $\RR$.
        Continuity of $f_x$ means there exists an open neighborhood $U$ of $x$ in $X'$ such that $f_x(U) \subset (-\epsilon, \epsilon)$.

        Expressing $B_d(x, \epsilon)$ in terms of $f_x$ as
        %
        \begin{equation*}
            B_d(x, \epsilon) = \{y \in X \mid f_x(y) < \epsilon\},
        \end{equation*}
        %
        we see that $B_d(x, \epsilon) = f_x\inv((-\epsilon, \epsilon))$.
        Since $f(U) \subset (-\epsilon, \epsilon)$, it immediately follows that 
        %
        \begin{equation*}
            U \subset f_x\inv(f_x(U)) \subset f_x\inv((-\epsilon, \epsilon)) = B_d(x, \epsilon),
        \end{equation*}
        %
        as desired.
        Since $x$ and $\epsilon$ are arbitrary, we may construct such an open neighborhood of any $x \in X'$ contained in any open ball $B_d(x, \epsilon)$.
        Thus, the topology on $X'$ is finer than that on $X$.
    \end{proof}
\end{solution}
\newpage

\begin{exercise}[ID=2.20.4]
    Consider the product, uniform, and box topologies on $\RR^\omega$.
    \begin{enumerate}[label={(\alph*)}, align=left, leftmargin=\parindent, listparindent=\parindent, labelwidth=0pt, itemindent=!]
        \item In which topologies are the following functions from $\RR$ to $\RR^\omega$ continuous?
        %
        \begin{align*}
            f(t) &= (t, 2t, 3t, \ldots), \\
            g(t) &= (t, t, t, \ldots), \\
            h(t) &= (t, \tfrac{1}{2} t, \tfrac{1}{3} t, \ldots).
        \end{align*}
        %
        \item In which topologies do the following sequences converge?
        %
        \begin{alignat*}{2}
            & \bw_1 = (1, 1, 1, 1, \ldots),\qquad   & & \bx_1 = (1, 1, 1, 1, \ldots), \\
            & \bw_2 = (0, 2, 2, 2, \ldots),\qquad   & & \bx_2 = (0, \tfrac{1}{2}, \tfrac{1}{2}, \tfrac{1}{2}, \ldots), \\
            & \bw_3 = (0, 0, 3, 3, \ldots),\qquad   & & \bx_3 = (0, 0, \tfrac{1}{3}, \tfrac{1}{3}, \ldots), \\
            & \quad\ldots                           & & \quad\ldots \\
            & \by_1 = (1, 0, 0, 0, \ldots),\qquad   & & \bz_1 = (1, 1, 0, 0, \ldots), \\
            & \by_2 = (\tfrac{1}{2}, \tfrac{1}{2}, 0, 0),                       & & \bz_2 = (\tfrac{1}{2}, \tfrac{1}{2}, 0, 0, \ldots), \\
            & \by_3 = (\tfrac{1}{3}, \tfrac{1}{3}, \tfrac{1}{3}, 0, \ldots),    & & \bz_3 = (\tfrac{1}{3}, \tfrac{1}{3}, 0, 0, \ldots), \\
            & \quad\ldots                                                       & & \quad\ldots
        \end{alignat*}
        %
    \end{enumerate}
\end{exercise}

\begin{solution}
    \begin{enumerate}[label={(\alph*)}, align=left, leftmargin=\parindent, listparindent=\parindent, labelwidth=0pt, itemindent=!]
        \item In which topologies are the following functions from $\RR$ to $\RR^\omega$ continuous?
        %
        \begin{align*}
            f(t) &= (t, 2t, 3t, \ldots), \\
            g(t) &= (t, t, t, \ldots), \\
            h(t) &= (t, \tfrac{1}{2} t, \tfrac{1}{3} t, \ldots).
        \end{align*}
        %
    \end{enumerate}
    %
    \begin{proof}
    \end{proof}
\end{solution}

\end{document}