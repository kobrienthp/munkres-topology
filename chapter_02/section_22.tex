\subsection{The Quotient Topology}

\begin{exercise}[ID=2.22.2]
  \begin{enumerate}[label={(\alph*)}, align=left, leftmargin=\parindent, listparindent=\parindent, labelwidth=0pt, itemindent=!]
    \item
      Let $p: X \rightarrow Y$ be a continuous map.
      Show that if there is a continuous map $f: Y \rightarrow X$ such that $p \circ f$ equals the identity map of $Y$, then $p$ is a quotient map.
    \item
      If $A \subset X$, a \textbf{retraction} of $X$ onto $A$ is a continuous map $r: X \rightarrow A$ such that $r(a) = a$ for each $a \in A$.
      Show that a retraction is a quotient map.
  \end{enumerate}
\end{exercise}
%
\begin{solution}
  \begin{enumerate}[label={(\alph*)}, align=left, leftmargin=\parindent, listparindent=\parindent, labelwidth=0pt, itemindent=!]
    \item
      Let $p: X \rightarrow Y$ be a continuous map.
      Show that if there is a continuous map $f: Y \rightarrow X$ such that $p \circ f$ equals the identity map of $Y$, then $p$ is a quotient map.
  \end{enumerate}
  \begin{proof}
    Let $p: X \rightarrow Y$ be a continuous map.
    Assume there exists a continuous map $f: Y \rightarrow X$ such that $p \circ f: Y \rightarrow Y$ is the identity map of $Y$.
    Let $y \in Y$ and let $x = f(y)$.
    Since for all $y \in Y$, $(p \circ f) = y$, it follows that
    %
    \begin{align*}
      (p \circ f)(y) &= p(f(y)) \\
                     &= p(x) \\
                     &= y.
    \end{align*}
    %
    Thus, $p$ is surjective.

    Since $p$ is continuous by hypothesis, it remains to be shown that given a subset $U$ of $Y$, if $p\inv(U)$ is open in $X$, then $U$ is open in $Y$.
    Let $U$ be a subset of $Y$ such that $p\inv(U)$ is open in $X$.
    Since $p \circ f$ is the identity map on $Y$, we have that
    %
    \begin{equation*}
      (p \circ f)\inv(U) = U.
    \end{equation*}
    %
    By continuity of $f$, the set $f\inv(p\inv(U))$ is open in $Y$.
    But $f\inv(p\inv(U)) = (p \circ f)\inv(U) = U$.
    It follows that $U$ is open in $Y$.
    Thus, $p$ is a quotient map.
  \end{proof}
  \bigskip

  \begin{enumerate}[label={(\alph*)}, align=left, leftmargin=\parindent, listparindent=\parindent, labelwidth=0pt, itemindent=!]
    \addtocounter{enumi}{1} 
    \item
      If $A \subset X$, a \textbf{retraction} of $X$ onto $A$ is a continuous map $r: X \rightarrow A$ such that $r(a) = a$ for each $a \in A$.
      Show that a retraction is a quotient map.
  \end{enumerate}
  \begin{proof}
    Consider the inclusion map $\iota: A \rightarrow X$ from $A$ to $X$.
    The composite map $r \circ \iota: A \rightarrow A$ maps any element $a \in A$ to $r(\iota(a)) = r(a) = a$ and is, therefore, the identity map on $A$.
    Since, by Theorem 2.18.2, inclusion maps are continuous, it follows from the result of Exercise 2.22 (a) above that $r$ is a quotient map.
  \end{proof}
\end{solution}
\newpage

%
\begin{exercise}[ID=2.22.3]
  Let $\pi_1: \RR \times \RR \rightarrow \RR$ be projection on the first coordinate.
  Let $A$ be the subspace of $\RR \times \RR$ consisting of all points $x \times y$ for which either $x \geq 0$ or $y = 0$ (or both);
  let $q: A \rightarrow \RR$ be obtained by restricting $\pi_1$.
  Show that $q$ is a quotient map that is neither open nor closed.
\end{exercise}
  %
  \begin{solution}
    \begin{proof}~

      \subsubsection*{$q$ is a quotient map}
      %
      Given any real number $x$, we have the point $x \times 0 \in A$, the image of which under $q$ is equal to $x$.
      Thus, $q$ is surjective.

      Since the projection $\pi_1$ is continuous, so too is its restriction $q$ to the subspace $A$.
      Thus, it remains to be shown that any set $U \subset \RR$ with a preimage $q\inv(U)$ open in $A$ must itself be open in $\RR$.
      Let $U$ be a subset of $\RR$ with a preimage $q\inv(U)$ open in $A$.
      Let $x \in U$ and note that $x \times 0 \in q\inv(\{x\}) \subset q\inv(U)$.
      Since $q\inv(U)$ is open by hypothesis, it follows that there exists an $\epsilon > 0$ such that $V = \big(B(x, \epsilon) \times B(0, \epsilon)\big) \cap A$ is an open neighborhood of $x \times 0$ contained in $q\inv(U)$.
      Now
      %
      \begin{align*}
        q(V)  &= q\Big(\big(B(x, \epsilon) \times \{0\}\big) \cup \big(B(x, \epsilon) \cap [0, \infty) \times B(0, \epsilon)\big)\Big) \\
              &= q\Big(\big(B(x, \epsilon) \times \{0\}\big)\Big) \cup q\Big(\big(B(x, \epsilon) \cap [0, \infty) \times B(0, \epsilon)\big)\Big) \\
              &= B(x, \epsilon) \cup \big(B(x, \epsilon) \cap [0, \infty)\big) \\
              &= B(x, \epsilon)
      \end{align*}
      %
      is open in $\RR$.
      Since $V \subset q\inv(U)$, the image set $q(V)$ is an open neighborhood of $x$ contained in $U$.
      As $x$ was chosen arbitrarily, it follows that $U$ is open and, thus, $q$ is a quotient map from $A$ to $\RR$.
      \bigskip

      \subsubsection*{$q$ is neither open nor closed}
      %
      Let $0 < \epsilon < 1$ and consider the subset $U = \big(B(0, \epsilon) \times B(1, \epsilon)\big) \cap A = [0, \epsilon) \times B(1, \epsilon)$ of $A$.
      By construction, $U$ is open in $A$.
      However, $q(U) = [0, \epsilon)$ is not open in $\RR$.
      Thus, $q$ is not an open map.

      Consider now the continuous function $f: \RR \times \RR \rightarrow \RR$ given by $f(x, y) = x y - 1$.
      Since $\{0\}$ is closed in $\RR$, it follows by continuiuty of $f$ that
      %
      \begin{equation*}
        f\inv(\{0\}) = \{(x, y) \in \RR \times \RR \mid x \neq 0, y = 1 / x \}
      \end{equation*}
      %
      is closed in $\RR \times \RR$.
      Then the set $C \cap A = \{(x, y) \in \RR \times \RR \mid x > 0, y = 1 / x \}$ is closed in $A$.
      However, $q(C \cap A) = (0, \infty)$ is not closed in $\RR$.
      Thus, $q$ is not a closed map.
      Therefore, $q$ is a quotient map which is neither open nor closed.
    \end{proof}
  \end{solution}
